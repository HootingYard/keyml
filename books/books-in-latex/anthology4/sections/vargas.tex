\chapter{Vargas}

VARGAS, THE MOUSTACHIOED Mexican cop played by Charlton Heston in Orson Welles' classic \emph{Touch Of Evil} (1958), had a walk-on part in one of the more curious episodes of Dobson's life. Mystery surrounds the sudden appearance in Mexico of the out of print pamphleteer, although the oft-repeated story that he hove into view on the very spot where, a few seconds earlier, Ambrose Bierce had vanished, never to be seen again, can be discounted on the basis that it is chronologically incoherent. What makes the idea of Dobson-in-Mexico so perplexing is that he was notoriously unsuited to hot temperatures. Like Horace Walpole, he often had a bucket of ice close to hand, though not, of course, when he was in Mexico, for in the high noon of a sweltering day such as the one when he made his inexplicable appearance in that hot land such a bucketful would have melted away within seconds. As one might expect, Dobson was dressed inappropriately. Witnesses record that he was enwrapped in a fur muffler and some sort of reindeer-hide kagoul, his large ungainly feet slotted in to a pair of padded boots as worn by Alpinists. It would be helpful, I think, to have a goodly supply of words in Spanish to deploy when setting the scene. Alas, that language is not among my accomplishments, nor are most of the languages spoken and written in the world, so you will just have to picture the pamphleteer tottering unsteadily down a dusty road in a Mexican village. No one knew where he had come from, how he had got there, nor what the ramifications of his presence would be. And you can bet there would be ramifications. There always were with Dobson. He was not, to be blunt, the sort of pamphleteer who could shrink into the shadows, like a discarded and overlooked violet. If he did not always make a lot of noise, he somehow \emph{seemed} to. Things would crash around him, or he would disturb the kinds of animals that howl and screech, such as dogs and wolves and screech owls and monkeys, or he would set off clanging alarm bells. At least, such rackets occurred on his foreign trips, for when he was at home in his dismal backwater silence could sometimes reign for days on end, broken only by the endless thrumming of rain upon the roof. There was no rain here in Mexico, not today, just a broiling and battering sun in a sky innocent of clouds. Beneath it tottered Dobson, a pencil in one hand and a notebook in the other. Had anyone dared ask him what he was bent upon doing, he would have explained that he was engaged in what he liked to call 'pamphleteering in the field'. By this he did not mean the sort of field he was used to at home, with its cows and rusty farm equipment, but the abstract 'field' beloved of anthropologists and ethnographers, and indeed of all sorts of persons who charge about the place imagining that they are grappling with the 'authentic'. Dobson did not care two pins about authenticity, delusional or otherwise, but he fancied himself as the kind of pamphleteer who could wring a pamphlet from whatever circumstances he found himself in, and once he had hit upon the 'pamphleteering in the field' phrase, he made a meal of it. Thus in the year of which I write he had been stumbling aimlessly from one place to another, pencilling pamphlets as he went. Now, in Mexico, he slumped against an adobe horse-related street appurtenance, lit one of his crumpled cigarettes, and wrote in his notebook:

\emph{Pamphlet In The Field, Number Ten. I appear to be in a Mexican village. There will be ramifications, but as yet I do not know what they will be.}

It was at this point that Ramon Miguel 'Mike' Vargas came upon the scene. He was off duty from his top job in the Mexican narcotics bureau, but his presence in the small dusty village has never been satisfactorily explained. Perhaps, like Dobson, he was just there, for no real purpose. History is full of such apparently meaningless conjunctions. Consider that Stalin and Trotsky first met each other in what is now a McDonald's restaurant on Whitechapel Road in east London, or that Richard Milhous Nixon left Dallas from Love Field mere hours before John F Kennedy flew in on that fateful November morning in 1963. Can the encounter of Dobson and Vargas be said to have the same resonance? Certainly, what passed between them seemed unimportant at the time. Remembering that he had to buy some fruit pastilles for his wife Susie, and wishing to jot down a note, Vargas asked to borrow Dobson's pencil. The pamphleteer obliged, mindful of the quiet authority of the Mexican lawman, but as he handed over the pencil he managed, in that Dobsonian way of his, to frighten some hens who were coming to eat some grain that had been scattered near the adobe horse-related street appurtenance. If you have ever seen a gaggle of panicked hens fleeing from a pencil-brandishing pamphleteer, you will know quite well what chaos can be wrought in a dusty village. There was uproar, and shouting, and the clattering of many cooking pots, and semi-automatic gunfire. By the time things settled down a few minutes later, after the village hen person wove his henly spell over the hens to placate them, Vargas had forgotten all about Susie's fruit pastilles and Dobson had quite lost his train of thought. Both men might have forgotten the entire incident, but their lives were changed forever. It is not clear precisely what happened when Vargas returned to his motel room fruit pastilleless, and it would be foolish to speculate. We know, however, that Dobson underwent a neurasthenic miasma when he found he was incapable of completing \emph{Pamphlet In The Field, Number Ten}. By nightfall, he had left the Mexican village as suddenly as Ambrose Bierce had vanished. Indeed, he had left Mexico altogether, and was aboard a packet steamer, bound, eventually, for home. He spent the entire voyage, and the connecting voyages on any number of other seagoing vessels, huddled in his cabins, sucking on vitamin tablets and mopping his brow with wrung-out dishcloths. His notebook remained unopened, unwritten in, partly due to the neurasthenic miasma and partly because, in all the mayhem of the panicking hens incident, Vargas had popped Dobson's pencil into his pocket, and he had neglected to return it. The pamphleteer fetched up at home months later, still wearing his fur muffler and reindeer-hide kagoul and padded Alpine boots. The rain was thrumming on the roof and Marigold Chew was fixing a tarpaulin over the guttering. She greeted Dobson brightly.

'Hello Dobson! How was the field?'

'I am done with the field,' he muttered, 'It has broken me. From now on, I shall write all my pamphlets sitting at my escritoire, a pot of pencils and a pencil sharpener in easy reach.'

And without another word, he went and sat at the famed escritoire, and began to write the pamphlet we know today as \emph{The Unutterable Chaos Caused By Panicking Hens} (out of print). As you probably recall, he dedicated it to Ramon Miguel 'Mike' Vargas.