\chapter{Other Glubbs}

MAUD GLUBB, THE aviatrix and author of \emph{The Book Of Gnats}, is just one among a number of Glubbs of note. This ought not surprise us. After all, as H P Lovecraft wrote in \emph{The Thing On The Doorstep, It began with a telephone call just before midnight. I was the only one up, and sleepily took down the receiver in the library. No one seemed to be on the wire, and I was about to hang up and go to bed when my ear caught a very faint suspicion of sound at the other end. Was someone trying under great difficulties to talk? As I listened I thought I heard a sort of half-liquid bubbling noise 'glub\ldots glub\ldots glub\ldots' which had an odd suggestion of inarticulate, unintelligible word and syllable divisions. I called 'Who is it?' But the only answer was 'glub\ldots glub\ldots glub-glub.'} (The one-B Glub is a common American variant of the more standard two-B Glubb.) It can be argued that Lovecraft's 'glub' is a repetition of a single Glubb, much as we accept that Edgar Allan Poe was referring to only one person when he shouted the name 'Reynolds!' repeatedly as he lay dying in the Washington College Hospital in Baltimore. Yet Lovecraft clearly indicates 'unintelligible word and syllable divisions', which sensible people who have their wits about them will take as hard evidence of multiple Glubbs.

One such other Glubb, if we use that term to distinguish our subjects from the aviatrix, was Old Mother Glubb. This fine upstanding dowager was not the mother of Maud Glubb, by the way. In fact as far as we can ascertain, Old Mother Glubb had no children. She was dubbed 'Old Mother' because she was very aged, at the time we learn of her existence, and because she bred moths. It is easy to see how people assumed that 'Mother', to rhyme with 'Hiawatha', should be pronounced to rhyme instead with 'brother'. From such tiny presumptions can titanic historical errors occur. Several bright and promising genealogists saw their careers ruined, their health destroyed, and their lives wasted as they tried and failed to track down Old Mother Glubb's non-existent progeny. Had they known about her revolutionary moth-breeding programme, and the attention it gained from moth experts on two or three continents, things may have been very different indeed. Or perhaps not. Perhaps each of these genealogists had a fatal flaw which sent them chasing phantoms, and had they not driven themselves mad with Old Mother Glubb, they would have alighted upon some other hopeless pursuit. There are many, in the groves of academe.

There is, for example, the case of another Glubb, Binnie Glubb, the man who became Professor of Futile Studies at a large important university. Sometimes called the senile grandparent of postmodernism, Binnie Glubb spent years and years writing incomprehensible twaddle, in unreadable prose, about - well, about god knows what. If we knew what he was writing about it would suggest that occasionally he was both comprehensible and readable, and he was neither, ever. And yet his screeds were typeset and bound and published and sold and stuck upon shelves in libraries across the land, and he had his photograph taken, smoking a pipe, shoulder to shoulder with a French intellectual or a Maoist psychopath, and airheads wrote fawning profiles of him for the Sunday supplements.

No such plaudits for the next Glubb in our set, the one-legged bobsleigh competitor Digby Glubb. He was a sports-mad youth who was nonetheless completely useless at everything he tried. Failing at p\'{e}tanque, he took up pingpong. Failing at pingpong, he tried vinkensport. Savaged by finches, he turned to curling, and failed again, failed better, as Beckett might say. All this time he had two legs. On his thirtieth birthday, still utterly useless in all sporting events, he toppled into a ditch full of fierce biting ants, which ate most of one of his legs before he was rescued by an ant-killing patrol. Recovering in a superbly sterile clinic, Digby Glubb researched sports he could take part in while sitting down, and decided to devote his life to bobsleigh, in spite of the fact that he did not quite understand the point of it. For two decades he continually crashed any bobsleigh he sat in, whether solo or as part of a team. What drove him on, from one calamity to another? Was it perhaps revenge against the fierce biting ants which had hobbled him? There are those who can be spurred on by often harmful derangements, be they vengeance or jealousy or preening vanity.

Consider the \emph{cravattiste} Shelvington Glubb. Convinced that, when he wore one of his cravats, he might be mistaken for a young Apollo, this Glubb pranced about the boulevards of Pointy Town watching Pointy Towners gasp and swoon at his beauty. Some of them were blinded, he was so like the sun.

There, you have some Glubbs to be going on with.