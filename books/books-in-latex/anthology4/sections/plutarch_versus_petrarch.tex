\chapter{Plutarch Versus Petrarch}

THE CHIEF REASON Plutarch and Petrarch never met in a he-man wrestling bout is a matter of simple chronology. Consider their dates of birth and death, Plutarch (46-120) and Petrarch (1304-1374). More than a thousand years separates their days on our little planet, and none of the fantastic time-travel contraptions dreamed up by sci-fi writers and visionaries has ever been built, at least not in any working form. Had one been made, then Plutarch could have been whizzed into the future, or Petrarch into the past, and, suitably attired, or possibly naked and greased like the wrestlers of certain ancient civilisations, the pair could have entertained the crowds, displaying all sorts of he-man wrestling holds, and grunting, and throwing each other around the ring. If their bout was fought according to a brutal set of rules, someone may have needed to stand by with a pail and a mop, to clean up any shed blood, and someone else, preferably a chirurgeon, would be needed to place splints on any broken bones. It is unlikely that either Plutarch or Petrarch would agree to fight to more genteel wrestling rules, for they would not wish to appear namby-pamby to their thousands upon thousands of supporters.

Whose side you come down on depends to a large extent upon your own cranial blips. If you have spent much of your adult life poring over the \emph{Parallel Lives} and the \emph{Moralia}, scribbling a lot of notes in the margins, or in a pad, then you will probably cheer on Plutarch and hope that sickening crunching noise you heard is not one of his bones being shattered. On the other hand, if you like nothing better than to curl up in a hammock with a copy of \emph{De Remediis Utriusque Fortunae} or the \emph{Secretum}, then you will be backing Petrarch, and wanting to see that pail filled with the blood of Plutarch. Or, if you have wasted your life and never read a word by either of these titans, you may be swayed by, say, Plutarch's beard or by Petrarch's hat. The position you will not want to be in is one of neutrality, for you will see how the adherents of both the 'Big P-Archs' are violently partisan, with a lust for gore, kept apart by a fence of iron stakes. Not to join one mob or the other is to miss out on the frenzy of the day, and in any case, you will have to choose to sit in one section of the ringside. And in spite of their screaming and gesticulating and spitting, the mobs do sit, quite neatly even, on their benches or their bucket seats. No one wants to spoil the view of the ring, wherein Plutarch and Petrarch land forearm smashes and trip each other up and stamp about in a great show of he-man grappling.

Once you have plumped for your champion, you will want to take out a bet with one of the ringside bookies. Gambling at time-travel wrestling bouts is big, and sometimes ugly, business. Punters' scuffles tend to break out, and rampant bookie-hounds are unleashed. These are fearsome dogs, each one individually cloned from the DNA of Cerberus, or whatever Cerberus' equivalent of DNA is. That would be a matter for the mythologists, and it is not a good idea to get embroiled in their arguments, for you would soon go loopy. The bookies set their bookie-hounds on any punter scuffling, or any punter they just don't like the look of. But carry yourself with grace and good humour and you ought to have no problem placing your bet, whether it is on Plutarch or Petrarch.

This is not, of course, a he-man wrestling match to the death, for both writers need to return at some point to their own times, albeit bruised and bloodied and broken. If for any reason one or both of them were not plopped back into their own small world, whether it be in Boeotia or Arezzo, or thereabouts, there would be a hideous juddering panic-inducing crumplement of the space-time continuum, with unforeseen consequences. One such consequence, weirdly, might be that you wake up in the morning to discover that the world is run by giant hamsters, and all because, after their close-fought he-man wrestling bout, Plutarch and Petrarch wandered off together out of the ring and forgot to return to the wholly fictitious time-machine at the appointed hour. Giant hamsters in charge is one of multifold possibilities, but one I suspect we would all wish to avoid. However attractive the idea of the two literary giants hobnobbing as they vanish over the horizon, arm I arm, it really is important to bundle both of them back, or forward, to their own times, for the good of all, except perhaps the hamsters.