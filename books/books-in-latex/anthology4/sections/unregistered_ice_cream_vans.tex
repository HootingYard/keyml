\chapter{Unregistered Ice Cream Vans}

IN THE MIDDLE of a tirade the other day, I mentioned an unregistered ice cream van, and I have been asked to explain in what ways such an ice cream van differs from a registered one. It is all to do with robins.

Some years ago, an inpector appointed by the regime started to notice the presence of threatening robins whenever an ice cream van went barrelling along the lanes around Pang Hill and its satellite dire and dirty villages. The more the inspector looked, the more robins he saw, in the sky, upon branches, in shrubbery and hedges, atop buildings, and even perched on the power lines that are strung from paling to paling with no apparent purpose in this gas-fuelled faubourg. When the wheezy tinkling of an ice cream van was heard, the number of robins seemed to double, then triple, and a sense of menace was made manifest.

The inspector passed this information on to an officer back at headquarters. The officer, resplendent in furry pelts and Hohenzolleren cavalry marshal's boots, immediately hauled in as many ice cream van drivers as his agents could pull from their beds at midnight, and had them questioned.

Have you, or has your van, ever been attacked by robins?

This was the gist of all the interrogations, however they were phrased. The replies contained much dissembling, many evasions, and outright lies, but that was only to be expected, and the graph on which the results were plotted was adjusted accordingly. The officer had half a dozen copies of the graph printed, in four colours, green and pink and dun and scarlet, and sent them by pneumatic tube to his superiors. He was commended for his rigour, and awarded yet another medal, this one of tin, circular, and expensively beribboned.

The committee of six, recognising their own lack of ornithological expertise, empanelled a birdy nutter to add weight to their deliberations. He was no robin specialist by any means, being more of a corncrake man. He spent most of his time out in the wilds, living in tree hollows, surviving on berries and water from rills, and the first meeting of the committee to decide what to do about the menacing robins had to be delayed while he was tracked down. The task of doing so was assigned to a cadet who happened to be a cousin of the original ice cream van investigator. Though cousins, they had been reared as brothers, as close as twins. The cadet's name was Bim and the investigator was called Bam, and they were both fanatically loyal to the regime, in spite of the fact that their other cousin, Shevelham, was a treasonous cur languishing in one of the prison forts on the windswept plains out west.

Bim found the corncrake expert hiding in a forest, eating a choc ice, peering at birds through a pair of binoculars. They were not robins.

There was a deal of difficulty in getting a seventh copy of the graph printed for the birdman. There was no shortage of coloured inks, nor of paper, but the regime was faced with terrific transport problems. So many vans had been converted to the vending of ice cream, despite the robins, that few vehicles remained abroad for other purposes. Ink and paper could not go by rail, nor by coastal paddle-steamer, due to ancient regulations none dared overturn. It was, after all, that sort of regime.

More and more robins were gathering in the sky and the trees and between the palings.

It was on the third or fourth day after the empanelment of the committee that the idea of registering the ice cream vans was proposed. The bird person, hefty, and somewhat disorientated out of his woods, remained unconvinced that robins were intelligent enough to tell a registered ice cream van from an unregistered one. The person sat to his left explained that unregistered ice cream vans would, if found upon the lanes and boulevards, be blown up with bazookas.

Fewer vans, fewer robins.

This was the committee's slogan. They had it stamped on items of stationery and teacups and beakers as part of a campaign. A protocol was devised whereby ice cream van drivers could apply for registration, at post offices and bureaux and at vanishing points down grim horrifying alleyways. This last was picked as a wheeze to do away with the more stupid ice cream van drivers, those whom it was thought may be attracting more than a fair share of the maleficent robins. The birdy panellist was asked to draw up a special report on the meaningful brain activity of both ice cream van drivers and a sample of captured robins, to find out if there was any correlation. He decamped back to the forest before completing his work, and no search party ever found him, however frantically they crashed through the undergrowth beating the foliage with sticks.

Much was left undone that ought to have been done. Yet the registration of ice cream vans remained as an emblematic law. As the robins gradually dispersed, due to the depredations of owls and weasels and monkeys and a change in the shapings of the sky, so the enforcers with their scanners and firearms were removed from the kerbside kiosks, and unregistered ice cream vans again braved the roads.

That is how it was.