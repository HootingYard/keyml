\chapter{Inky Puck Stampings}

IN HIS LATER years, Blodgett amassed a collection of inky puck stampings, kept in an album bound in the starch-stiffened fleece of a lamb. The fleece was spotted with unexplained bloodstains which Blodgett made no attempt to remove. He could have used a patent bloodstain eradication spray goo as manufactured by Don Federico's Royal And Ancient Portugese Spray And Paste Company, but he chose not to. Boffins in a lab were recently given the opportunity to scrape minuscule quantities of the blood off the binding. When they subjected it to tests, they were able positively to identify it as the blood of a fruitbat. Curious indeed, but no more curious than much else about Blodgett's later years.

In his new television series \emph{The Piteous Whimpering Of A Soul In Torment}, celebrity historian Simon Sebag Stimmungbag examines in detail the final decade of Blodgett's life, and unearths some starling facts. I'm sorry, that should read startling facts, although among them are a number of Blodgett-starling collisions. If it seems unlikely that a man could collide with a starling on repeated occasions, as per being struck by lightning, Stimmungbag has at his fingertips a mass of convincing evidence, including ornithological records, accident reports, and ticket stubs from showbiz bird displays.

He also gives us a remarkable account of the time Blodgett decamped to a loggia, neglected to keep a log of his stay there, and upon returning home spent some six weeks dementedly chopping logs with a very sharp axe, despite being over eighty years old. He then carted the entire supply of chopped-up logs back to the loggia, dumped them outside the door, and kept a log in his journal of their gradual depradation through theft and rot.

There are other distinctively Blodgettesque glimpses: hen harrying, bricks on the brain, tormented scribblings on parchment regarding soup, starling collisions, misted glass obscuring a decisively important bus timetable, things chewed and spat out, intimations of mortality, imitations of Christ, intimacy with a mute milkmaid, delusional vampires, card games, ditch digging, reading aloud \emph{A Fiery Flying Roll} by Abiezer Coppe to an audience of stunned potters, other potters encountered in hospital corridors, smashed-up lobster pots, a zest for crumpled things... the historian takes us through it all, at a pace sometimes gentle and at other times hectic, and occasionally incomprehensible unless one is already familiar with the material. That is Stimmungbag's way, as viewers have come to expect from his previous documentaries on topics such as collisions in the sky and on starlings.

For most of us, though, whether or not we are students of Blodgett, it is the attention paid to the collection of inky puck stampings that is truly revelatory. Indeed, I had no idea that Blodgett maintained such a collection, nor that he kept it with such uncharacteristic care in a starch-stiffened lamb's-fleece-bound album stained with the blood of a fruitbat. Again, one has to admire the way Stimmungbag marshals the evidence, a particularly difficult task when one considers how many similar collections were destroyed after the coup which brought the new regime into power. There will be younger viewers who have never known about inky puck stampings, let alone that people used to collect them. Of course, few were kept in albums as magnificent as Blodgett's, it being far more common in those days to shove them haphazardly into cardboard pouches or discarded agricultural sacks. What shines most brightly in this excellent television series is the almost inhuman concentration with which Blodgett attended to his collection, peering at the stampings for at least three hours every day no matter what else was going on in his life or in the world at large. It is remarkable that on the day 'Lovin' You' by Minnie Riperton hit number two in the British singles chart, Blodgett spent at least nine hours not only peering at his inky puck stampings but rearranging them within his album, getting through an entire packet of stampings hinges, each one torn in half as was his usual habit. I think it says something about the man that he did not even collide with a starling that day. And it says something about Simon Sebag Stimmungbag that he has crafted such a long, blurry, black-and-white television documentary series with a deafeningly loud yet simultaneously muffled soundtrack to which one must listen with one's ears pricked up and one's mouth hanging open, drooling into a pewter pot held by one's unpaid companion on the balcony of a sanatorium upon which the snow falls, and does not melt.