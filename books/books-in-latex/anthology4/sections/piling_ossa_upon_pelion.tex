\chapter{Piling Ossa Upon Pelion}

MY APPOINTMENT WITH destiny, or dentistry, I forget which, was cancelled, and I had an afternoon to play with, so I thought I would try my hand at piling Ossa upon Pelion, as the Aloadae did in the old story. In some versions, they piled Pelion upon Ossa, so to be on the safe side it seemed best to attempt both. Now obviously my withered limbs and general puniness prevented me from literally piling one mountain on top of another. I had in mind to construct miniatures, to scale, out of cardboard and rags and cotton wool and glue.

Before turning my hand to this exciting if pointless project, it occurred to me that it was just the kind of thing Tiny Enid might have done when she found herself at a loose end. The plucky infant fascist could not bear to be idle, and it was quite possible that, between adventures, she might have piled Ossa upon Pelion, or vice versa, although in her case I am sure she had the resourcefulness to tackle the real mountains instead of small lightweight copies. Had she ever passed the time in this fashion, I was keen to pick up any tips, so I consulted the literature. Ever since the publication of Mavis Gasball's majestic \emph{Complete Reference Guide To All The Doings Attributed To Tiny Enid, In Twenty Volumes, With Twelve Rotogravures By Noted Rotogravurist Rex Rotograv}, it takes even the dull-witted a matter of minutes to track down the most obscure episodes in the life of the heroic tot. The afternoon was still young when I slammed the books shut, satisfied that there was nothing Tiny Enid could teach me about the task ahead. There was mention of neither Ossa nor Pelion in the index, nor of the Aloadae, nor of Otus nor Ephialtes, and the sole reference to Mount Olympus led to a thrilling, yet unrelated, account of Tiny Enid setting fire to a paper aeroplane upon its pinnacle at the culmination of the \emph{affaire d\'{e}sagr\'{e}able} in 1955. I was too familiar with this to reread it, so I replaced the books on the shelf, buckled up my boots, and pranced off across the greensward to the hut wherein I kept my cardboard and rags and cotton wool and glue.

Was ever a hut so cherished as mine? It is filthy and in a state of collapse, but to me it is a kind of paradise.

I switched on my radio to listen to Cardboard Mountain Modeller's Playtime as I worked. They were playing Scriabin. How curious, I thought, that so accomplished a pianist had such tiny little hands! My own hands are leaden and fat and clumsy, more's the pity. I am afraid that after an hour or two of inexpert fumbling and mashing and prodding I had created a quartet of shapeless compacted clumps. A quartet, because I strived to make two model Ossas and two model Pelions, that I might pile Ossa upon Pelion, and pile Pelion upon Ossa, simultaneously rather than consecutively. Perhaps, in so doing, I was overambitious, and would have obtained better results had I been satisfied with a single pair, the positions of which, Ossa atop Pelion, or Pelion atop Ossa, I could have switched as often as the fancy took me, or, indeed, never, were one tableau more pleasing to the eye than the other. As it was, all I had to show for an afternoon of strenuous cackhandedness were four almost identical messes of cardboard and rags and cotton wool and glue, a fuming temper, an overheated radio set, and a sense of defeat I would struggle to shake off for years to come.

I bundled my Ossas and Pelions into a burlap sack and, on my way home, chucked the sack into a pond, where it floated for a while, until it was eventually destroyed by the ferocious pecking of swans.