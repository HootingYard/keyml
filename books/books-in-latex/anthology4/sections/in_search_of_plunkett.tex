\chapter{In Search Of Plunkett}

\emph{Herewith Chapter 49 of} Pallid Ada, The Crippled Heiress:
\\
\\
COUNTLESS FATHOMS DEEP, far far below the roiling ocean waves upon which our battered leaking ship is pitched and tossed, down, down in the black depths at the very sea bed, in a crevice of a rock encrusted with a billion years of salty encrustment, what wonders would we see, lit by the glow of fat eerie beings that pulsate and drift in the undersea world! Directly below our ship, far, far below, nestled in the crevice, there is a man! Dressed in garb half tattered black cloth half trailing weed, he is breathing freely, from his perfume bottle atomizer air bulb invention. Around him are gathered ranks of aquatic beings, the finned and the tendrilled, the suckered and the eyeless, the translucent and those that are mere blobs. They are rapt. For, with glubs and gurgles, the man is preaching to them the Word of the Lord. The man is none other than Alonzo Plunkett, the kindly Christian gentleman we last saw being dragged into the sea from a deserted stretch of British beach in the lobster-like claws of a sea monster!

It is with the aid of compass and sextant and stout-hearted prayer that we have dropped anchor at the very spot, in the vastness of the oceans, below which the semi-amphibious benefactor of Pallid Ada, the Crippled Heiress, is converting the denizens of the deep. Following his briny abduction, a committee of seaside worthies raised funds for a voyage to go in search of Plunkett. No better captain could we have than the terrifying and God-fearing retired Admiral Pipstrew. Poring over his charts, sucking on his pipe, beseeching the Lord, our captain has steered us hither and thither for six long years, following the most fleeting of clues, until now the prize is in our grasp!

Auks and guillemots wheel madly in the sky above our ship as, out on deck, crewmen Totteridge and Whetstone laboriously engarb themselves in the very best shiny brass Victorian diving gear. The captain stands to one side, his highly-polished peg leg glistening in the sunlight. Having at the last affixed to their helmets unfeasibly extensive lengths of pneumatic rubber tubing, the crewmen topple over the side, but not before Admiral Pipstrew booms out a hymn, and we all join in. And then, first Totteridge, then Whetstone, are gone, vanished below the waves. We can but wait for their return, Alonzo Plunkett safe and sound in the net they carry between them.

Sound, did I say? However kindly, however Christian, however expert in the geological treasures of the British coastline, how sound can a man be after spending six years in the blackest deep, his only company the weird aquatic life-forms that God, having made, at once condemned to remain submerged in the depths of the blue tumultuous waters? From poop to orlop, as we wait, there are mutterings that the Plunkett restored to the world of men will be a twitching and shattered glub-glubbing wreck, drooling brine, his once fine bouffant shockingly tangled with weed and sea-scum. The tension mounts, and we busy ourselves with abstruse nautical activities, as our captain broods on the bridge, absent-mindedly prising barnacles off a stray timber plank.

Night falls. Night at sea is very different to night on land. Better writers than I, immeasurably better, have evoked the arresting aura of the sea-night. So much greater are they, the very air around them is rarefied and pure and I am not fit to breathe an atom of it. If ever I were to meet such a writer, which I would not, but if, say, I was mistakenly invited to a swish cocktail party at which one was present, I would expect to be squashed beneath their boot like a mite. I am but a humble jack tar, eking my sorry living in the rigging, scribbling prose in my too few idle moments, smudging the brine-soaked pages as I sweep from them the crumbs of the hard tack biscuits that fall from my caried gob.

Dawn breaks, and still there is no sign of Totteridge and Whetstone. Dawn at sea offers a very different prospect to dawn on land. There are great painters whose canvases will show you why. My own daubs, done with the dregs from tins of ship emulsion, will not.

Our only means of communication with the two crewmen gone below to rescue Alonzo Plunkett is a coded system of tugs to the ropes to which they are attached. It had been thought prudent not to extend the complexity of the system by adding a series of secondary tugs to their pneumatic rubber tubing, as a single misjudged tug might disrupt their air supply and consign them to a watery death. Admiral Pipstrew, who has not slept, is ready to command one tug on the ropes, to signal, as previously arranged, 'Is Almighty God keeping you safe from harm in the immensity of His fathomless oceans?' But he stays his hand when an eagle-eyed cabin boy pipes up to announce that both Totteridge's and Whetstone's ropes are being tugged from below. And as we look, indeed they are, not merely tugged but yanked, pulled with main force, so threatening to topple the mast to which they are lashed. That mast is the mainmast, so as it topples, the ship itself will keel over. We are in mortal danger!

Whatever is tugging those ropes has inhuman strength. Our captain has to decide whether to sacrifice two of his crew for the good of all. But a night without sleep has scattered his wits, and he delays too long. With a last mighty heave, the ship is overturned, and every man jack of us is tossed into the sea.

Our descent to the depths is calm and dreamlike. There is no panic, no thrashing about, no bobbing to the surface for a last hopeless gulp of air before plunging again. We sink, every one of us, slowly and peacefully, until we land, with scarce a bump, on the sea bed. And there before us, in translucent shimmers, we see Alonzo Plunkett, and Totteridge, and Whetstone, and any number of fantastic marine life-forms, greeting us with glub-glub hosannahs, in the undersea paradise built with his bare hands by the kindly Christian gentleman during six years of hard subaquatic toil, to the glory of God.