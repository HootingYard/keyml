\chapter{The Lost Lozenge}

I FELT PANGS when I lost my lozenge. I was in a bricked-up brutalist bricky building when I noticed it was missing. I'd mislaid it before, once on a Thursday and once outside a tent on a campsite of many gusts. It's a yellow lozenge, a small cake or tablet of medicine and sugar meant to be held in the mouth and dissolved, but I have never put it in my mouth. I carry it in the pocket of my trousers, whichever pair of trousers I am wearing, and it is from my pocket it must have fallen, earlier, without my noticing. On the Thursday I found my lozenge within minutes, it had dropped on to the floor, and the floor was covered in bright red linoleum, so the yellow of my lozenge was easily visible. Outside the tent I was perplexed, but a passing widow woman approached me holding my lozenge in her black-gloved hand and said she had seen it tumble on to the grass when I was doing calisthenics a few minutes before. Usually when I do my jumping about and somersaulting and so on I wedge my lozenge deep into my pocket and push a scrunched-up rag or dry dishcloth in on top of it, but at the campsite I neglected to do so, for I was distracted by the millions of starlings swooping in the sky. In the bricky building, however, there was no red linoleum and nor was there a helpful widow woman. Also, it was bricked-up, so there was little light for me to see by, and soon it would be dusk and the bricky building would be darker still. I did not know if I had lost my lozenge here or elsewhere. No wonder I felt pangs.

The pangs began in the pit of my stomach, as pangs often do, and slowly moved upwards until I felt a constriction in my throat. Pangs like those, that interfere with one's breathing, can be lethal. To think that I might perish through pangs for something as tiny as a lost lozenge! And it was a very tiny lozenge. When manufactured, in, I supposed, a lozengery, it had been somewhat bigger, but before it came into my possession it had been partially sucked upon and some of the outer coating had thus dissolved. The semi-sucked state of the lozenge was the reason it was so precious to me, and why I kept it in the pocket of my trousers, and felt pangs when it was lost. To be more precise, it was the identity of the sucker that was important, for this lozenge had been sucked by my all-time hero, the wrestling champion Bruno La Poubelle. He choked on it, and spat it out, and it landed at my feet, for I happened to be standing next to my hero on the balcony of a plush hotel. He was a guest, of course, whereas I was a mere employee of the hotel, a mopper of balcony tiles, with my mop and bucket. Bruno La Poubelle stopped choking as soon as he expelled the lozenge, turned around and swept back into his suite, and I picked up the lozenge and popped it into my pocket. That was years ago, and I have treasured the lozenge ever since, and kept it safe, except for the Thursday and the gusty campsite and, now, today, when for the third time I have mislaid it.

I had other La Poubelle memorabilia: a milk tooth, a discarded cochlear implant, a battery from his wrestling training machine. These were displayed in a small votive shrine I had made in my kitchenette, but the lozenge I preferred to keep on my person. Now, leaning against a wall in the bricked-up building, I mentally retraced the steps on my journey here, trying to recall where I might have been when last I was sure that the lozenge was in my pocket. There were many, many steps to remember, for I had come from afar, on foot. I knew that I had the lozenge at Sawdust Bridge, because I had to turn out my pockets at the customs post. I distinctly recalled putting it back in my pocket after the frantically-eyebrowed customs man gave me the tap with his tappy stick, clearing me to carry on across the bridge and down through the subterranean car park, past the gated Hazchem compound and up again on to the path alongside the allotments. From there I hurried past the haunted zoo and slowed my pace as I forded the river at Shallow Sludge, crossed the football pitch and the park and the airfield, then rolled down the slope towards the puddles and ponds and the hermitage. There was another customs post hereabouts, but it was deserted, watched over by a solitary mordant heron. I struggled through bindweed and brambles and then followed for a few miles the line of the Great Celestial Pneumatic Railway until I reached the engine house, where I turned off past the crushers and hooters and thumpers and clunkers and carried on past the swan hospice and the post office and the ear clinic, where Bruno La Poubelle had discarded that cochlear implant and had a second one fitted, and the glue factory and the paper mill and the lunatic asylum and the terracotta army parade ground and the canoe maker's and the clown shop and the windmill and the newsagent and the Aztec fundamentalist temple and the trendy so-called 'suet pudding skyscraper' and the bordello and the ironworks and the futuristic plasma ray gun repair shop and the guide dog enclosure and the bell foundry and the cave full of vampire bats and the other cave and the country club and the patch of muck and the bowling alley and Rolf Harris's house and the pit of fire and the pit of doom and the buttercup fields and the sinister laboratory and I ended up here at the bricked-up brutalist bricky building and discovered that I had lost my lozenge.

When I arrived, of course, the bricky building was not fully bricked-up. I was able to make my entrance through an unlocked side door which led into the gloomy vestibule, on the walls of which hung the stuffed heads of otters and badgers and giraffes. I lay down on the floor to take a nap, and when I awoke I found that, while I slept, the remaining unbricked-up bits of the building had been bricked-up. I was so beset with pangs about my lost lozenge that it did not immediately occur to me that I was trapped. Now I have realised that is the case, I am going to have to do something about it. I am a shape-shifter, so I will shimmer ethereally for a few seconds and become like a beetle, and scuttle out of the bricky building through pipes and ducts. It will take much longer to retrace my steps in beetle-form, but it will be another fortnight before I can shape-shift again, and I will search diligently for my lozenge, and hope I am not trampled underfoot by the crowds making their way to the wrestling stadium for Bruno La Poubelle's final bout.