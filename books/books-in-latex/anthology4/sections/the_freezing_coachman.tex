\chapter{The Freezing Coachman}

\begin{quotation}
'Tolstoy tells the story of an aristocratic woman at the theatre weeping at the imaginary tragedy enacted on the stage. At the same time, outside in the cold, a real tragedy is taking place: her old and faithful coachman, awaiting her in the bitter winter night, is freezing to death.'

-- Raymond Tallis, 'The Freezing Coachman : Some Reflections On Art \& Morality', in \emph{Newton's Sleep} (Macmillan, 1995), abridged version in \emph{Theorrhoea And After} (Macmillan, 1999)
\end{quotation}

COUNTLESS READERS, COMING upon the words 'the freezing coachman', will think of neither Tolstoy nor Tallis, but of the indefatigable paperbackist Pebblehead. It is well nigh impossible to keep track of the short stories featuring the eponymous frozen hero he taps out on that battered old typewriter of his, pipe packed with scraggy Montenegrin tobacco clamped Simenon-like in his jaws. Unlike Tolstoy's character, Pebblehead's freezing coachman remains alive, a ghoulish figure covered in ice, with a reproachful gaze and a booming monotone. In many, but not all stories, he has a Dutch accent.

Pebblehead has been criticised, by the snooty and the hare-brained, for the wild inconsistency of his coachman. In \emph{The Freezing Coachman And The Blunkett Cow Attack}, for example, he is a sort of mystic cow-whisperer, a gentle and benevolent soul with a heart of gold. Cold gold, but gold nonetheless. In \emph{The Freezing Coachman And The Carpets Of Madness}, by contrast, he is evil personified, so evil that Beelzebub himself is reduced to a quivering gibbering wreck in his presence. And then of course there is the famous story \emph{The Freezing Coachman Goes Rogue!}, and we all know what happens in that one!

But the variations in his character are as nothing when compared to the bewildering number of guises under which the 'coach' of which he has charge appears. It is described, in one Freezing Coachman story or another, as a coach or a carriage or a landau or a landaulette or a britzka or a gig or a trap or a charabanc or a float or a buggy or a hansom or a shandrydan or a post chaise or a brougham or a droshky or a berlin or a wagon or a calash or a jitney or a pony cart or a minibus or a caboose or a caravan or a sleigh or a fiacre or a dray or a jeep or a lorry or a sulky or a cab or a van or a brake or a crate or a taxi or a rattletrap or a sedan chair or a bus or a tin lizzie or a carriole or a curricle or a dustcart or a stanhope or a quadriga or a phaeton or a trolley or a tumbrel or a troika or a saloon or a hearse or a diligence or a bubblecar or a fourgon or a flivver or a clarence or a growler or a conveyance or a roadster or a tilbury or a runabout or a jalopy or an oxcart or a hackney cab or a tarantass or a black maria or a barouche or a tractor or a tonga or a tank. This may be a case of Pebblehead being slapdash, or playful, or simply not having a clue what he is talking about, given that some of these vehicles can hardly be described as a 'coach' by any sensible person.

The moral of Tolstoy's tale, that an appreciation of great art does not necessarily make one a good person, is obvious. Equally, nearly all of Pebblehead's stories have a clear moral point to make. We are told that one should never bury a dog while it is still alive, never accept toffee apples from spooky strangers, always rain curses upon a cow that attacks a blind Member of Parliament, or upon a blind Member of Parliament who attacks a cow, depending on which version of the story you believe, never walk widdershins three times around a kirk, never push a boy scout into a crevasse, don't count your chickens, eat five portions of fruit and vegetables a day according to government guidelines, never put all your eggs in one basket, always check the accuracy of George Orwell's daily egg count, always uphold the ineffable majesty of the tinpot king of the land of Gaar, never get too close to the edge of the bottomless viper pit of Shoeburyness, and don't ever, ever wave a towel in the face of an Ampleforth Jesuit. It is true that sometimes we put aside a Freezing Coachman story feeling that Pebblehead has lectured us rather than entertained us, and some of the lessons we are taught are fit only for five-year-olds, but at their best these tales can be both unforgettable and devastating.

I am thinking, for example, of \emph{The Freezing Coachman Sorts The Abstract Expressionist Wheat From The Chaff}, a thinly-disguised and blistering attack upon the adolescent cod-mystic witterings of Barnett Newman, who tried to imbue his big flat boring daubs with universal and eternal significance. There is an irresistible urge to clap with glee when, in the final paragraph, the Freezing Coachman steps out of his cabriolet and upturns a pot of emulsion over the head of the ludicrous painter 'Bennett Nerman', before beating him with a spade, poking him with a stick, and tying him fast to railway tracks upon which the 4.45 non-stop express to Uttoxeter is due to thunder within the next couple of minutes.

It may well be the finest of all the Freezing Coachman stories, but do not take my word for it. Read every single one of them, the brilliant and the witless, and make up your own mind.