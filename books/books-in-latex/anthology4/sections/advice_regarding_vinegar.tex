\chapter{Advice Regarding Vinegar}

THE BEST THING to do, in certain circumstances, is to lie on your side, upon the grass, in a meadow, and have an acolyte pour vinegar into your ear through a funnel. When you stand up, in the middle of the meadow, and tilt your head, shaking it a little, the vinegar will be expelled from your ear and you will feel the benefits. 

It is important that you have an acolyte who can properly adjudge the amount of vinegar to pour into your ear. Too little, and the whole exercise is pointless. Too much, and you will be tilting your head and shaking it until the cows come home, and you will find it very difficult to expel all the vinegar. 

When the cows come home they may be disconcerted to find you in their meadow, with your tilted head, and some of them may become fractious. Fractious cows can be dangerous, so it will help if you have your acolyte armed with some sort of cow-protection device. This might be made of corrugated cardboard, or alternatively of tin foil. Best to consult a catalogue of cow-protection devices beforehand, with your acolyte at your side. 

Choosing an acolyte to whom you are prepared to entrust the pouring, and the cow-protection, is a fraught business, believe you me. It is a process during which you can expect much heightened emotion, many tears, a certain amount of wailing, and, now and then, fencing contests, with flashing \'{e}p\'{e}es. It has even been known for rival acolytes to bash each other about with spades, so it is advisable not to give them access to the keys to the potting shed. 

You will probably have at least one set of duplicate potting shed keys, hanging from a hook in the pantry, so make sure you keep the pantry out of bounds to your acolytes, save for those who need to enter it to fetch tins of tinned plums and tinned radishes and other tinned goods. It is a simple matter to give but one acolyte the responsibility for the fetching of tins, and that acolyte can be disqualified from even the possibility of pouring vinegar into your ears in the middle of the cow meadow, while you lie on your side, by having him blinded or having his legs broken and confining him to the house. 

Another thing to bear in mind when choosing the appropriate acolyte is that they must be able to get you from the house to the middle of the meadow with the minimum of fuss. Fuss is corrosive of the soul and has been known to result in horrible bodily eruptions such as sores and boils and suppurating patches of pus in such tender places as the groin and the armpits. You will want a level-headed and charming acolyte, one who, confronted by menacing geese on the way from the house to the meadow, will soothe them by singing something by Kevin Coyne in a deeply lovely voice. Geese are usually placated in this way, even the most ferocious ones. I would recommend something from his album \emph{Marjory Razorblade}, perhaps "Pig Latin" or "Eastbourne Ladies", the latter a favourite of John Lydon as long ago as his punk days when he was known as Johnny Rotten.

You will probably want to be carried from the house to the meadow on a palanquin, given your preening self-regard. You will thus require additional acolytes to do the carrying, one of whom can also be the vinegar-pouring acolyte, if you so wish. It is therefore a very good idea to have some distractions at hand to entertain those who, once their carrying is done, have nought to do until you command them to carry you back to the house when you have flushed the vinegar out of your ear. 

Most acolytes can be happied by board games. Poopy, The Kronstadt Rebellion, and Waiting Around In A Meadow While Vinegar Is Poured Into The Ear Of Your Hero are splendid and complicated games using dice, counters, and the feathers of placated geese, but of course there are many other games, board-based and otherwise, which you may consider packing in your pippy bag when making preparations to be carried from the house to the meadow. If you have acolytes who are resistant to the allure of exciting board games, it is probably best to dismiss them, with their tails between their legs, as the saying goes. 

The dismissal of acolytes can be problematic, particularly if they are clingy. Clinging acolytes are known to use gum to affix themselves to fixtures and fittings, such as the doors of pantries and the railings bordering manses. Be on your guard against them. Rifle fire tends to deter all but the clingiest, who may have to be detached from their gummy emplacements with gum-dissolving fluids in spray canisters. When making a purchase of these canisters, you may be asked to fill out a form declaring that they will not be used to dissolve the gum affixing an acolyte to the railings of your manse. In these circumstances, just lie. You will go to hell, but would you not rather be in hell than be subject to the fawning of a gummed acolyte?