\chapter{How To De-Fang Your Venomous Serpent}

SOONER OR LATER, most owners of venomous serpents will wish to de-fang their cold-blooded pets. Neighbourhood Watch gauleiters and local busybodies often make life difficult for the venomous serpent owner, particularly when the paths and lanes in the vicinity are littered with the bodies of poisoned innocents with tell-tale puncture marks and faces frozen in a rictus of twisted horror. You have to weigh up the pleasure of having a happy serpent giving full vent to its instinctual drive to sink its fangs into the flesh of a passing greengrocer, and the opprobrium which is an almost inevitable result. Social death, and a want of invitations to elegant drawing-room soir\'{e}es, are regrettably the lot of the venomous serpent owner, as if somehow it is the keeper rather than the pet who has been slithering about, dropping unexpectedly from the branches of trees, and injecting lethal toxins into everybody from the postmistress to the community hub outreach worker.

It should be noted that I am referring to singularly aggressive venomous serpents, those which attack without provocation, due to their being agents of Beelzebub.

Comes the time, eventually, when one tires of black looks from one's fellows in the bus queue and of always being served last in the butcher's shop. It is at this point that the venomous serpent owner concedes that the only solution is to de-fang their pet. Doing so is not without its risks, especially if the venomous serpent gets an inkling of what is afoot and decides to strike first. The obituary columns of the village newspaper are chocker with the names of rash wannabe de-fangers whose venomous serpents turned on them. Particularly quick-thinking venomous serpents have been known to plunge their fangs into the neck of their owner as a pre-emptive measure, before the owner has even resolved to go down the de-fanging route.

The only guaranteed method of de-fanging your singularly aggressive agent of Beelzebub is to mesmerise it. Once it has been placed in a trance, it is a simple matter to extract its fangs with a pair of pliers, and then to dab on to its gums some sort of dual-action antiseptic anaesthetic jelly. There are plenty of proprietary brands to choose from at your local chemist, if of course you have not been barred from there following the agonising death of the pharmacist, struck down by your venomous serpent on an otherwise unremarkable village afternoon. If that is the case, which it probably is, you will have to go further afield, to a different village, and in such circumstances it is best to place your venomous serpent in a creche facility while you are away. Taking the venomous serpent along for the ride has its pitfalls, such as the novelty of a fresh set of victims unlikely to be on their guard against its sudden, lethal attacks. You will not want to be a social pariah in two separate villages, as this will only compound your problems.

When you snap your de-fanged venomous serpent out of its trance, it will become fractious. Deprived of the ability to cause almost instant death by biting, it will seek new ways to express its inherent malevolence. And remember that Beelzebub will be taking an interest in its welfare. Unjust as it may be, you will quite possibly find yourself held responsible for a plague of stranglings and crushings in the village, depending upon the size of your ex-venomous serpent. Just as you were looking forward to a mantelpiece crammed with invitations to sophisticated dinner parties and potato show prize-givings, your hopes may be dashed, and you may have to mesmerise your serpent again. It is never easy the second time.