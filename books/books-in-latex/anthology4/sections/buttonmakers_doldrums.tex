\chapter{Buttonmaker's Doldrums}

HEPCAT BUTTONMAKERS GRAVELRENCHE are, I am sorry to say, in the doldrums. For decades, as fashions came and went, Gravelrenche buttons remained impossibly with-it and groovy, favoured by everyone from beatniks to dowager duchesses. Wander as one might from establishment drawing rooms to counterculture flophouses, the sharp-eyed buttonist would spot a Gravelrenche everywhere, on cardigans and greatcoats and weskits and spats. For this reason alone, banks and hedge funds and venture capitalists were willing to advance untold sums of cash to the company, without asking any questions or, indeed, specifying a date upon which they wanted their money back.

Well, the crunch de la credit has put paid to that jolly state of affairs, and over the past six months Gravelrenche has been unable to secure any funding at all, just at the point where button sales have dried up. In the last quarter, the company sold just three buttons, to a demented oligarch, and there is not a single order on the books. Grim-looking envoys from the banks have been seen loitering in the vicinity of the Gravelrenche buttonarium, armed no doubt with terrifying legal papers. The buttonmaking executives, however, are nowhere to be found. But those seeking them are asking the wrong question. Instead of wondering 'Where on earth are they?', they should be asking 'Who the hell are they anyway?'

Because the company purports to have been founded by brothers Pierre and Claude Gravelrenche, and operates in the sickly world of fashion, there is an assumption that, when located, its managers will be found to be stylish Eurosophisticates, the Jose Mourinhos of the world o' buttons. Well, 'boff!', as the French say. The presiding genius of Gravelrenche is in fact a toothless, evil-smelling lumberbones who lurks in a battered seaside boardinghouse and keeps all that cash he has eked over the years under his mattress. It is an enormous mattress. His name is neither Pierre nor Claude, nor even Gravelrenche, but something unpronounceable, the sort of chewy polysyllabic name that demands guttural improbabilities and an excess of phlegm if one wishes to speak it aloud properly. In the unlikely event that this man ever found himself in the boardroom of a bank, he would be turfed out on his grubby ear, mistaken for a vagrant.

What path did so unprepossessing a figure take to become the world's fabbest buttonmaker? Before answering that question, I want to digress for a moment to take a look at that word 'unprepossessing'. What's that all about? 'Possessing' means having, or owning. 'Prepossessing' would mean already having or owning, being in possession before the fact. The 'un-' prefix suggests that, far from already owning or having, one has not nor owns not. It is all a bit of a muddle as far as I am concerned, but that does not stop me from deploying the word as and when I want to, without a care in the world. If I wish to be verbose, then verbose I shall be, and a pox upon your strictures!

As for the path of the buttonmaker, that is a fairly straightforward matter. In spite of the unprepossessing figure he presented, and his unpronounceable name, and the stainage upon his clothing, and the vermin creeping in his bouffant, and his curd-like pallor, and his toothlessness, and his frayed elbow patches, and his stink, and his lasciviousness, and his grubby ears, and his filthy neck, and his gullet like a pelican's, and his squalid patrimony, and his lack of scruples, and his horrible head, and his one eye bigger than the other, and his unfamiliarity with soap, he had an almost eldritch talent for button design. He learned as much early, when he bumped into King Zog I, Skanderbeg III of the Albanians in the street, and the monarch was so smitten with the homemade buttons on the buttonmaker's homemade cardigan that he emptied his pockets of Albanian and other currencies' banknotes and coins, pressed them into the buttonmaker's mucky paws, and begged to be given the buttons in exchange. That very same evening, the King sported the buttons upon his fantastic kingly garb at a palace reception for wealthy Eurogits, and the buttonmaker's future was assured. Over the ensuing decades, clients such as Ringo Starr and Pat Nixon and Christopher Plummer and Krishnan Guru-Murthy and Monica Vitti and the brothers Miliband and Kathy Kirby bought hundreds and thousands of Gravelrenche buttons, even millions in the case of Mick Jagger, all of them under the impression that they were dealing with swish, effortlessly stylish Pierre and Claude.

Now we know those two dashing Gallic fashion titans never actually existed, and it seems the banks and hedge funds and venture capitalists may have caught on too, for ever since the crunch came, very little cash has been shoved under that enormous mattress in that fetid boardinghouse room on that windswept seafront where the buttonmaker lurks, chewing fish-heads and still, still, making his magnificent groovy buttons.