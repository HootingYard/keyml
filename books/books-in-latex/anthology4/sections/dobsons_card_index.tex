\chapter{Dobson's Card Index}

\begin{quotation}
'Along the path, glued to the window panes or hung on the bushes or dangling from the ceiling, so that all free space was put to maximum use, hundreds of little placards were displayed. Each one carried a drawing, a photograph, or an inscription, and the whole constituted a veritable encyclopaedia of what we call 'human knowledge'. A diagram of a plant cell, Mendeleieff's periodic table of the elements, the keys to Chinese writing, a cross-section of the human heart, Lorentz's transformation formulae, each planet and its characteristics, fossil remains of the horse species in series, Mayan hieroglyphics, economic and demographic statistics, musical phrases, samples of the principal plant and animal families, crystal specimens, the ground plan of the Great Pyramid, brain diagrams, logistic equations, phonetic charts of the sounds employed in all languages, maps, genealogies - everything in short which would fill the brain of a twentieth century Pico della Mirandola.' 

-- Ren\'{e} Daumal, \emph{Mount Analogue : A Novel Of Symbolically Authentic Non-Euclidean Adventures In Mountain Climbing}, translated by Roger Shattuck (1952; 1959).
\end{quotation}

THE ASTONISHING THING about the 'little placards' displayed by Father Sogol, the Professor of Mountaineering in Daumal's novel, is how similar they are to the immense card index maintained by Dobson, upon which he relied when writing his out of print pamphlets. Dobson would have approved, too, the Professor's method of displaying the cards - at least, sometimes. One of the pamphleteer's more irritating characteristics was his inability to settle on the keeping of his cards. At times, like Sogol, he pinned them up on every available surface. Then a frenzy would take him and he would tear them all down and shove them into one of his innumerable cardboard boxes. Marigold Chew reports that Dobson spent hours upon hours arranging the cards when they were in their boxes, ordering and reordering them according to various abstruse cataloguing systems. No sooner was he done than he would once again tip them out of their boxes and pin them up on walls and screens and pinboards and what have you. And of course, all the time he was adding new cards to the collection.

Much of Dobson's card collection perished in the Potato Building fire, and ever since researchers have been attempting to reconstruct it. This is probably an impossible task, but that doesn't stop them trying. The reward would be to create a sort of cardboard model of the innards of Dobson's pulsating brain - not to be confused with the cardboard model of the carapace of Dobson's brain which is currently being carted around the globe by a devotee. According to the timetable posted on the Cardboard Brain Of Dobson World Tour website, the cart with its precious contents is en route to one of the -nesses at the moment, either Skeg- or Dunge- or Foul-.

There was a flap of controversy some months ago when a previously unheard-of Dobsonist, one Bunko Chongue, claimed to have recreated an accurate cardboard box's worth of index cards. After painstaking study of clues littered throughout the pamphleteer's out of print works, and a visit to a stationery shop, the mysterious Chongue placed on display the results of his research. Purists' suspicions were roused by the fact that one had to pay an exorbitant fee to get through the door of the Nissen hut where the exhibition was held. Inside, however, there was an attempt to reflect the pamphleteer's indecision, with half the cards gummed to the walls and half crammed into a cardboard box. The cards themselves, too, demonstrated the variety that was characteristic of Dobson's collection, as it was of Sogol's. One visitor to the hut, later to denounce the show as a 'despicable farrago of falsehood and Nissen hut windowlessness', made a list of the cards he saw.

\emph{Instructions for the proper care of ostriches in captivity. Street map of Wivenhoe. Photo of a duck escaped from Rouen. Pig brain diagram. Bootlace aglet comparisons. Lopped Pol Pot poptart. Torn and rent stuff. Widow's buttons. Tips on bell ringing. Sandwich paste reviews. Drawing of ghost. Railway station smudge. Voltage statistics. Unsullied napkin from a remote canteen. Gunshot punctures. Drool from a pauper. Old Halob's hat measurements. Imaginary portrait of Tecwen Whittock. Muggletonian dinner menu. Fatal microbes. Winnipeg pumpkineer's cravat knot schema. Potter's duffel bag toggle analysis. Starling feathers. Stalin brooch. Desiccated plum pulp. Rubberised atomic sackcloth scrap. Latch. Pins. Bolt. Set of amazing stains. Devotional card of St Abodwo, arguably the patron saint of monkeys. Periodic table of the crumplements. Gravy recipe. Tabulation of Orwellian egg count. Snapshot of Schubert's grave. Mezzotint of Schubert's boot. Handwritten screed of gibberish. Lock of Pontiff's hair. Gummy ick. Definitions of flotsam and jetsam and plankton and krill and lemon meringue pie. The dust of death. The dewdrops of doom. Pointless scribblings.}

The Dobsonist who made the list, whose name has never been made public, was initially impressed by the exhibition. A few days later, however, in a letter to the \emph{Daily Nisbet Spotter}, he got into a fit of the vapours about the windowlessness of the Nissen hut, pointing out that, depending on the disposition of the purlins, it is quite simple to insert windows into the hut's frame. It is rare for one who spends his life studying Dobson also to have expertise in the construction of huts, whether Nissen or not, and this suggests that we may be able to identify the writer, if anyone can be bothered to sift through the documentation in the register, if there is indeed such a register, as the rumour mill insists is the case, though of course its existence may be a wild fantasy. We know of such phenomena, of fictional imagined registers, not least because Dobson himself wrote so forcefully of them in his pamphlet \emph{Wild And Unhinged Fantasies Regarding The Existence Of Wholly Imaginary Registers} (out of print). We can only guess how many index cards the pamphleteer used during the writing of this frankly blithering text, which Marigold Chew for some reason typeset to make it look like a pipsy-popsy book for infants.

Following the writing of his letter to the press, our unidentified Dobsonist had second thoughts about the exhibition. Where he had been positive, he now heaped execrations upon it, at first privately, shouting at his reflection in a mirror. He seems to have been oddly reluctant to bruit his views abroad. This changed after he spent a prolonged stay in a sensory deprivation tank and emerged hopelessly bonkers. He was seen wandering around various post offices babbling at anybody who would listen, and then he was seen scampering like a mad thing in the hills, and then he was seen weeping and rending his garments at the graveside of fictional athlete Bobnit Tivol. Then he vanished. He was missing during the dog days of the year, emerging as they petered out to publish his magnificent counterblast to Bunko Chongue, which I cited above.

By quoting his words, I do not necessarily lend them my imprimatur. For one thing, I did not see Bunko's show myself, so I cannot say whether he grasped the essence of the Dobson card index in all its lost glory. And for another thing, I rarely lend my imprimatur to anything. It can be rented at a cost, usually a cost involving blood and body parts, and undying fealty, and one or two tangerines, and seeds, and the plasticine head of a wolf on a stick.