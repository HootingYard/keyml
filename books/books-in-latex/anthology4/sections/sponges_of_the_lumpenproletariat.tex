\chapter{Sponges Of The Lumpenproletariat}

\emph{SPONGES OF THE Lumpenproletariat} is a magnificent work of social history, a companion volume to \emph{Blood Sausage Of The Petit Bourgeoisie}. As in the earlier work, the author has tracked down several representatives of the class and interviewed them in depth. Where before we learned so, so much about petit bourgeois blood sausage consumption, now we find out almost too much about the lumpenproletariat and its devotion to sponge, in both its cake and bathtime forms.

This is history turning the world topsy turvy, of course. We think of blood sausage as a workers' aliment, perhaps a snack to be eaten while tramping between factory and pigeon loft, or in the subterranean gloom of a mine, while canaries tweet in their cages as reassurance that no toxic gases are about to fell the pitmen where they huddle. It is interesting to note that birds feature in both those examples of lumpenprole snacktimes, yet the blood of birds is rarely to be found in blood sausage. Rare, yes, but not entirely unknown. One of the author's key findings in \emph{Blood Sausage Of The Petit Bourgeoisie} is that a small subclass of shopkeepers and rent collection men developed a taste for sausages made from the blood of starlings, nuthatches, and whooper swans. Such a commingling of avian gore is against the laws of God, which is perhaps why it was a short-lived fad. Not so with the more common sausages made from the blood of pigs and hens and goats, which are shown to be decisively petit bourgeois rather than lumpenproletarian. Accepted wisdom is knocked on its head.

I wish I could remember the name of the author. I would be lying if I said it was on the tip of my tongue, because it isn't, chiefly because I no longer have a tongue. A year or so ago, much like accepted wisdom, I too was knocked on the head, by a malefactor, who stole my wallet and my cardigan and, for good measure, broke both my arms and tore out my tongue. When I recovered, I was told I had been set upon by none other than the killer Babinsky, so I was lucky to be still alive. As far as I know, the maniac is still at large, so make sure you lock your doors and stick pins into your little wax Babinsky doll before you go to bed.

And just as we learned about the true social sphere in which blood sausage is prized, so in the new book by wotsisname we discover the importance of sponge to the lumpenproles. I had no idea, for example, that horny-handed sons of toil were quite so partial to dainties such as fairy cakes and sponge fingers. Equally surprising, given they are sometimes called the great unwashed, is the value placed upon bath sponges by the lumpenproletariat. Even those who use their bathtubs for coal storage treasure their sponges, often holding soapsud squeezing contests on some patch of waste ground behind their noisome hovels. I have a dim memory of seeing some such event when I was tot, as the rag and bone man's cart clopped by pulled by a great grey drayhorse. My pa told me the horse's bright and battering sandal was fettled for him by Felix Randal, the farrier. As an agricultural worker rather than a lumpenprole, Felix Randal was probably not that big on sponges, either cakey or bathtubby, although had he lived long enough to build up his fettling business into a chain of franchises, he might have aspired to becoming a petit bourgeois, and then he could have stuffed himself with blood sausage. It is unlikely, though not impossible, that he would have developed a taste for the blood of horses.

All in all, I cannot recommend this book highly enough. Quite apart from a text which bounds along in sprung rhythm, it has a mighty armature of footnotes, though they are printed in so tiny a font that I wish I had received my review copy before I was attacked by the killer Babinsky. So terrifying was his countenance, ever since that awful night I have had to wear shaded spectacles \`{a} la the Irish minstrel and tax evader Paul Hewson. Not only does this becloud my vision but, as you can imagine, it has shattered my reputation as a person of effortless swish and \'{e}lan.

Oh, and not only footnotes but gorgeous, gorgeous illustrations, in vivid colour, of sponges and blood.

And not only footnotes and illustrations but a really terrific index.

And not only footnotes and illustrations and an index, but a tear-out coupon in the frontispiece of each copy which, when taken to your newsagent, can be exchanged for a blood-soaked sponge. The idea, I think, is that one squeezes the blood from the sponge much as a lumpenprole would have squeezed soapsuds from it, and then one forms the blood into a sausage, and eats it, perhaps with a side helping of sponge cake.

And thus is the clash of classes in the grand sweep of history quietly subverted by this author whose name I am at a loss to recall.