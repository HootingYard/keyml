\chapter{Potter's Arch Or Potter's Crank?}

POTTER'S ARCH OR potter's crank? It's a choice you have to make, when tobogganing, in a split second. Pick the wrong manoeuvre and bones might be broken, or at the very least sprained, and you would almost certainly end up with a mouth full of snow. If your movements were impaired, as well they might be by dint of bone damage, and a fresh fall of snow occurred, from those expansive bleak grey skies, with little wind, you could be buried, all trace of you erased, for in the morning a passing hiker or cadet would see a smooth untrodden white blanket, stretching from here all the way across to where the woods begin at the foot of the mighty mountain. Your toboggan would be buried alongside you, some yards away, at the point where you were tossed from it into the snow, like an Eskimo rag doll.

Make the right choice, in that instant, between arch and crank, and no such calamity will befall you. You will continue zooming downward, whatever the gradient, with bumps and buffets to be sure, but joyously, until, as the slope evens out at the end of the course, you will slow gradually, and come to a halt at the scoring station. It is just a little hut, the station, where officials in woolly hats await you, and mark your time and elegance in their records.

They are passionless men, these officials. If you choose wrongly, between potter's arch and potter's crank, and are helpless in the snow with broken bones somewhere up on the slope, they send no search parties. They wait and wait, sipping Schnapps from their flasks, pointing to pines, scanning the sky, until the sun begins to set and they wend their way along the Hopfskag to the village, to homely hearths and warm beds. They will not even think of you, alone on the mountain slopes as snow falls from the sparkling night sky, burying you and your bashed-up dented toboggan.

You should not believe what you have heard about big dogs coming to snuffle you out, with brandy-barrels fastened about their necks. There are no such hounds in the Hopfskag. It is said they are frightened away by the mountain spirits, the groaning wraiths that prey upon the souls of crashed tobogganists.

That is why, in that split second you have to choose your manoeuvre, to make that decisive potter's arch or potter's crank, you should trust to neither skill nor instinct, but to the mountain spirits. Offer yourself to them, brain and bone and body and soul, frame and core, in a howl of subjugation to their power, and make your move. They will tell you whether to arch or to crank. But be warned. They are mischievous and fickle. The dogs learned that long ago. Tobogganists have not, yet.