\chapter{Potty Baron}

ONCE UPON A time there was a baron who was completely potty. His head was roughly the same size and shape as most heads you'll come across, but the stuff that went on inside its spongiform crannies was deeply, deeply weird. To give one example, sometimes, when the baron looked in the mirror, he saw a swan staring back at him. Granted, in those days looking-glasses were rarely as smooth and as free of flaws as those that are manufactured by the great mirror-makers of modern times, but barons and swans were pretty much the same as they are now, so one can't really blame the looking-glass for the baron's visual delusion. It is not as if he had a particularly long or sinuous neck, either, which might have been some excuse. Like his head, his neck was unremarkable, and could have been mistaken for anybody else's neck at a glance, or in dim light. And the light was often dim in those days and in his baronial bailiwick, for they did not have electric lightbulbs or fluorescent strip lighting or Kleig lights or any of the other immensely illuminating benefits of our day. What they did have were candles, of tallow or blubber or occasionally beeswax, if the bees were happy at their work. So the artificial light the baron had in his castle was of a flickering and smoky nature, and much, so very much, of the castle was steeped in shadow and gloom, for candles could not be placed everywhere. That would have been too expensive, and too much of a fire risk. By the standards of the time, the baron was a wealthy baron, but even he couldn't afford to light up so enormous a castle. Perhaps he had gone nuts because the castle was so bloody big. There must have been times when he got lost, roaming around the place, looking for one of his toilets or pantries or dining halls or rumpus rooms. The thing about this baron was that he lived alone, shunning the company of courtiers and sycophants and hangers-on and even of servants. It was unusual in those days for anyone to know such solitude, unless they were clapped in a dungeon somewhere for grain theft. In those circumstances, the villain could be chucked into an oubliette and chained to its wall and all but abandoned save for once a day having a mouldy loaf and a pig's bladder filled with brackish water thrown down to them. At least they got fed, whereas the potty baron had to fend for himself. Fortunately, his pantries were very well stocked. His uncle, the previous baron, had not been potty at all. He was a careful and calculating kind of baron, who made use of his reason, and laid up so many supplies to guard against famine and war and pestilence that he could have lived on the contents of his pantries for hundreds of years. But of course, not even in those days did people live so long, no matter what you might have read in stories. The potty baron's uncle was barely forty years old when he died after being kicked in the head by a rampaging hog. His head, too, had been of a standard size and shape, but it certainly wasn't after the hog had done with it. I'll spare you the details, because you might be sick. The baron vomited all over the shop when he was brought to view his uncle's hog-dented head. It looked unnatural and misshapen, like a Frankenstein experiment gone awry. The potty baron would not have put it in those words himself, of course, for this was hundreds of years before Mary Shelley was even born, but the idea of somebody, or something, creating artificial life was already present in the baron's potty head. He was ahead of his time in that sense, though it would be more accurate to say that he was out of his time, out of any time, existing as he did in a kind of weird and bonkers daze with little or no purchase on reality. Every so often, the notions sloshing around inside his head made sense, as was the case with his vow, upon inheriting the baronetcy, to expel all the hogs from the castle. What he might lose in future hog-meat banquets he would gain in not getting his head kicked in and an untimely demise. Thus he acquired the services of Skippy, a perky little dog which chased all the hogs away, petrified as they were by its yap. The baron gave Skippy a tabard to wear, made out of whole cloth, to advertise its official baronial status. But once the hogs were driven out the baron grew irritated by Skippy, and had him placed in kennels. Never again was any living thing to share the enormous castle with the potty baron, except for tiny, barely visible insects, and a great variety of bacteria. Where his uncle had maintained an army of drudges armed with mops and pails and the most up to date cleaning substances available to a baron of long ago, his reclusive nephew seemed quite content to allow filth and muck to cake every castle surface. The pong was indescribable, but the potty baron was impervious to it, as he tended to wander his domain with plugs of fine linen stuffed up his nose. Like the rest of his head, there was nothing out of the ordinary about his nose. It had a somewhat corvine cast. He removed his linen plugs when out and about in the open air, striding along the battlements or taking a turn around the moat in a rowing boat. In his uncle's day the water in the moat had been channelled from a burbling spring in a neighbouring dell, and many fat and healthy fish swam within it, waiting all their lives to be yanked out by a baronial angler and taken to the kitchen and fried for the baron's breakfast. The potty baron had dismissed the anglers as he rid himself of everybody else, and he stopped up the channel from the burbling spring, and now the water in the moat was stagnant and all the fish had perished and only slimy green tendrils of weed were to be found there. In places the weed was so thick that the potty baron had to row with all his might to force his rowing boat through it. This proved to be splendid exercise, and was one of the reasons why the baron did not waste away in his germ-ridden solitude. He especially liked to row around the moat during cataclysmic storms, daring lightning to come strike him. He would even shout at the sky, as he rowed, struggling through the glut of moatweed, in the night, his cries louder than the thunder, but not once was he struck. The castle was, though, regularly, sometimes catching fire as a result, just as it would have done had the potty baron been able to fill every inch of it with candles. These fires never took hold. Invariably they burned themselves out, conquered perhaps by damp and mould and puddles. There were hundreds of puddles within the castle, for most of the roofs and ceilings had rotted away, and when the rain poured down, as it did most days, for this was a very wet country, it poured straight down into the toilets and pantries and dining halls and rumpus rooms, forming puddles. The potty baron splashed through them like a duck. Certainly his movements were more akin to those of a duck than to a swan such as the one which glared back at him from his looking-glass, occasionally. For he did not always see a swan in the mirror. Sometimes he saw himself, linen plugs in his corvine nose, eyes curiously resembling those of a coquettish courtesan, hair matted and adorned with berries and string. At other times, the most terrifying times, he saw nothing at all in the mirror. When this happened the potty baron would tremble and shriek. Had he known of vampires, he might have thought himself one, for as we know these befanged and becaped Transylvanians are never visible in mirrors. But the baron's castle was very very far away from Transylvania, and he had never heard these tales, and not even within his demented cranium had the thought of a bloodsucking count of exclusively nocturnal habits who cast neither shadow nor reflection ever presented itself. As far as the potty baron was concerned, on those days when his mirror was blank he assumed he did not exist. It was the awful horror of mortal obliteration which caused his trembling and shrieking. He learned, albeit slowly, to overcome his mind-snapping fear by turning away from the looking-glass and heading straight for the nearest pantry, there to stuff himself with some of the foods preserved by his canny uncle. He would eat so much that he would eventually fall asleep and when, later, he awoke crumpled in a heap on the pantry floor, surrounded by crumbs and stalks and pips, he simply stood up and dusted himself off and went about his day, or night, whichever one it was, shoving the memory of his dreadful annihilation into a pocket of his potty brain. Then he would avoid mirrors for days, or weeks, keeping as much as possible to those parts of the castle where there were no looking-glasses, or where those that had existed in his uncle's day had been smashed, or just shattered by themselves, through neglect and hopelessness. One such place was the castle belfry. The old baron, and all the barons before him, had rung the castle bells for a purpose, to celebrate a victory in battle, to warn of plague, or to summon the peasantry to come and pay their tithes and taxes, under pain of being flung into dungeons. Not one of these things was of the remotest concern to the potty baron, whose baffling interior world was oblivious to them. When he set the bells pealing, it was to quite unfathomable ends. He may have been warding off sprites, or calling them unto him, or any of a thousand other incomprehensible reasons. And yet his bell-ringing was surprisingly melodious. We can surmise that it may have sounded not unlike a Shostakovich symphony, or a Gabriel Faur\'{e} sonata for cello and piano, done by bells. Afar, in the fields, the peasants hearing the potty baron's carillon would cease their hoeing and grubbing, and mop the sweat from their brows, and glance at each other, a thousand years of rustic wisdom shining from their eye sockets, and they might then form a ring in the field, mud-caked hands holding mud-caked hands, and they would bow their heads and regard the clogged soil they worked day in, day out, all their lives, as the rain poured down upon their heads, heads for the most part of the same normal size and shape as that of the baron, save for a few peasants with microcephalic or macrocephalic or hydrocephalic heads, who despite their condition still tilled and ploughed and scrubbled with all the other peasants, tirelessly, muddily, soaked by rainfall, in their peasant rags, frayed and filthy, from dawn to dusk, and often throughout the night, beset by demons of the darkness, working the land with broken spades and blunted forks and desperately, with sticks and twigs, in icy winds, and stooped together in their fields, forming a circle, they snicked the wax out of their ears and listened, each one of them thinking 'The potty baron is clanging his bells, and all shall be well in this life and all shall be well in the life to come'.