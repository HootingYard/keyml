\chapter{Character Flaw of Mediaeval Peasant}

HELLO. MY NAME is Cleothgard and I am a mediaeval peasant. I am calling to you across the centuries because I want to tell you about my character flaw. It is not, I am afraid to say, a character flaw that would elevate me into the realm of the tragic. That would be a splendid flaw to have, but I am a mere peasant, and as I look about me in this vale of tears it has not escaped my attention that tragic, and indeed heroic, character flaws tend to be displayed by princelings and such. Mine is more what you might call a mundane flaw. One of the reasons I decided to bellow so hectically that time itself is bedizened and shrunk is that I know with a fair degree of certainty that no poet nor playwright is ever likely to consider my character flaw a fit subject for their pen, or I should say quill, or scratchy stick. I am not Prince Hamlet, nor was meant to be. I don't even know who Prince Hamlet is, or was, or will be, and I was always meant to be a peasant. 'Twas writ upon the stars.

My character flaw is a tendency to overdo the grovelling when confronted by a baron. When I see one approach, upon his horse, in all his finery, glittering and clanking, accompanied by his retinue, I immediately start to snivel and slobber and I pitch myself forward face down into the muck. Mediaeval muck is much, much filthier than your modern muck. It oozes and stinks and harbours all sorts of minuscule disgusting life-forms, things you have eradicated through science and hygiene. While thus prostrate, I begin to groan incoherently. What I am actually trying to express is the sense that I am but a worm unfit to exist on the same planet as the baron, who is brighter than the sun and completely fantastic, but my abasement is such that I cannot form the proper words with my mouth, which in any case is by now packed with mud.

Bear in mind that at this stage, the baron and his retinue have only just hove into view on the horizon. This is partly what I mean by overdoing the grovelling. My fellow mediaeval peasants are all still going about their business, tilling the fields or scratching at their buboes or, in the throes of the \emph{chorea imagnativa aestimative}, dancing in a frenzy. When the baron gets closer, a goodly number of them will tug their forelocks and dribble with happiness at the sight of him, but I feel this urge to outdo them. That is my character flaw.

And it is made all the more pathetic by the fact that the baron will not even notice my existence. I am pretty certain this is the case, for over the years, not a single baron passing through our bailiwick has ever acknowledged me. They might command their retinue to smash up the mediaeval farming tools and hack at those tilling the fields with their big sharp swords, or they might burn down our huts, or, very very occasionally, they might turn out to be a so-called 'good baron' and distribute alms and largesse. But because of my ludicrous grovelment, flat on my belly in the muck, groaning away, I seem to escape their attention. There is, I suppose, an advantage to this in that I have never had one of my limbs sliced off or smashed to pieces. Thereagain, nor have I been presented with a groat by a good baron's alms-giving dwarf.

Swings and roundabouts. Fate is meant to decide the lot of a mediaeval peasant like me, but the nature of my character flaw is such that I become a sort of invisible peasant, merged with the muck in which I sprawl, and so Fate passes me by. If I could, just once, stand straight and tall - well, not tall, exactly, because nutritional deficiencies mean I am stunted and shrivelled - but if I could at least stop abasing myself quite so preposterously as soon as I spot a baron half a mile off, then who knows what I might reap?

I dare say in your world o' the future there are still peasants with character flaws, and good barons and bad barons. And I expect the peasants still till and scratch and dance, and the barons are still really terrific and shine so brightly. Fate will arrange things just so, I assume, even though Fate has taken one withering look at me and left me where I am, slobbering in the muck.