\chapter{Dobson's Cacodaemon}

EVEN THE MOST learned of Dobson scholars has difficulty with his pamphlet \emph{How I Thwarted My Cacodaemon With A Pointy Stick And Some Bleach} (out of print). For one thing, who knew Dobson had his own personal Cacodaemon? It is never mentioned elsewhere in the canon, nor does it make an appearance in his voluminous diaries. Occasionally, like other indefatigable diarists, Dobson had recourse to codes and symbols, but all of these have been deciphered after decades of study by Aloysius Nestingbird and their significance revealed in his magisterial survey \emph{The Meanings Of Every Single One Of Those Enigmatic Symbols And Scribbles In The Journals Of The Out Of Print Pamphleteer Dobson}, itself, alas, now out of print too. Nestingbird realised that the childish drawing of a horned and hooved goaty devil figure brandishing a spit fork, usually done in red ink, which appears in the diaries from time to time without additional written comment, had nothing whatsoever to do with some putative Cacodaemon of Dobson's, but was simply the pamphleteer's idiosyncratic manner of noting that Hungarian football ace Ferenc Puskas had played a blinder in a match that day. Puskas was never known by a nickname aligning him with a devil of any kind, but Nestingbird shows convincingly that the inside of Dobson's head was rarely in accord with the wider world.

Nor do we find any reference to a Cacodaemon in any of the recorded utterances or memoirs of Marigold Chew. Surely the woman who knew Dobson better than anyone else would have known of it? There is a possibility, of course, that she did know, but kept a judicious silence for fear of exposing her inamorato to ridicule. But then, there was much else that was preposterous about Dobson, from his boots to his handwriting, and she seems to have happily acknowledged, even celebrated, his various absurdities.

What of the pamphlet itself? In its startling opening sentence, the pamphleteer announces that he is going to tell us all about how he thwarted his Cacodaemon with a pointy stick and some bleach, and that if his prose were paint, in this pamphlet it would be matt rather than gloss. The fact is, Dobson continues in some of the glossiest prose he ever wrote. Indeed certain passages are so glossy that Nestingbird, among others, has recommended reading it through a screen or veil to dull its unearthly sheen.

Dobson gives his Cacodaemon no 'back story'. He does not explain when it first began to haunt him, nor how terrible, or otherwise, has been its impact upon his life. It merely shimmers before him after breakfast one drizzly morning in April, and he reports this matter-of-factly, as if it is a familiar accompaniment to his post-breakfast drizzly April morning doings. On the particular morning of which he writes, Marigold Chew is away, which may in itself be significant. Dobson does not tell us where she has gone, but by checking the calendar one can conclude she was probably on one of her periodic jaunts to Shoeburyness as part of the bottomless viper-pit study group.

Dobson then recounts how he loses patience with his Cacodaemon. It is making demands upon him, as we are given to understand it 'always does', and the pamphleteer snaps. He goes to the broom cupboard and takes out a pointy stick, and dips the end of the stick in bleach, and charges across the room at the Cacodaemon, shouting his head off and threatening to impale it upon the stick. At this point, with a hideous sort of sucking and seething and squelching noise, the Cacodaemon seems to implode in upon itself. Bringing himself to a halt just before he clatters into the wainscotting, the pamphleteer peers down at the floor and sees a tiny smudge of noisome goo. This, he suggests, is all that is left of his Cacodaemon. He leans the pointy stick against the wall, and goes to the draining board to fetch a rag. He wipes the smudge with the rag, pours more bleach into a bucket, and drops the rag into the bucket. There is, he writes, 'a faint echo of the sucking and seething and squelchy sound, as if heard through a funnel blocked with pebbles and dust'.

And thus the pamphlet ends, save for a rather curious colophon from which not even Nestingbird has been able to wring any meaning. I suppose we have to ask if Dobson was just making the whole thing up. We know there were times when he felt compelled to write a pamphlet even when his head was empty of ideas. Perhaps this was one of those times. Further light will no doubt be shed on the matter with the publication of Aloysius Nestingbird's forthcoming study \emph{Dobson's Head, Its Innards, And What They Reveal About The Colossus Of Twentieth-Century Pamphleteering}.

I had hoped to be invited to write an introduction to this book, but I was told, in a dream, that there would be no such invitation, that Nestingbird had never heard of me, and that my pretensions to Dobsonist scholarship were flimsy and pathetic and doomed. Hard to argue with that, belched and spat out as it was from the fiery maw of a Cacodaemon.