\chapter{A Celebration Of The Bufflehead In Prose And Song}

EVERYONE HAS THEIR favourite type of duck, and for Prudence Foxglove it was the bufflehead. In the summer of 1894, the unsung Victorian genius took a break from writing her daringly modernist plays and compiled a fat volume entitled \emph{A Celebration Of The Bufflehead In Prose And Song}. It was illustrated with her own cack-handed pencil drawings which, it has to be said, look much more like teal or mergansers than buffleheads. The dramatist had thousands of copies of the book printed at her own expense, a cost she could easily afford after the astonishing success of her plays such as \emph{See How The Intoxicated Brute Wallows In A Swamp Of Moral Turpitude Until His Ravaged Soul Is Uplifted By Muscular Christianity In The Personage Of A Pugilist Vicar} (1894). The closing scene in that fine play, in which the Reverend 'Nobby' Attenborough rains his beboxinggloved fists down upon the head of the intoxicated brute, is unforgettable.

Prudence Foxglove had been collecting snippets about buffleheads from books, periodicals and food packaging since childhood. Most girls of that era would have pasted their cuttings into a scrapbook, but Prudence disliked both paste and scrapbooks, and instead she stuffed her snippings into an ever-burgeoning accumulation of burlap gunny sacks. When a sack was plump and full, she stitched it up using exemplary needlework skills, and entrusted it to the keeping of one of her many gardeners for use as a pillow. 

The Foxglove family estate had extensive grounds, grown wild over generations of neglect, and it was Prudence's mother Hepzibah, improbably green-fingered, who determined to tame them, employing hundreds of snag-toothed peasants from the surrounding hovels to dig and prune and hoe and harrow.
 Under the spell of the social reformer Rufus Crank, Hepzibah Foxglove built a model village for her gardeners to live in. Each had their own hut, with guttering and drainage and a spigot and a sink and a pallet with a mattress and a shelf of improving tracts and prayerbooks and a picture nailed to the wall of Christ commanding the woman to throw the sack full of beetles and locusts and flies and snakes and hornets and wasps into the sea. It was unusual in those days for gardeners and other servants to have pillows, so Prudence's gunny sacks were particularly welcome. At her mother's insistence, she had taken the precaution of seeking approval from Rufus Crank himself. By then over ninety, the reformer wrote back to her in the famous 'pillows for gardeners' letter, in which he laid out a set of principles we would do well to abide by today, if, that is, we still had gardeners in huts in the grounds of our estates.

\emph{Be warned,} he wrote, \emph{that a gardener plucked from his hovel and given a hut with modern appurtenances such as a sink and a spigot may get hoity-toity if allowed to rest his oddly-shaped head on a pillow. Yet Christian compassion tells us he must be given the chance so to do. The risk of hoity-toityness can be tempered, if not wholly eradicated, by observing some general principles.} 

\emph{1. Rent the pillow to your gardener rather than giving it to him outright.}

\emph{2. Stuff it not with kapok nor duck feathers nor soft downy empadment, but with gravel or pebbles or sand.}

\emph{3. Paper and cardboard cuttings on the subject of buffleheads, or of birds that look roughly like teal or mergansers if drawn cack-handedly are an acceptable stuffing, in extremis.}

\emph{4. Use burlap gunny sacks or other rough fabric.}

\emph{5. Forbid pillow-cases.}

Young Prudence got herself into a tizzy when trying to decide how to charge the gardeners for pillow rental, for they were unfamiliar with coinage, being peasants. Then, at one of her mother's regular soir\'{e}es, she met the so-called 'potato economist' Hicks, an extraordinary man whose theories ought to have resounded down the ages, but did not, possibly because he expounded them in prose so dense and clotted and awash with spelling errors that it was, and still is, mistaken for gibberish. When he spoke, however, Hicks was a model of clarity, and as he stood next to a blazing hearth at the soir\'{e}e, his massive Victorian beard at risk of catching fire, he told Prudence to accept payment of pillow rent in potatoes. It was advice she took to heart, so much so that as the years passed, and she grew to adulthood, and found worldly success as a playwright, still she found time each week to potter from hut to hut in the grounds of the estate of which she was now the mistress, demanding - and receiving - a potato from each of the increasingly creaky gardeners who at night rested their heads on pillows stuffed with her collection of bufflehead clippings.

Many of these potatoes were destined for the grubby lodgings where Hicks lived in penury. Every Thursday morning, Prudence Foxglove emparcelled a potato and took a horse and cart down to the village post office, where a new-fangled pneumatic funnel system sent packages whizzing across the land. What Hicks did with all these potatoes is an enduring mystery, for his diary is, if anything, even less comprehensible than his published writings. 

At one point it was thought a small clue was to be found in the startlingly fat book \emph{Table Talk Of Bearded Victorian Intellectuals}, wherein nestles a report of a figure thought to be Hicks standing next to a blazing hearth making canny observations upon the dietary habits of impecunious bearded Victorian intellectuals, in which the potato features prominently. But Hicks, famously, viewed the potato as a currency, and he is unlikely ever to have eaten one. He was, in any case, an unrepentant carnivore, having trained his stomach to digest liver and gravy in huge quantities, as recommended by the nutritionist Bristow, another bearded Victorian intellectual but one who had never, so far as we know, been invited to one of Hepzibah Foxglove's soir\'{e}es.

After her mother was killed as a result of an implausible yet all too real cartographic mishap, Prudence at first continued with the soir\'{e}es. Hicks, Jetsam, Baxter, Coughdrop, Fig, and Figby were all regular visitors, trudging across the filthy fields to the estate in all weathers, keen to propound their various nostrums and being plied with hibiscus syrup and water biscuits while Prudence's sister Drusilla tinkled sentimental songs at the piano. Drusilla's repertoire was small, but she compensated for it by devising hectic improvisational passages, so that a mournful dirge such as \emph{Bring Me Your Winding Sheet, Oh Mother Of Mine}, which lasts about two minutes when played straight, could last up to sixteen or seventeen hours, by which time the assorted bearded Victorian intellectuals had often become so argumentative that shouting and fisticuffs were not uncommon. Prudence's drudge complained about forever having to mop up bloodstains from the rug adjacent to the blazing hearth, and threatened to seek drudgery elsewhere. Petrified of losing so tireless a mopper, Prudence decided to call a halt to the soir\'{e}es. 
This left Drusilla at a loose end, until she was persuaded to transcribe her piano pieces for a chorus of voices, and, on their afternoons off, she trained and rehearsed the gardeners, leading them on loud and lusty singsongs roaming across the fields and hills and alarming cows and pigs and ponies.

Without a drawing-room full of pontificating bearded Victorian intellectuals, Prudence Foxglove too may have found herself in want of anything to occupy her. But instead she relished the solitude, and took her first faltering steps towards writing. 

The plays which would make her rich and famous lay in the future, and her early efforts were in verse and prose. She tried her hand at detective stories, ballads, non-fiction (\emph{A History Of Eggs}), an epistolatory novel, and automatic writing dictated from the spirit world. For the latter, she had her arm hoist in a canvas sling, her hand grasping a steel pen, and stunned herself with laudanum. Publishers rejected everything she sent to them, with the single exception of the magazine \emph{Mawkish Chaff}, which accepted her poem \emph{The Hopeless Hollyhocks}. Her excitement was somewhat dimmed when it turned out that the editor was a Hicksite, and paid for her poem in potatoes.

Nevertheless inspired by her appearance in the public prints, albeit in a magazine with a tiny circulation, Prudence fired off a series of similar pieces, including \emph{The Lugubrious Lupins}, \emph{The Dismal Dahlias}, and \emph{A Spinney Choked With Marshland Weeds}, and impressed the editor with her industry. He invited her to visit his office, in a grim northern mill town, and so it was that Prudence found herself taking her first ever railway journey. The trip itself was uneventful, but not so her arrival at the grim northern mill town railway station. 

The editor of \emph{Mawkish Chaff} came to meet her in person. He was a dashing cad with exquisite manners and the morals of the sewer, and as she disembarked from her train, he swept her into his arms, protested his undying passion for her, crossed the platform to bundle her on to a small branch line train and carried her away with him to a boarding house in a shabby seaside resort for a week of sinful debauch. Initially smitten, Prudence soon came to her senses. On the pretext of popping out of the boarding house to buy a couple of choc-ices from a sea-front kiosk, she went straight to the police station and shopped her seducer. As the cad was carted off to clink by a team of rozzers, she took the train back to the grim northern mill town, let herself into the offices of \emph{Mawkish Chaff} with a key she had secretly had cut, and set about running the magazine herself.

Prudence Foxglove proved to be an editrix of genius. Pages once filled with sentimental pap now played host to extraordinary talents, as she called on all those bearded Victorian intellectuals who had attended the soir\'{e}es to pen essays and manifestos and epic poems and novellas and visionary burbling and, occasionally, automatic writing dictated from the spirit world while their arms were hoist in canvas slings. Hicks himself was given three entire issues to expound his theories of potato economics, and then a further three when he decided he had not quite finished, followed by a couple of supplementary issues to tie up a few loose ends, and a Christmas Special to repeat the more pertinent points. Prudence was the first to publish both Fig's hallucinatory ravings and Figby's twee nature notes. It was in \emph{Mawkish Chaff} that Coughdrop predicted, in the coming century, the appearance upon the world stage of a pamphleteer he mistakenly identified as 'Bodson'.

Throughout this blizzard of editorial activity, Prudence continued to cut out clippings about buffleheads and to send them to Drusilla, who had been charged with stuffing them into burlap gunny sacks and collecting the pillow rent from the gardeners. It is curious that not a single cutting was ever taken from Prudence's own magazine. One can search through the bound volumes and never once find the word 'bufflehead', whereas the names of other types of ducks turn up frequently, particularly sheldrakes, which were an obsessional interest of Baxter's, no matter what he was meant to be writing about.

Although she did not wholly neglect her own work during these years, penning a so-called 'pneumatic romance' and a study of the beards of Victorian intellectuals, Prudence had yet to hit upon the formula that she would make her own. The germ of that first, ground-breaking play was a walk she took through a patch of broken ground in the shadow of a grim mill one Sunday. She chanced upon a derelict, sloshing a bottle of turps and babbling to himself, and approached him with a charitable tuppence in her outstretched hand. The human wreckage grabbed at her wrist and would not let go, and Prudence realised with an awful pang that it was her boarding house cad. His manners were no longer exquisite and his morals were no longer even of the sewer, for they had been utterly blasted away through strong drink and turpentine. She had him carted off to an asylum in the hope that he would one day recover, and visited him there on subsequent Sundays, listening to his incoherent jabbering as he told her his story, from the despair of prison to the greater despair of the broken ground in the shadow of the grim mill. She felt impelled to share this terrible tale with the world, and it became the basis of her play \emph{The Dashing But Debauched Cad And His Descent Into A Netherworld Of Turpentine-Fuelled Depravity} (1894). Writing it, she realised her inborn talent for dramatic dialogue, stage directions, interludes of knockabout comedy, emotionally wrenching climaxes, and daring modernist interventions, the latter owing something perhaps to Drusilla's piano and choral techniques. From the very first performance, in which the legendary actor-manager Sir Hector Bombast played the dashing but debauched cad, the play left its audience stunned. Prudence Foxglove had found her Muse, and her future was assured.

She returned the reins of the magazine to the now partially-recovered cad and headed back to her estate, where her many decrepit gardeners and her drudge welcomed her as a heroine. Drusilla, too, was overjoyed to have her sister back, for she had grown lame and the weekly traipsing from hut to hut to collect the pillow rent had become a sore trial to her. There was a folly next to a ha-ha in the grounds of the estate, and Prudence took it as her writing-room. The plays poured out of her, on an almost daily basis, and soon enough her dramas were being put on in every theatre in the land.

Thus it was that, come the summer, she decided to take a break. One fine Friday afternoon, she took delivery of a set of sandbags and wheeled them in a barrow around the gardeners' huts, retrieving each of her burlap gunny sacks and leaving a sandbag in its place. Then, holed up in her folly by the ha-ha, she emptied out the sacks and laboriously copied out each and every bufflehead-related clipping into a series of exercise books. When she was done, she took the horse and cart to the village post office and sent the embundled books to a printer. A month later, she received via the pneumatic funnel system thousands of copies of \emph{A Celebration Of The Bufflehead In Prose And Song}.

Unlike her plays, it was not a success. The public had come to expect from her stern moral invective, drunken brutes, comedic japes, crippled orphans, vapid drivellings, pugilistic vicars, and daring modernism. None of these was to be found in what was, after all, just a forbiddingly fat anthology of miscellania about ducks. When Prudence Foxglove died in 1922, all but one of the copies was found rotting in packing cases in the cellar of her estate. And that single, presumably sold, copy? The story is told that its owner, a grand-nephew of Hicks, had it in his luggage when he stepped aboard the airship Hindenburg in Frankfurt on the third of May 1937.

