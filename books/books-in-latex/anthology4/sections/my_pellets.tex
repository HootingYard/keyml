\chapter{My Pellets}

YOU ASKED ME to tell you about my pellets, so here goes. Some of the pellets were regurgitated by cats, some were gobbed up by owls, and some are made of metal, to be fired from a shotgun. All my pellets fall into one of those three categories, I think. If I think some more, which I am not going to do, it might occur to me that some of my pellets have a provenance other than those three, but I can always issue a corrective at a later date, after I have thought some more. Even if I recall some other type of pellet, it remains the case that the vast majority of my pellets are either those vomited up by cats or owls or those made of metal meant to be shot at something, such as a crow or a scarecrow or an irritating person.

Yes, I have been known to fire metal pellets from a shotgun at persons who irritate me. I am sure that is lamentable, even criminal, behaviour, but we all have our breaking points, and if you start to moralise with me and suggest that I flip my lid a little too readily, I might well agree with you. But you and I have not sat on the same buses nor traipsed the same retail facility aisles, so you would be better off holding your tongue.

Of more interest than my metal pellets are my cat and owl pellets, which are of course organic, and often contain the barely recognisable remains of small mammals such as mice and fieldmice, or of the tinier birds. Some of these pellets have been dropped in my doorway, as gifts, and some I have collected, on trawling expeditions in the forest. It is an extensive forest and home to many owls. They perch on the branches of trees and hoot as night falls, and in the darkness they swoop upon mice and fieldmice, and in the morning they gob up pellets, and there is me, with my sack, wandering the forest at dawn, on the lookout. I wear a charm bracelet when I wander in the forest, to keep me safe from kelpies.

It is unfortunate that my charm bracelet does not protect me from irritating persons. I did try to modify it, by adding hawked-up pellets from a bird of prey, from a hawk in fact, but still I was beset, on buses and in retail facilities, by the rude and the gormless. Thus it was that I added to my collection of metal pellets, for the firing of them, from a shotgun, as necessity demanded.

Watching a cat hawk up a pellet from its innards is an educative experience. There often appears to be much undigested grass from lawns impacted in the pellet, and yet I can never recall seeing a cat feeding upon grass, much as if it were a cow, which we are used to seeing eat grass. At least, I am used to such a sight, for I often watch cows, it is my hobby. If I have been put somewhere where cows are scarce or non-existent, I will travel to find them, so I can watch them, of an afternoon, or of a morning, or even all bloody day if I am in a cow-watching mood. Cows becalm the soul. And yet as far as I know they do not regurgitate pellets, as cats and owls do.

I do not watch owls, I simply trace their presence in the forest, armed with my sack, and collect the pellets they have gobbed up. I do not think it would becalm me to watch owls, while wearing night-vision goggles, in the depths of the forest. I would always be on my guard against kelpies, even when wearing my charm bracelet. My heart would be hammering.

Irritating persons are, of course, the opposite of becalming, and impervious to the magick of my charm bracelet. That is why they have to be dealt with by metal pellets from a shotgun. Peppered with pellets, they run away screeching. Before I wore my charm bracelet, I used to run away screeching from kelpies. I would hoist my sack upon my back and go wandering into the forest, at night, to collect pellets hawked up by owls, and very often I would be pursued or set upon or threatened or menaced by kelpies, and, with an empty sack, run screeching until I was safely back in whatever hut I had been put, by the authorities. My cat and metal pellets outnumbered my owl pellets to a great degree, there was a terrible imbalance, and it was my recognition of this that led me to make the charm bracelet. I followed instructions from a pamphlet written by a man whose life had been blighted by kelpies but who had been able to deter them by wearing a bracelet of beads and baubles and pellets and bones and teeth and feathers and sugarcubes. It was, for me, always an awful temptation to suck upon and crunch the sugarcubes on my charm bracelet, for I have a very sweet tooth, but I enrolled in a twelve-step programme run by Sugarcube Suckers And Crunchers Anonymous and that sorted out my head.

You have to sort out your head, sooner or later, wherever the authorities have put you, be it a hut or a shed or an outbuilding. I found that developing an interest in my pellet collection was the thing that rescued me from a hopeless, pelletless existence. See them, my pellets, all aligned and catalogued, in my cabinets, the cat and the owl and the metal, and not a kelpie within a hundred yards of my hut. I am a happy man.