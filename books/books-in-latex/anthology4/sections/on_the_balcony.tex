\chapter{On The Balcony}

I WAS DROOLING into a pewter pot held by my unpaid companion on the balcony of a sanatorium upon which snow was falling, snow which showed no sign of melting, and I realised it was going to be a hard winter. My companion was a deaf mute, and I could have used sign language to communicate this insight to her, but I was wearing thick woolly mittens against the cold, so I let it pass. She could not lip read my language, and even if she could, my beard is now so majestic that my lips are thoroughly obscured. It further occurred to me that, in a hard winter, I might perish on this balcony, covered in snow. Well, I thought, worse things happen at sea. It is a notion I have often clung to for comfort, but God knows, given some of the calamities that have blasted me over the years I do wonder just how much worse things could have been had I been aboard a boat or a raft. The idea that maritime disasters were worse than anything that could possibly happen to me on land was instilled in me at my mother's knee, and kept me shorebound even in the teeth of press gangs, blandishments and wanderlust. The press gangs were the hardest to overpower, of course, lumbering brutes flailing cutlasses at me as I staggered drunkenly from a dockside tavern at dawn. Lucky for me that even in my cups I have a gaze that could wither a row of hollyhocks. I mean that literally. It was a trick I picked up from a mountebank in a fairground tent, and one I deploy sparingly. Would it have helped me to face down imperilment at sea? I will never know, it is much too late for me to go a-sailing now. How late it is, how late. That was the title of a book by a sweary Scotchman, I recall, a book that won a prize. I doubt he had a clue what he was talking about. I stare at the falling snow from my balcony, drooling into a pot, and I think I'm the one who knows what `late` means.
The pewter drool-pot is a memento from the long ago morning of my life. It belonged to my mother, as it had belonged to my grandmother and, I think, to my great-grandmother, all of whom kept buttons in it. Women of those eras amassed thousands upon thousands of buttons, never discarding a single one. Into the pewter pot a button would go, with all its fellow buttons. When the pot came into my possession I stuck it in a cupboard for years. Then, when my deaf mute unpaid companion signalled to me that she needed a container to catch my drool, I fossicked in the cupboard and found it. I tossed the buttons, all of them, over the balcony, a different balcony, the one in the home I was ejected from, but below that balcony too lay snow, deep and crisp and even, and the buttons fell there, and were soon buried under another fall, and just as the snow here is unlikely to melt, for we are so high and the air is so cold and thin, so too the snow at home will never ever melt. It is a frozen place, where I come from, and very far from the sea.