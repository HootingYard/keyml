\chapter{Potsdam Windbag}

SO UNIVERSALLY IS the author of \emph{Tales Told By An Idiot} known as the Potsdam Windbag that, in a new anthology of his work, his real name does not appear, even on the garish dust jacket. I have to confess that I do not know what that name is, and nor have I bothered to find out. I could have consulted an encyclopaedia, or a dictionary of nineteenth century Teutonic bloviators, but I am currently adopting an air of foppish lassitude, and I could barely bring myself to squelch across the sodden fields to the railway station to collect the copy of the book that had been left there for me, in a postal pouch, by a postal pouch person employed by the railway.

But bestir myself I did, during one of our recent thunderstorms, and thus for the past week I have been overjoyed to reacquaint myself with the prose of a true master. Admittedly, it is difficult to say of what precisely the Potsdam Windbag is a master. His stories are baffling, pointless, often idiotic, and grind on at pitiless length. One example included in the new book, \emph{The Tale Of The Something Or Other, I Have Not Yet Worked Out What It Is Though}, prates on for over four hundred pages before stopping abruptly in the middle of an ungrammatical sentence. Perhaps that is not fair. There is a sort of grammar at work, but it is one unique to the Potsdam Windbag. He wrote exclusively in fractured English rather than in his native tongue, despite never leaving his beloved Potsdam nor ever, so far as we know, communicating with any English speakers. One of the most arresting facts about him is that, when he died, his extensive library was found to contain not a single work in English except those he had written himself.

And my! did he write. This new anthology is so hefty that I had to hire a peasant and his cart to carry it back across the fields. The thousands of pages of dense, sometimes incomprehensible prose are a mere fraction of the Potsdam Windbag's outpourings, estimated to run to more millions of words than the combined works of hundreds of other windbags whose forgotten and unread books happily clog the bookshelves of our proper libraries.

Now, the surprising thing - surprising to me, at any rate - was that the peasant who hauled the tome home for me in his cart was steeped in the writings of the Potsdam Windbag. When he was growing up in a bare cabin in the forest, a paperback of the popular selection \emph{Tales Told By An Idiot} was the only book his parents owned. His father, a woodcutter, had been an autodidact who had taught himself to read through persistent study of the texts, and thereafter read the \emph{Tales} as bedtime stories to the peasant throughout, and indeed beyond, his childhood. This, I thought, might account for the strangulated vowel-sounds and guttural grunts which littered his speech-patterns. We were talking, the peasant and I, over a shared bowl of soup at my kitchen table, for having hired him for the carting and enjoyed his company I was loth to watch him vanish into the downpour. In fact, he has not left my hut since I invited him in a week ago, and has taken to sleeping on the floor of the pantry.

He told me of his favourite Potsdam Windbag story. Sadly, it is not one that has been collected in the new anthology, so I cannot reproduce it here. It is an early example of science fiction, in which the world becomes convinced that a character called 'Stephen Fry' is a super-intelligent being with an all-powerful brain. Characteristically, our author never makes a convincing case why so many should fall under the spell of this pandemic delusion. His tears dropping into our soup, the peasant wept as he recalled the terrific sadness of the story's end, where 'Fry' is revealed as merely an average man with a reasonably large vocabulary. I wondered if the Potsdam Windbag was trying to say something about himself in the tale he called \emph{The Cleverest Man In The Universe}. His command of the language he chose to write in, if eccentric, is highly impressive, and, as I said, his work-rate was prodigious. But it is one of the later stories, and it could be that, looking back on his life's work, the Potsdam Windbag was seized by the thought that it was all a waste.

The biographical details are sparse. Born and died in Potsdam during the nineteenth century, wrote acres of clotted prose, may have hobnobbed with Potsdam's movers and shakers from time to time. Other than that we know little. Yet what does it matter? We have the work, and - in the form of this bulging anthology, available to everyone with the physical strength to heave it home from bookshop or library - I hope a brand new readership, who will be well-rewarded in fighting their way through the coagulated morass of these teeming thousands of pages.

The peasant will awake soon. We shall share soup, and read to each other.