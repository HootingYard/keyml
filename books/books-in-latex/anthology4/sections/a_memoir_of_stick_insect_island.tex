\chapter{A Memoir Of Stick Insect Island}

I HAD SEVERAL reasons to sail across the Sound to Stick Insect Island. There were rumours of murder and mayhem and pagan sacrifice. My brother had made the crossing a fortnight before, and no word had come from him. My own homecoming was long overdue. And I wondered if the tiny post office still sold those amusing wax dolls of Captain Tod and Cadet Jarvis. The poking of them with pins was a delightful memory of my childhood, and I wanted my own nippers to share the experience, even though it would never be quite the same on the mainland.

I had been too long gone, I realised, as the skipper brought the boat into harbour. The stone walls were greatly weathered, the fishermen's huts were dilapidated, the ice cream kiosk was a burned-out shell. Gulls swooped and rampaged.

'You're quite sure you want me to leave you here?' asked the skipper, as I disembarked. I nodded, slipping some coins into his hairy hand. They were counterfeit, of course, but he would be dead before he could spend them.

I walked up the slope, past the notary's office and the chapel, and sat on the old familiar bench by the fountain. There was nobody about at this hour. In the square, the stone statue of Cadet Jarvis, much becrumbled, gazed sightlessly towards the woods, as it had done for a century or more. I hoped my business would not take me into the woods.

It was in the woods we found my brother. I was six years old, out with my father for a moonlit walk. Usually, when he went out at night to lay poisoned bait for wolves, he went alone, but on this occasion, apparently, I had been fractious and keening all day, and he thought the moonlight might becalm me. My brother was wrapped in a filthy blanket and wedged in the branches of a tree, a sycamore I think, at about my father's head height. He was about six months old, and fast asleep. My father placed him gently in the poison bag and carried him home. We never did discover who had abandoned him there. When his hair grew, it was lank and straight and tarry black.

The town, if you could call it a town, began to stir. The butcher came marching up the street, bearing his bloody meat cleaver proudly, like a soldier on parade. The beadle poked his head out of the bailey and sniffed the air. The lantern extinguisher rolled along in his wheelchair, extinguishing the lanterns one by one. Shutters were raised and bells clanged. When the duckman approached the fountain with his ducks in tow, it was time for me to move.

I could not help glancing back at the woods as I made my way, as inconspicuously as possible, towards the stationery shop. The flat above it was where my brother told me he was going to stay. When we parted, on the quayside, two weeks ago, I did not tell him of the knot in the pit of my stomach, wrenched so tight I thought I might die. I did not warn him about the flat over the stationery shop. I did not warn him at all.

The shop was not yet open. I pretended an interest in the window display, of typewriter ribbons set in a riot of stanhopeas, with their pale tiger flowers which exhale from afar a strong and acrid breath, as from the putrid mouths of convalescent invalids. I could smell it through the vents in the window. Eventually I heard bolts drawn back and a latch lifted. Before the door was opened, I gave it a shove, knocking the stationer to the floor. Looking down at his puffy piggy face, I felt both rage and nausea rising within me. But I left him there and, without a word, barged through to the back of the shop and took the staircase two steps at a time and flung open the door of the flat. I knew at a glance it was unoccupied. Popping a Sigsby pill to steady my nerves, I began a search that the most diligent bloodhound would envy. I was careful to leave the place looking untouched, but in a quarter of an hour I had examined every inch of the flat and found not a trace of my brother. The single anomaly was a black and white photograph of the balletomane Nan Kew propped on the mantelpiece. It was singed in one corner. My brother was deaf and blind to the ballet, as he was to all the performing arts.

I returned downstairs to find the stationer in the company of three ruffians. The shop door was locked and bolted again, and so thick was the pungent foliage in the window that the view of the street was completely obscured. I could not see out, and, of course, no passers-by could see in. Two Tilly lamps had been lit to light the interior.

'Well, well, well,' I said, 'The old gang's back together. Or have you never been apart? Still ironing each others' trousers?'

I sounded more confident than I felt. It was many a long year since I had been face to face with the Weltschmerz Boys. They had terrorised my brother and me as children, they had terrorised the island, and it seemed they did so still.

'You're going to come with us to the woods,' said one of the gits, his head even puffier and piggier than the stationer's. His ears looked as if they had been stuck on the wrong way round.

I was about to laugh - or shriek - but one of the others suddenly lunged forward and thumped me in the stomach.

'We're going to take you to the badger setts,' he said, thumping me a second time. Those were the words I had hoped never to hear again.

There is one dream, or nightmare, that has recurred throughout my nights. My brother and I, sometimes as children, sometimes as adults, are being driven, relentlessly, towards the badger setts in the woods, pursued by the Weltschmerz Boys, the four of them, or sometimes many, many more, more than ever existed in the waking world. They are armed with sticks or bludgeons, and they are roaring. We trip and stumble on twigs or tendrils, the wind howls through the swaying trees, an uncanny, terrible swaying, our clothing is torn to rags, and we are pressed ever closer to the badger setts. I always awake before we reach them. For that I am thankful. Once, sitting in a canteen on the mainland, fiddling with a croissant, my brother, usually so self-possessed, so brusque, confessed to me that he had the same dream, at least once a week. Often, it had incapacitated him for days afterwards.

The stationer unbolted his door and we stepped out into the milky morning light. The duckman was sitting on the bench, smoking, while his ducks plashed in the fountain. All the lanterns had been extinguished, and the lantern extinguisher had parked his wheelchair outside the tavern, waiting for it to open. From the butcher's shop came the sound of savage cleaving, and a steady stream of blood trickled out of the doorway into the gutter where it was lapped up by dogs. Cadet Jarvis' eyes of stone continued their eternal gaze towards the woods.

I was too winded from the thumps to make a run for it. And where would I run to? None of the islanders would help me, not even Mistress Pym in the post office. Nobody even looked at me as we made our way along the street, across the green, and on to the lane. There were dozens of people out and about now, some like sleepwalkers, some hailing each other or stopping to exchange a few words, words I knew, bitterly, would be incomprehensible to me now. I was a stranger here, become invisible. They all knew better than to meet my eye, to register my presence. I was too much changed for any of the older folk to recognise in me the boy I'd been, and the plastic surgeons on the mainland were too skilled to leave telltale scars.

I did not even recognise my brother, the first time I saw him after I'd fled the island. They had changed his height and posture and gait as well as his face. It was only when the unfamiliar figure, in the kit of a goatman, a doxy on his arm, leaning on a post in the shadows of the harbour, took from his pochette the filthy ragged blanket he'd been wrapped in as a foundling, and shyly waved it at me, that I knew him. I ran then, tumbling towards him, and kissed him.

Would he be there at the badger setts, changed once again, transformed, made into something awful and strange and grotesque, with paws or flippers, and a gigantic, twisted head, and no eyes, or one eye golden and the other of ruby, gibbering a mad litany? I had never been able to forget the names of all those who, year after year, on the feast day, were taken into the woods and never seen again. Oh, I tried to forget. We all tried to forget. But we never forgot a single name. Captain Tod, Cadet Jarvis, Nan Kew\ldots  the names we grew up with, whispered, muttered, barely breathed. When we were children, they sent delicious tingles of terror down our spines. We cherished our fearful shudders, gathered by the ice cream kiosk under the moonlight, each slosh of seawater against the harbour walls prompting another name, counting them off, one by one, year by year.

We were never told what happened to them, once they'd been taken deep into the woods, where the badger setts were. Now, my brother knew. And soon, very soon, as the Weltschmerz Boys steered me away from the lane into the woods, I would know too.