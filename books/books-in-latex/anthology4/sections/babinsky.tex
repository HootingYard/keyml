\chapter{Babinsky}

I AM GOING to tell you how I took my revenge on the monster Babinsky, but first I have to say a few words about the duckpond. Well, I don't have to, but I want to, to clear my head. It was the kind of duckpond that had clouds of gnats hovering over it, and from which the ducks had long since fled, supplanted by swans, particularly savage swans, so rightly it ought to have been called a swanpond rather than a duckpond, but these terms have a way of sticking. At least for me they do. I tend to use the same names for things as I did when I was still tiny, which was a very long time ago, so long ago that I had never even heard of Babinsky. Nor had the world heard of Babinsky then, for he was yet to commit his terrible crimes. Funny to think that I grew up in a world so innocent.

This duckpond was one of the first ponds I came to in my days of eggy Wanderlust. You know how it is, when you stuff yourself full of eggs, hard and soft, and feel compelled to go a-roaming o'er the hills and the meadows until you strike upon a duckpond or two. I no longer eat eggs, and I no longer go a-wandering as I did in those days. When I had a belly full of eggs I had vim and a compulsion. Rare was the day I did not stamp across fields grinding daffodils underfoot, on my way to a pond, in the teeth of storms. The ducks are gone, and the swans make a din, but the gnats still hover, and now my head is clear and I can tell my tale.

What is that Holland-Dozier-Holland song, the one about 'Empty silence surrounding me / Lonely walls they stare at me'? I would sing it to you if I could sing. Not the whole song, you understand, just those lines, to give you an idea of the circumstances in which I write. Solitude and silence and gloom - just the ticket. In the past, when Babinsky still roamed the earth, I had to write when and where I could, on the deck of a packet steamer or out in the wind and the rain on a pier or bundled in the back of a cab careering along broad urban boulevards. But now I can choose, and I choose a room of gloom. There is just me and my tortoise, Destiny's Child, and we are content.

I was at the duckpond when I heard Babinsky's name for the first time. Swans had already frightened away the ducks, and I was, in those days, very keen to learn as much as I could about the intricacies of swan behaviour patterns. I camped out in tentage at the edge of the pond for what I hoped would be a jolly fortnight. On the second day, reports reached me of a terrible enormity committed by Babinsky at a nearby farmyard. It was the kind of thing Truman Capote might have written about, but was certainly not a fit subject for a song by Brian and Edward Holland and Lamont Dozier. As for me, I had not become the word-drunk penman I am now, so it did not occur to me to write about it. No, I hid inside my rented tentage and blubbed like a baby. When I was done I hied over to the farmyard to see what horrors the monster had wrought. Then I vomited into a churn.

I was singing Hosannahs at a service in a consecrated cabin in the foothills of some very important mountains when next Babinsky struck. Earlier I told you I cannot sing, but Hosannahs are different. Try them and you will see. I felt it imperative to finish the Hosannah in spite of the havoc Babinsky had wrought, at an off licence in a village in a neighbouring foothill, and thankfully the rest of the choir agreed with me. How I treasure the memory of that mighty hymn of praise! It was like a slap in the face to Babinsky, or at least that is how it seemed. We did not learn until later that day that he had gone on immediately to down an airliner, using the same method as deployed by Colonel Stuart in \emph{Die Hard 2} (Renny Harlin, 1990).

You may be wondering what the coppers were doing all this time. It grieves me to say that they were utterly witless. Their photo-fit showed a lugubrious man with a pencil moustache and pimples and one eye alarmingly larger than the other. In other words, almost the opposite of Babinsky. The tape recording they claimed was of Babinsky making one of his criminal demands turned out to be of Mick Jagger engaged in idle chitchat. Worse, the psychic investigator attached to the case was only able to communicate with long-dead Aztecs. I began to realise that if Babinsky were ever to be brought to justice, it would fall to me.

In truth, I wasn't interested in justice. I wanted revenge. The farmyard, the off licence, the airliner, the countless other targets of his unquenchable criminality - it mattered not where or how he struck. I knew it was me he was after. And yesterday, at long last, I put an end to his reign of terror after, what, thirty, forty years?

I saw him at the duckpond. I went out that way partly for old time's sake and partly because I still take a vague interest in the way swans conduct themselves. I saw a stooped and shambling figure walking a dog around the pond. Don't ask me what kind of dog it was, I neither know nor care. How typical of Babinsky to be walking a dog around a duckpond, as if his soul was spotless! I rushed at him without warning and bashed him over the head with a spade, and he fell. I picked up the dog and wrung its neck and then I bashed Babinsky's head in with repeated blows of the spade. Then I threw him and his dog into the pond, where the particularly vicious swans made short work of them.

I walked slowly home to my room of gloom with my spade slung over my shoulder, and then I stuffed myself with eggs. I felt young again.