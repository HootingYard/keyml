\chapter{Magnet Boy! The Boy Magnet}

MAGNET BOY! THE Boy Magnet, the most magnetic cartoon character of the Atomic Age, was the brainchild of a washed-up has-been called Lamont Pinochet. As with his namesake, the brutal Chilean dictator, Pinochet's surname was mispronounced as 'Pinoshay' by all and sundry. 'I have a hard T, like Turandot!', the cartoonist used to shout, slumped in the gutter, as he often was before the late-flowering success he found with his magnetic hero. Even after that success, he spent much of his time in the gutter, for he had grown familiar with it, and felt comforted by the proximity of drains.

It is said that once, when he was in the gutter, Pinochet got embroiled in a terrific argument with a passer-by who insisted that General Augusto Pinochet himself pronounced his name 'Pinoshay', and that Giacomo Puccini specifically intended Turandot to be pronounced 'Turandoh'. The cartoonist was by this time so washed-up that he could barely summon the energy to respond, but it was the one thing he felt fierce about, so he inhaled the fumes from the nearest drain and gave his opponent a verbal battering. I have my doubts that this incident can be true, for at the time the Chilean would have been a youngster, and not yet a general, and unknown on the world stage. Be that as it may, there is something uplifting in the picture of the bedraggled cartoonist, dressed presumably in rags, gathering his wits in a Winslety gathering way, and demanding his final T be spoken aloud. Which of us can be sure we would have such gumption, even if it seems to be a trivial thing to get into a lather about?

Gumption, of course, is the quality we most readily associate with Magnet Boy! The Boy Magnet, together with perkiness and magnetism. In all his adventures, set in the fictional city of Magnetville, battling evildoers and outwitting communists, Magnet Boy! The Boy Magnet shows such incredible levels of gumption that, even as we cheer him on, we fret about his health. How is it, we ask, that such a little chap, albeit one whose physical form takes the shape of a snub-nosed perky head atop a large horeshoe magnet, can display such gumption week in, week out, without falling prey to the sort of debilitating weakness and neurasthenia that put Edgar Allan Poe in his grave at the age of forty? Ah, but then we remind ourselves that Magnet Boy! The Boy Magnet is fictional, and does not suffer the worldly buffets that beset Poe, and we are relieved. But worldly buffets certainly beset Magnet Boy! The Boy Magnet's creator Lamont Pinochet. Even before he found himself in the gutter, curled up by drains, he had been thrown from horses, trapped in the mountains, buried alive, stranded on a beach, pursued by bears, ravaged by toxins, imprisoned for tomfoolery, shoved in front of an express train, attacked by irredentists, smothered by pillows, punched by a pig farmer, locked in a cubicle, spat at by Mormons, burned by the sun, plagued by whitlows, bled by cupping, tarred and feathered, and pelted with pebbles. He had also repeatedly had his ears syringed by charlatans. He faced each and every one of these outrages with whining and self-pity, crawling ever further down a moral slope towards degradation and disgust.

And yet all this time he was scribbling his cartoons, on scraps of paper, on the backs of fag packets, on his own forehead. Somehow none of these earlier creations ever caught the public imagination, or even Pinochet's own imagination. There was a strip based on the more recondite essays of John Ruskin. There was a plethora of chapbooks featuring a talking celery stick called Drax, but because Drax came from another, celery-dominated planet, he spoke in space-gibberish. There was a character called Unconscious Squirrel!, a squirrel that was unconscious. Pinochet plugged away, trying out each and every idea that popped in to his head, no matter how stupid, and all the while the gutter beckoned. The gutter, the gutter\ldots the gutter that, miraculously, inspired Magnet Boy! The Boy Magnet. Oh, how I would love to reproduce just one, tiny picture of my cartoon hero. But I cannot, for to do so is forbidden. In his last will and testament, done in cartoon strip form, Lamont Pinochet declared that, with his death, all trace of his life on earth be wholly and utterly obliterated. 'I shall be expunged!' sings his alter ego, a singing ringing carpet beetle. And so it came to pass.

