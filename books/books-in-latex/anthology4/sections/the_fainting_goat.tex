\chapter{The Fainting Goat}

YOU WOULD DO well to remember, if ever you are out walking in the vicinity of the farmyard at Scroonhoonpooge, that you may come face to face with the fainting goat. If you encounter it on the lane leading out of the farmyard towards the orchard, and as soon as it sees you it topples over in a swoon, you must not be alarmed. You must certainly not think that the goat has fainted because you have caused it fright, by dint of something alarming in your appearance. Even if there is something terrifying about you, such as a twisted-up face or a too-brightly coloured clinker jacket or your being armed with a mail order Mannlicher-Carcano sniper's rifle, none of these things will be what causes the goat to faint. The goat will faint for the reason it is known as the fainting goat, which is that it is constantly fainting, dozens of times a day, even dozens of times an hour.

This constant swooning is a mystery as far as the local vets are concerned. There are several vets with practices in walking or short bus journey distance of Scroonhoonpooge farmyard, and all of them at one time or another have been called to tend to the fainting goat. They have tried all sorts of treatments, from goat-friendly smelling salts to the deployment of Peruvian whistling vessels to simply shouting very loudly into the goat's ear, and though such techniques may revive the goat from its faint, none have served to stop it clattering over in a dead swoon again and again as the long countryside day draws on towards dusk and rainfall. When it is conscious, the goat seems hale and hearty, even frisky, and engages in all the normal activities one might expect of a farmyard goat. I would list these activities but I am sure you are thoroughly up to speed with the doings of goats, given the demographic of the Hooting Yard readership.

There has been a certain amount of bickering among the local vets, as each of them grows frustrated at their inability to stop the continual fainting of the fainting goat. When they passed out of their veterinary colleges, they were all brimming with confidence, armed, as they thought, with the knowledge and expertise to handle all sorts of bestial maladies, from the workaday to the exotic. Whether it be a cow with a pox or an ostrich beset by Von Straubenzee's Gruesomeness, these vets believed they could march into a farmyard or menagerie and win the undying gratitude of farmers and menagerists by weaving their vetty spells. An injection here, a siphoning off of fluid there, and to the gasps of their keepers the cow or ostrich or whatever beast it may be would leap up, restored to vigour, and there would be a round of applause and the discreet passing of banknotes into the pocket of the smug vet.

But the fainting goat goes on fainting, day in day out, and not one of the vets has a clue what to do about it. When they gather of an evening on the balcony of the Caf\'{e} Simon Schama, at first they boast of their breakthroughs, the splint affixed to the leg of the sparrow, the gunk drained from the badger's boils, the palsied pig unpalsied. But as they sip their fermented slops, tempers fray, and the talk soon turns to the fainting goat, that damned intractable fainting goat, and harsh words are said and there is spitting and chucking and fisticuffs, black eyes and bruises and the odd dagger slash. And so it goes on, night after night.

Now curiously enough, during the night the fainting goat never faints. It remains wide awake all night every night, either in its comfy pen or out in some field, doing goaty things, things other goats do in daylight. Apprised of this singular information, some have posited that the goat's swoons are not swoons so much as its repeatedly falling asleep from exhaustion. It is indeed a cogent case, but it is nevertheless mistaken, for reasons crystal clear to those, such as some among the vets, who have made studies of the goat's neurological peculiarities. It sleeps not, yet it faints. The one is understood, and explicable, the other not. There are more curious cases among the goat population, as among other farmyard beasts, but not many, sure enough. That is why the vets fret so.

But you will not fret, will you, as you wander past Scroonhoonpooge farmyard, on your way to the orchard, to pluck persimmons from the trees, illegally, and you come upon the fainting goat upon the path and it faints at your feet? You will pat its little horns and lift it to its feet, and send it tottering off along the lane to its next collapse, for you are wiser than the vets, you are wiser than the farmer. The only thing wiser than you is the fainting goat itself. No goat was ever wiser, nor had so explosive a brain