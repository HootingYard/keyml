\chapter{Instances Of Inanity In Blodgett}

LET US CONSIDER three particular Blodgettian inanities. There are, of course, many, many more, so many they are numberless. But it is worth looking in detail at these three, if only to get the measure of the man.

\emph{His tin shadow.} The tale is told that Blodgett awoke one day in a state of terror. Whether he had had a night of awful dreams brought on by a bedtime snack of processed goat's cheese triangles and gooseberry paste, or whether he was just in a flap, we do not know. What we do know is that when he flung open his curtains to greet the day, Blodgett found the sky to be hazy and overcast, and the sunlight so weak that it cast no shadows. In his tumultuous mental state, Blodgett took this as evidence that he was becoming, or indeed had already become, insubstantial.

A sensible person would have tested this misperception by, for example, the Dr Johnson trick of kicking a stone, but there were no stones on the floor of Blodgett's hotel room, not even a pebble. Doubtless there are other experiments Blodgett could have tried, such as bashing his body against the walls, or plunging off the balcony. But the mania seems to have had him in its grip. Looking at himself in the mirror was no help, as Blodgett always had a grey and ghostly pallor. It was one of his defining features. As a tot, he was always cast as a ghastly wraith in the school play, even when such a character was not actually required. Peering at himself now, in the milky light of his Tyrolean hotel room, Blodgett fancied that he was becoming transparent.

Hastily dressing in what fashionistas would deride as 'tatterdemalion casual', Blodgett crashed out of the hotel into the abnormally bustling streets. All these Tyrolean folk going to and fro, bent on their mysterious Tyrolean business, seemed solid enough. Blodgett, on the other hand, felt himself wafting, as if he were but a wisp that would be blown away by the first gust. The haze was oppressive however, and there was no hint of wind. Blodgett found a cafeteria attached to a secondhand snowplough dealership where he took breakfast. As he dunked iced dough fingers into a thin broth, he kept checking to see if his shadow had appeared, but there was no change in the light. It does not seem to have occurred to him that nothing else was casting a shadow in that town, on that morning. He was, as usual, a monster of egocentricity.

The reports tell us that after breakfast, Blodgett visited the town's one and only metallurgical institute, where he badgered the janitor to let him in. It appears that he then armed himself with some hammers and cutting blades, found a supply of tin, hammered a quantity of tin into a flat sheet, and cut an outline of his body with the blades. He was seen carrying his tin effigy through the streets, heading towards a Tyrolean glue and adhesive supplier. The next witness statements indicate that Blodgett had glued the feet of his tin self to his heels, so that as he strode through the streets and lanes and expansive boulevards of the town, he dragged the tin Blodgett behind him, like a shadow. It is said that he was much becalmed, and no longer jangling with terror.

The Swiss dramaturge Rolf Turge wrote a squib based on Blodgett and his tin shadow, in which the lead character goes berserk when the haze disperses and sunlight batters down upon the town, casting shadows so strong they are as black as pitch. In real life, Blodgett was oblivious to the sun, and he dragged his tin shadow with him for months and months, until the glue dissolved when he stepped into a chemical puddle outside a post office in Pepinstow.

\emph{His dockside groans.} Can one reasonably include Blodgett's dockside groans in a list of his inanities? After all, which of us has not groaned when trudging around the docks? There is surely something about all that clanking and shouting, the winches and bales, the crates and chains, the chugging and hooting, the stink of oil and fish and brine, that elicits a groan from the sunniest of dispositions, and not just a single groan but a whole series of them. Why, then, charge Blodgett with inanity, when his dockside groans were of a piece with yours or mine? Do we succumb to inanity too? Well, no, of course we don't. We are level-headed, sensible persons. And Blodgett, of course, was not. He lived in a fool's paradise. So when we consider him plonking himself down on an iron bench at Pepinstow docks, and groaning, we think to ourselves, 'there is a man flailing helplessly in the extremes of inanity'. He may no longer have a tin shadow glued to his heels, for the glue dissolved just a couple of hours ago in a chemical puddle outside the post office, but he is by no means freed from his embonkersment. Look, a gull has perched on the bench next to him. Now, soberly, taking your time, judge them both, the man and the bird, and choose which one you would trust to best perform a simple task such as savagely ripping and rending a sturdy cardboard box to shreds. Your answer will not, I think, be the man with the ghostly pallor who sits there groaning, groaning at the dockside.

\emph{His futile picking at unbuttons.} Blodgett devoted much of his time, one autumn, to a study of the unbutton. At first, he went off on completely the wrong track. Adducing that the unbutton was 'that which is not a button', Blodgett mistakenly concerned himself with 'that which is, where the button is not', in other words, the buttonhole, the emptiness, the void the button will, one day, occupy, or, perhaps, once did occupy, before its thread snapped and it fell into a puddle, perhaps even the chemical puddle outside the post office in Pepinstow, where it lay alongside Blodgett's unglued tin shadow. But of course a buttonhole is but a buttonhole, not an unbutton. Autumn was a month old before Blodgett realised his error. He had been shuttered in his Tyrolean hotel room picking futilely at buttonholes, only occasionally stepping out to wolf down breakfast and afternoon tea and dinner at a cafeteria. Then, one morning, he had an epiphany. A monologue devised years later by the Swiss dramaturge Rolf Turge gives us a flavour, albeit imagined, of the Blodgettian brainpan pirouettes of that day.

\emph{I was picking futilely at a buttonhole when a crow landed on my Tyrolean hotel room windowsill. I cast aside the buttonhole and looked at the crow, and the crow looked at me. I thought, if I were to make a puppet of the crow, out of black rags and tatters, I would use buttons for its eyes, would I not? And then I thought, perhaps the crow is thinking of making a puppet Blodgett, out of torn-up shrouds and winding-sheets. Would it, too, make my eyes out of buttons? Or, being a crow, primed by the bird-god that made it to peck out my eyes, would it need, for its puppet, not buttons, but unbuttons? That is when I realised that the unbutton is something greater, stranger, far more uncanny than a mere buttonhole. The crow flew away, bent on Tyrolean worms no doubt. But I had seen the error of my ways, and I stamped my foot repeatedly upon the buttonhole I had been picking at with such futility, and I crashed out of my hotel room into the street, the abnormally bustling street, and my eyes glowed brightly, real eyes, not shiny buttons on a puppet, and I strode with my head held proud and high, seeking afresh the true unbutton I knew, now, was there, somewhere, hidden in plain sight.}

By the time autumn turned to winter, Blodgett had found an unbutton, or at least what he took to be one. Certainly it met the definition of 'that which is not a button', and Blodgett pounced upon it, there in that Tyrolean town. Yet, having found it, what did he do? It is a measure of the man's inanity that he simply picked at it futilely, for days on end, sitting on an iron bench at the dockside, groaning, shadowless, having fled the Tyrol for Pepinstow, in the autumn of 1963, just before the Kennedy assassination, and the Beatles' first LP.