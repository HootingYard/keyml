\chapter{Bungled Heists}

AT THE LAST count, Blodgett is thought to have been involved in no fewer than six bungled heists. By comparing the circumstances of each heist, we may learn not only about their bunglement, but something, too, about Blodgett the man.

First heist. The plan was to steal a consignment of birdseed being delivered to a crow sanctuary. Prices in the millet market had rocketed, and a tidy sum could be expected when the ``hot'' birdseed was offloaded to a fence. The gang spent weeks hidden behind a hedge observing the routine. At exactly 11 o' clock each morning, a truck arrived at a gate in the perimeter fence and, after a cursory check of paperwork, it was waved through and driven at snail's pace to the silo, whereupon a sanctuary worker hauled the vacuum-packed bags of millet off the truck and put them on a hoist which was winched up to the top of the silo. There, on a platform, a second worker slit each bag open with a birdseed-bag-cutter and dumped the contents into the silo. The empty bags were chucked back to the ground and replaced on the back of the truck, which then drove off, through the gate. The entire operation took about fifteen minutes. Blodgett's role was to thump the truck driver and the gatekeeper, disabling them for sufficient time to allow the gang to steal the birdseed before the truck entered the crow sanctuary. At this time, Blodgett carried quite a thump, and he practised it on life-size cardboard cut-out persons, which toppled over at the first thump. This was the key to the embunglement of the heist. Both the truck driver and the gatekeeper were great thick-set brutes, much less flimsy than Blodgett's practice figures. When thumped, neither of them toppled over. Instead, they thumped back, the two of them, with alarming violence, until Blodgett was sprawled on the ground battered and bloodied and unconscious, at which point they summoned Detective Captain Cargpan by walkie-talkie.

Second heist. Blodgett joined a different gang for his next heist. This was a smaller-scale affair, the aim being to pinch a packet of arrowroot biscuits from a half-blind doddery octogenarian crone as she creaked along a secluded lane. Technically, it can be argued that such a venture falls outwith the strict definition of a heist, but quite frankly I am not prepared to countenance such a cavil, as it would threaten the basic integrity of my narrative thrust. The idea was that the gang would hide behind a clump of aspens, and, at the approach of the crone, Blodgett would leap out into her path and thump her. Taking advantage of her surprise, alarm, and possibly fatal injury, another member of the gang would snatch the packet of arrowroot biscuits from her pippy bag, and the gang would make off with all due speed, cackling. In this case, the bunglement consisted of failure to realise that the crone in question was Mrs Gubbins, herself a criminal mastermind, and one who could deploy her knitting needles to lethal effect. When set upon by Blodgett, she poked him in the solar plexus with a sharpened 4.25, jabbed his head with it as he crumpled to the ground, and then coolly tucked it back in her bag before calling Detective Captain Cargpan on her klaxon.

Third heist. Blodgett had rejoined his original gang, but made it clear he wished to have no part in any thumping on the next job. He was thus engaged as a look-out man. Blodgett did not pay attention, however, to a particularly riveting Dan Corbett weather forecast, and was ill-prepared when a dense and freezing and engulfing mist descended upon him as he sat in his perch overlooking the big cash-register warehouse. He was peering hopelessly into the murk when he felt the begloved hand of Detective Captain Cargpan nabbing him on the shoulder.

Fourth heist. This heist was, at least in its conception, the most ingenious. Inspired by the classic art-house film \emph{Snakes On A Plane}, had it been fictionalised for the cinema it could have been called \emph{Otters In A Laundry Basket}. Unfortunately, the otters escaped from the laundry basket and ran away to a riverside before they could be deployed. This was Blodgett's fault, as he had been enrolled into the gang specifically to train and control the otters. He was in bad odour after this, and considered becoming an informer for Detective Captain Cargpan, but instead holed himself up in a chalet in Jaywick for some years, lying low.

Fifth heist. Tempted out of his Jaywick hidey-hole by the prospect of a share in the proceeds from a daring smash 'n' grabby-type heist, Blodgett returned to the criminal fray as part of yet another gang. A plate glass window was to be smashed, and a display of ornate cornflake packets dripping with jewels was to be snatched. The packets were the work of a bumptious and bespectacled artist of great, if unfathomable, repute. Everything went according to plan, except that the gang left Blodgett to guard the art in a lock-up under the arches of Sawdust Bridge while they tracked down their expert fence, who was hobnobbing with hedge fund managers. Peckish Blodgett opened up the packets and ate all the cornflakes, dry, without milk, thus destroying their value as art. Left with nothing but a bunch of jewels, albeit valuable ones, the gang fell foul of a pasteman in the trade, who tricked them as a pasteman will, and turned them over to Detective Captain Cargpan, who was waiting outside with his ruffians.

Sixth heist. One can gain some idea of the duration of Blodgett's criminal career when one considers that the sixth heist took place more than fifty years after the first. By now, Blodgett was old and wheezy, and as creaky as Mrs Gubbins had been (see second heist). It was his creakiness which led to the bungling of his last heist to date. The vaults of the big important bank into which the gang broke their way with the aid of industrial slicing and cutting and burrowing equipment were, of course, heavily alarmed. Multiple sensors would pick up the tiniest sound or movement. One by one, each sensor was disabled by the gang's sensor disablement man, using his pliers or pincers or, in one case, a soaking wet dishcloth. Things were set fair for a successful heist. But Blodgett creaked as he crept towards the cash-cage, alerting a tiny rodent, which scurried in fear towards the big important bank's basement wainscotting, and in so doing dislodged some wiring, causing a short circuit which knocked out all the electrics. Plunged into Stygian blackness, Blodgett and the gang were helpless, and could do nothing but await the arrival of the janitor in the morning. This janitor was an old mucker of Detective Captain Cargpan, who was himself on the scene within seconds, blackjack and manacles at the ready.

According to a story in a recent issue of the \emph{Weekly Heist Intelligencer}, Blodgett is a member of a gang plotting a forthcoming heist at an amusement arcade in a seaside resort. Letters have since appeared in the correspondence columns pleading with the gang to drop Blodgett from their plans. The inherent sentimentality of the criminal demimonde suggests this is unlikely to happen. It is thought Detective Captain Cargpan has already splashed out on a railway ticket to the seaside resort.