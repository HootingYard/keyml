\chapter{Blodgett Boils My Lady Kent's Pudding}

IN LOOKING THROUGH \emph{Thaumaturgia, or Elucidations Of The Marvellous} by An Oxonian (1835) the other day, for a quotation about a delusional glass man, I came across this:

\begin{quotation}
\emph{We shall conclude our astrological strictures with the following advertisement, which affords as fine a satirical specimen of quackery as is to be met with. It is extracted from 'poor Robin's' almanack for 1773; and may not be without its use, to many at the present day. We will vouch for it being harmless, but as we are not in the secret of all that it contains, our readers must endeavour to get the information that may be wanted, on certain important points, from other quarters}
\\
\\
\emph{ADVERTISEMENT.}
\\
\\
\emph{'The best time to cut hair. How moles and dreams are to be interpreted. When most proper season to bleed. Under what aspect of the moon best to draw teeth, and cut corns. Pairing of nails, on what day unlucky. What the kindest sign to graft or inoculate in; to open bee-hives, and kill swine. How many hours boiling my Lady Kent's pudding requires. With other notable questions, fully and faithfully resolved, by me Sylvester Patridge, student in physic and astrology, near the Gun in Moorfields.'}
\\
\\
\emph{'Of whom likewise may be had, at reasonable rates, trusses, antidotes, elixirs, love-powders. Washes for freckles, plumpers, glass-eyes, false calves and noses, ivory-jaws, and a new receipt to turn red hair into black.'}
\end{quotation}

One man who knew a thing or two about boiling My Lady Kent's pudding, apart from Sylvester Patridge, was Blodgett. Blodgett first came upon the recipe when he was under the culinary tutelage of the so-called Culinary King of Cuxhaven, Binsey Poplars. Poplars himself unearthed the pudding details during his researches in an archive of pudding recipes at the Texas Recipe Book Depository in Dallas, bang next door to the more famous - or infamous - Texas Schoolbook Depository, from a sixth floor window of which, on a November day in 1963, Lee Harvey Oswald shot President John Fitzgerald Kennedy with a mail-order Mannlicher-Carcano rifle. Some would add 'allegedly' after that statement, but not me. I have read my Posner.

Nobody, so far as we know, ever shot anyone from a window of the Recipe Book Depository, not even Binsey Poplars, who, when he was not rummaging through old recipe books, could himself be impelled to acts of senseless violence. He once broke Blodgett's legs, for example, quite deliberately, with blows from a tent peg mallet. Poplars called this mallet his Hammer of Pedagogy, which was something of a misnomer, as he also used it to crack eggs, to bash out dents in his pans, and to hammer tent pegs into campsite mud. He was fond of taking his students on camping trips to the outskirts of Cuxhaven, and having them forage or starve.

It was on one such escapade, when Blodgett was still on crutches, that teacher and student fell into a lengthy conversation about puddings. The Culinary King had only recently returned from his Texas trip, and his head was full of the recipes he had discovered in the pudding archive. The countryside around Cuxhaven was at the mercy of roaring winds that weekend, and Poplars and his students were huddled in their tents. It was not foraging weather. The pedagogue made a point of sharing his tent with any student whose bones he had broken in a fit of temper, and so it was Blodgett on this occasion who sprawled at his master's feet. As far as puddings went, Blodgett knew almost as much as Binsey Poplars. He had immersed himself in the world of puddings since infancy, and it was this enthusiasm that had led him to sign up to the Culinary King's Crash Course in the first place. For though Blodgett could tell you about thousands of different puddings, he had no idea how to make a single one of them.

In the tent, as gales howled and canvas flapped, Poplars and Blodgett talked about puddings for hours.

'Of course,' said Binsey Poplars pompously, 'Sylvester Patridge claimed to know the correct boiling time for My Lady Kent's pudding, but the man was a charlatan and a fool, and if you boiled it for the time he recommended you would end up with a pretty sorry excuse for a pudding'.

'Tell me more,' said Blodgett, all ears, because here was a pudding that, remarkably, he had never heard of. And as his tutor prattled on, Blodgett scraped shorthand notes on to one of his crutches with a sharpened twig.

Years later, far from Cuxhaven, restored in limb, and now a dab hand at cooking the puddings he had once merely salivated over, Blodgett stumbled upon his old crutch and deciphered the scrapings he had made upon it. He transcribed them into a notebook, embellished them, and published them as part of \emph{Blodgett's Book Of Many Puddings}, a copy of which, fittingly, was acquired by the trustees of the pudding archive at the Texas Recipe Book Depository on Elm Street in Dallas, just along from the overpass on the Stemmons Freeway.

\emph{A fantastic challenge for any maker of boiled puddings,} he wrote, \emph{is the pudding named after My Lady Kent. Should it be steamed before boiling, or afterwards? Should it indeed be steamed at all, or should one just get on and boil it? What is the best type of pan in which to chuck the pudding ingredients prior to boiling? Does the pan matter? If the pan is dented, should one bash out the dents with the Hammer of Pedagogy beforehand? If one neglects to do so, will any indentations in the finished pudding caused by the dents add to its savour, or will they detract from it? Is there a place, in the contemporary world, for dented puddings, or should we be aiming for clean lines and smooth edges? Can a modern version of My Lady Kent's pudding compete with the original? Should we allow indentations irrespective of their effect simply because, in all likelihood, given the rough and tumble of the times, My Lady Kent's own pans would have been outrageously dented? Rare was the pan in those days that did not get bashed about and suffer because of that bashing. That may be one reason for the popularity of puddings, for there are cogent arguments claiming that the final shape of a pudding, particularly a boiled pudding, matters not a jot to the eater of the pudding. Are there any cases we can advert to where a pudding has been sent back from table with the complaint 'I cannot eat this pudding. It is dented.'? Such reservations are likely with other things one might eat. A duck of the wrong shape, likewise a pig's head or a pie full of misshapen blackbirds, will cause revulsion, for the eater may think, rightly, that they are being fobbed off with abominations of nature. But there is no such thing as the correct shape of a pudding, not even of My Lady Kent's pudding. And yet to make one that is succulent and lip-smacking remains a challenge, and takes years of study, sometimes in a tent, on the outskirts of Cuxhaven, while canvas is buffeted and fierce winds blow.}

It does not escape the reader's notice that Blodgett fails to answer many of the questions he, or Binsey Poplars before him, raises, and nor does he provide a workable recipe for the pudding he so enthuses about. That is Blodgett all over, of course, infuriating and exasperating yet strangely adorable for all that.

Incidentally, it is said that the dressmaker Abraham Zapruder, who filmed the famous footage of the Kennedy assassination on his top-of-the-range Model 414 PD 8 mm Bell \& Howell Zoomatic Director Series movie camera, was planning to spend the afternoon, following the passing of the presidential motorcade, in the Texas Recipe Book Depository, specifically to consult Blodgett's book. Whether he was intending to boil My Lady Kent's pudding, and was looking for helpful hints, we do not know, and now we never will, for history took a fateful turn on that sunny day in Dallas, and the dressmaker's boiled pudding thoughts were wiped clean from his brain. But not from yours, or mine.