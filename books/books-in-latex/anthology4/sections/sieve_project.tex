\chapter{Sieve Project}

GIVE ME A sieve or a riddle and, boy oh boy, before you can say 'lumme, guvnor, knock me down with a feather', I'll have a pan of fine powder from which all the cloddy clumps have been winnowed. By pouring a little warm water in to the pan from a bowl, and stirring it with a stick, I will soon have a paste. And you know what? I can add dye to the paste to make it any colour I like. It doesn't matter what colour the powder was originally.

The next thing I can do is to make the paste a little tacky by adding a binding agent. In 1859, William H Gregory published \emph{Egypt In 1855-56}, where he remarks, inter alia, that 'The rocky walls were black and sticky, and seemed to sweat a thick, fatty, viscous liquor'. That is precisely the kind of paste we want in our pan.

Now we return to the sieve, or riddle, which holds all the clumps that didn't make it through. The first thing to do is to remove them, one by one, and place them in a line on your countertop. Sort them in order of size, so that you have a row of clumps gradually ascending from tiny to titanic. Of course, the tiniest of the clumps will not be too tiny, for remember that what you have here are the leavings in the sieve. What I always do at this stage is to count the clumps, just to give me some idea of what lies ahead, in terms of time and effort. Then I get a duster from the duster drawer and very carefully swab all the clumps, one by one, wiping off any grime or filth from them.

When this part of the process is complete, shake the duster violently over a bucket or pail, to collect all the bits of unutterable filth. However energetically you shake the duster, some minuscule crumbs of dirt may cling to it. These can be removed by aiming a jet of fast-rushing air from a nozzle over the whole surface of the duster, pointing the jet in such a way that the last remnants of grubbiness fall plop into the bucket or pail.

If you have got this far with the project, give yourself a pat on the back, or, better, get someone else to pat you on the back, or to give you a bear hug. Try, though, not to be tempted to take a breather for a cup of tea and one of your cheap Bosnian cigarettes. I make every effort to press right on, fearing a lack of momentum. Maybe that's just me. You might be able to put your feet up and even take a nap, but I wouldn't risk it.

Next we wash our hands, before pouring the viscous paste out of the pan into a clean bowl. We lay paper towels out on the countertop. We grasp a pair of tongs. There are three basic classes of tongs, to wit: (a) tongs which have long arms terminating in small flat circular ends and are pivoted close to the handle, as in the common fire-tongs, used for picking up pieces of coal and placing them on a fire. (b) tongs consisting of a single band of metal bent round one or two bands joined at the head by a spring, as in sugar-tongs (a pair of usually silver tongs with claw-shaped or spoon-shaped ends for serving lump sugar), asparagus-tongs and the like, and (c) tongs in which the pivot or joint is placed close to the gripping ends, such as blacksmith's tongs or crucible-tongs. Which class of tongs you use is entirely up to you. I did not tell you what colour to dye your paste, nor am I going to limit your choice of tongs. There are ancient freedoms we must strain with all our might to protect.

Starting at the end of the row of clumps where you placed the tiniest clump, pick up the first clump with the tongs and dip it into the paste, until it is coated, and then place it carefully on the paper towels. Relax the tongs, and proceed to the next clump. Continue without pause until you have dipped and coated all the clumps with paste. Do not allow a squadron of Messerschmitts screaming across the sky to distract you.

While waiting for the clumps to dry, I take the opportunity to carry the bucket or pail full of filth to a municipal filth depot, where I upturn the bucket or pail and empty it. Back home, wipe the insides of the bucket or pail with a rag. Do not use one of your dusters. That is not what dusters are for.

You now have a set of pasted clumps, each of which has two little unpasted patches where it was gripped by the tongs. There is likely to be a third anomalous patch after you have picked up the clump to place it in the bucket or pail. Given the viscosity of the paste, a shred of paper towel may remain stuck to the clump. Treasure this imperfection.

When all the clumps are gathered in the bucket or pail, I always feel like singing a round of glees, but you don't necessarily have to. What you must do is to take the bucket or pail and place it in your porch. Nothing quite pulls a porch together as winningly as a bucket or pail of clumps covered in paste, each clump with its two or three little unpasted patches. And if you don't believe me, consider this entry from the diaries of Lady Chlorine Skippington-Pip, denizen of the Caf\'{e} Showoff:

\begin{quotation}
28th March 19--. Dennis [Prong] came to visit, accompanied by his wolfhounds and a crack troop of snipers fresh from dispersing unseemly rioters. I had hoped to treat them to tea and biscuits and lobster, but they spent the whole time shuffling around in the porch, absolutely transfixed by my bucket or pail of clumps covered in paste, each clump with its two or three little unpasted patches. It proved quite a hit. It was dark by the time they scuttled off. Despite the cheerfulness of the porch time, I felt suddenly overcome with desolation and anguish, and slumped on one of my carpets, sobbing, sobbing, long into the night, until distracted by a squadron of Messerschmitts screaming across the sky.
\end{quotation}