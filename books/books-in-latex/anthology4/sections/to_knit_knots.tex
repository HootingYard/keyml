\chapter{To Knit Knots, Peradventure}

MUCH HAS BEEN written, in the past, by people who knew of these things, about the knitting of knots. Knots, we learn, have been knit from cord and twine and rope and string and wool, among other materials. While it is true that more knots have been tied rather than not, without the aid of knitting needles, it remains the case that the knitted knot has its own special place in our hearts, whether our hearts flutter like a bird's or a squirrel's heart, or pound like a drum. For with the knitted knot we see a true craft, whereas it can be argued that the mere tying of knots, while sometimes requiring deftness and digital agility, can as well be done by a brute in a hurry. Not so the knitted knot.

Hurrying brutes, particularly those whose tails thump upon the ground as they rush headlong to the scene of their next enormity, are most unlikely to have the patience and wit necessary for the knitting of a knot. Nor are their paws likely to be dexterous enough to handle knitting needles, or even crochet hooks. Crochet is not knitting, of course, and the crocheted knot is a different creature to the knitted knot, and one with its own literature, exemplars, and paragons.

There exist pattern books containing instructions for knitting knots, and depictions of knots so knitted, but it would be a mistake to think that one needs such a pattern before embarking upon the knitting of a knot. Some of the finest knitted knots have been the work of improvisers, brave, adventurous souls who begin to knit with no other aim in mind than the knitting of a knot, its final form unimagined, not even a blurred wisp in the mind's eye of the knitter.

Even improv knot knitters, however, need a degree of foresight, for they will wish to avoid the act of knitting being interrupted by a hurrying brute. Such interruptions can prove fatal, if not to the knitter then almost certainly to the knot. A brute in a hurry, coming upon a knitter, will tear and shred and rip and rend, all the while roaring its brute cries as its tail thumps the ground. Thus the knitter of knots is advised, in many of these books of the past, to find a secluded haven in which to knit. To be hidden behind a clump of brambles, or snug in a concealed nook in a cave, or safe behind the ramparts of a mighty and towering fortress, each of these has been recommended. A knitter's choice of refuge will depend to some extent on the nature of the brutes who hurry through the lands in which they knit. There are single brutes who roam alone, and pairs, and occasionally trios, but by far the most common, and the most frightening, are those who hurry about in packs.

Various writers have pointed out that the knitter of knots can use the knots they have knitted as part of the apparatus to bind and immobilise a hurrying brute. This is undoubtedly true, but these same writers tend to neglect the inconvenient fact that, before such binding and immobilising and judicious use of knitted knots can occur, the hurrying brute must first be overpowered. In most cases, at least those cases that bear examination, the overpowering of a brute in a hurry requires inhuman strength, and the kind of musculature rarely found in the average knot knitter. Even more important, then, to ensure that before the very first clack of needle against needle, the knitter has located a place of safety in which to knit.

Perhaps the finest of the books I chanced upon when researching this article is actually more a pamphlet than a book proper. It is \emph{How To Knit Knots While Remaining Invisible To Hurrying Brutes} by Dobson (out of print), and contains a plethora of terrific mezzotints by the mezzotintist Rex Tint. Dobson claims to have invented a so-called 'enshrouding spectral ether-cloak' which, when activated, renders the knot knitter invisible, thus obviating the need for a time-consuming search for clumps of brambles, nooks in caves, or mighty and towering fortresses. It also silences the clack of knitting needles, or at least drowns out the clack, by generating a noise like the buzzing of a million hornets, audible only to a brute hurrying past, its tail thumping the ground. I suspect that Dobson's 'cloak' is wholly spurious, but the pamphlet is worth it for the mezzotints alone.