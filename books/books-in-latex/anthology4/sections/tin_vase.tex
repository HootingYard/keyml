\chapter{Tin Vase}

THERE'S A TIN vase on my mantelpiece where I keep my buttons. But where am I to put my unbuttons? A bright six-year-old would put their hand up and cry 'In your unvase!' But there is no such thing as an unvase, neither of tin nor of untin. Or, if there is such a thing as an unvase, that which a vase is not, it would not be possible to put anything into it, because it is the ability to hold things - buttons, unbuttons, flowers, coinage, treasury tags - that, in part at least, defines a vase.

I am particularly keen on tin vases, because they are cheap and light and batterable. Bash one with your fist while cursing the universe and it will not smash into smithereens, it will merely receive a dent or two. As we say in my little groupuscule, such dents can lend a tin vase character. Heaven knows what would happen if you bashed a tin unvase, or even an untin unvase. That is assuming you could do so in the first place, which is by no means guaranteed. I suppose if you staged it so that you, and the unvase, were reflected in a judiciously-placed mirror at the time of bashing, that might work. I must share that thought with the groupuscule when next we meet. Usually we gather in a hut, but we are keen to push envelopes, so we plan to hold our next meeting in an unhut. As yet, we have failed to locate one. That bright six-year-old might say, 'Well, a shed is an unhut, being a shed instead of a hut', but that is not strictly true. A shed is sufficiently similar to a hut to be mistaken for one, by most people, on most days, in most circumstances. What a palaver.

Another advantage of the tin vase is that, when struck with, say, a pebble, it makes a tinny clang. There are untinny clangs, and we can easily imagine an unclang, tinny or otherwise, for an unclang would be any noise that is not  a clang. Or, actually, it might be silence, dead silence, as one will find in the grave, when one is consigned there, eventually, six feet under, pushing up the daisies. I have things so arranged that when my time comes I am hoping to push up undaisies. I have circulated instructions, to the members of the groupuscule, in case they survive me. They might. Some of them are young and hale. I condescend to them, it is true, but they take it in good part. Perhaps they can see the shadow looming over my shoulder, the grisly worm-eaten shadow that is a sort of unguardian angel, or guardian unangel. Is an unangel a devil, or is it something more horrifying? I have wondered, from time to time, in the bath or upon a balcony, if an unangel is the kind of being so unutterably gruesome that, when one tries to speak of it, one's tongue cleaves to the roof of one's mouth, and one can only make incoherent muffled noises, like a small animal trapped in the sights of a larger one, and about to be torn to pieces, with great savagery, in bright battering sunlight.

One of the reasons I keep my buttons in a tin vase upon the mantelpiece is to give my brain a distraction from these dark and debilitating thoughts. As soon as I sense my mind rolling along the cold iron rails towards bleakness and death and the triumph of an unangel over my soul, stamping it underfoot, I hie to my mantelpiece, and take my tin vase, and I spill the buttons out of it onto a platter, and I count them, or I polish them, rubbing them with a rag steeped in bleach or swarfega, or I examine them closely, through an optical aid, one by one, holding each button between my forefinger and thumb, in my left hand, squinting, peering, until I am no longer conscious of the sounds in the garden, the awful sounds of large beasts slaughtering small ones, and the sounds of the gravedigger, in his filthy overalls, forcing his spade into the muck, again and again, and tossing each spadeful onto a heap, so slowly, relentlessly, while he whistles a tune both sweet and unnerving, a tune I have heard somewhere before, long ago, in my youth, when I played with Billy and Perkin, and the idiot child, in fields and hills, all summer long.

And when my whole head is numb, its innards like suet, I tip my buttons back into the tin vase and replace it on the mantelpiece, and I gather the groupuscule, and address them, in a voice dripping with contempt, or in a roar, until they clap, they clap, they clap, they keep on clapping, as if I were Stalin. But I am Unstalin. Remember my name. Engrave it on a piece of putty, and carry it with you, wherever you roam.