\chapter{Shipwreck Is Everywhere}

\begin{quotation}Si bene calculum ponas, ubique naufragium est.

-- Gaius Petronius Arbiter. 
\end{quotation}

THAT IS, 'IF you consider well the events of life, shipwreck is everywhere'. Nobody considered the events of life with as much rigour as the out of print pamphleteer Dobson, and he came to agree with Petronius. Indeed, late in life he became notorious for breaking up happy gatherings, such as cocktail parties and jaunty sporting occasions and infants' birthday celebrations, by brandishing mezzotints of famous shipwrecks in the faces of those gathered and reciting, in a booming voice, \emph{The Wreck Of The Hesperus} or \emph{The Wreck Of The Deutschland}, or both.

The mezzotints Dobson clipped from a magazine to which he subscribed for many years. \emph{Partridge \& Peacock's Weekly Shipwreck News} collected accounts of shipwrecks real and fictional, usually written in lurid prose, and illustrated them with mezzotints, many from the hand of noted mezzotintist Rex Tint. Neither Partridge nor Peacock had the slightest interest in improving safety at sea, nor did they campaign for better lifeboat provision or similar initiatives. Quite the opposite, in fact. Partridge and Peacock were a gruesome pair, who relished the horror of shipwrecks, clapping their hands in unseemly glee when they received fresh tales of maritime disaster. They employed a team of backroom scribblers to empurple and embroider the basic reports which came clicketyclacking into the office on some kind of tickertapeyfaxy gubbins the duo had themselves invented.

Dobson never wrote for the magazine, although both Partridge and Peacock begged him to do so. There was one particular winter when either or both of the creepy cousins came banging on Dobson's door offering blandishments, but the pamphleteer never succumbed. Even in the depths of penury, he appears to have held himself aloof, which is the more curious when one considers how devoted a reader of the weekly he was. Odder still that shipwreck is one of the few topics, one of the few 'events of life', to which Dobson did not devote a pamphlet of his own. It is true that he penned more than one blitheringly infantile encomium upon mezzotintist Rex Tint's shipwreck mezzotints, the ones he clipped so carefully from the magazine every Tuesday morning for untold years and which, late in life, he took to pressing upon the attention of jolly partygoers, but of shipwrecks in and of themselves, he wrote not a word.

Although she did not share Dobson's macabre interest, Marigold Chew once set \emph{The Wreck Of The Deutschland} to music. She was, at the time, a pupil of grim beetle-browed composer Horst Gack, who set her the task of using Father Hopkins' great poem as the basis for a harmochronotransduction for voice, piping, valves, and flute-to-be-played-while-standing-on-one-leg. Legend has it that she tried to get Dobson to sing the words during rehearsals in a farmyard barn, but that the project had to be abandoned when cows toppled over and goats got the vapours, hens became hysterical and rooks and bluebirds plummeted from the sky.

