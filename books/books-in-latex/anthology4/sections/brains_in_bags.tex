\chapter{Brains In Bags}

IT IS, I think, common knowledge that by eating the brains of certain animals we can boost our own mental powers. Granted, this is not a practice which has won the backing of the greatest living Maestro of the Mind, Tony Buzan, but the results can only be described as buzantastic. The difficulty, of course, has always been obtaining brains in the first place, and making them edible. Few of us are so ruthless that we would consider tearing the brains out of the heads of our domestic pets, our cats and dogs and budgerigars, and in any case, those are not the kinds of brains that will do much to supercharge our mental abilities. I know a poor soul who lived on a diet of budgerigar brains for a week, and he is now fit for little else but dribbling and writing features for the \emph{Guardian} weekend magazine. Similarly, although your local zoo will provide a far greater range of animal brains, some of them particularly mind-enhancing such as the brains of giraffes and of exotic birds, zoos tend to have security guards who will Taser you without compunction should you creep towards the enclosures at dead of night armed with a jemmy, a skull-slicer, and a spoon. Being Tasered does not improve your mental prowess, despite what you may have read in the \emph{Guardian}. That article was written by budgerigar-brains man.

It is a very welcome development, then, that there is a new section on the delicatessen counter at Hubermann's where lucky shoppers can buy a huge variety of boil-in-the-bag animal brains at ridiculously low prices. The selection seems to have been made with human mental agility boosting as the basic criterion, for we can find the brains of weasels and pigs and crows and cows and giraffes and hoopoe birds and jellyfish and starlings and wolves and locusts and okapi and trout and flamingos and bears and monitor lizards and corncrakes and carp and badgers and hornets and lobsters and ducks and gazelles and dozens of others, all conveniently packaged and ready to boil.

Faced with such a cornucopia there is an obvious temptation to go overboard and stuff your gob with particularly toothsome brains, such as those of the rooting hog. This is why the staff at Hubermann's are fully trained to advise on the government's five-a-day guidelines, and hand out free leaflets with every purchase. To maximise your brain potential, it is important to follow certain tips:

Your daily intake should include the brains of five different animals

Make sure you boil the brains in the bag until they are piping hot

Do not eat the bag

Best accompanied with a side dish of suet pudding

Having said that, there may be occasions when, in order to boost a particular area of your mental apparatus, a judiciously limited diet can be helpful. For example, you may wish to improve your ability to interpret the scores of the more complex madrigals of Thomas Weelkes (1576-1623), in which case you might want to eat a couple of boil-in-the-bag conger eel brains for breakfast and supper each day. Studies have shown that there are substances in the brains of all eels, but especially the conger, which stimulate those parts of the human mind receptive to madrigal score complexities. Admittedly, these studies are very much in their early stages, and have yet to be given the imprimatur of any recognised academic institute, but the experiments conducted so far have been more than promising. Separate research is being done by historians of both eels as food and of choral music on whether Thomas Weelkes himself ate the brains of conger eels during his time as a Gentleman Extraordinary at the Chapel Royal.

Generally speaking, however, unless you have a specific mind empowerment scheme you wish to propel forward, it is best to stick to those five-a-day guidelines. Make sure you pick up one of the leaflets from Hubermann's delicatessen counter, and study carefully the many diagrams in the fold-out section so you can learn to tell the difference between the various animal brains available, as it must be said that they all look quite similar when packed into bags.