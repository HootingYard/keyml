\chapter{Binder's Fogwife}

SHORTLY AFTER COMPLETING his twelfth symphony, Binder decided to take himself a wife. He had been distressed at the critical reception given to his latest work, which was considered to be extremely sordid. Binder felt that perhaps some feminine influence might temper his moral grubbiness. But he knew no women, at least, no women who would consent to wed him in a million years. Though he was a successful composer, he lacked a certain vital human pith. This, at least, was the judgement of his confidante, the dwarf Crepusco, who told Binder that he seemed more like a cardboard cut-out or a man of cellophane, rather than a person of flesh and blood. Poor Binder!, we might think, yet there was truth in the charge. Ever since his days at the Academy For Tiny Musical Magnificos, he had lived in a sort of rarefied ear-world. If he were to bag a wife, that would have to change.

The composer was living at the time in a godforsaken estuarine village surrounded by marshes and mud flats. So one foul misty Thursday evening he hied himself down to the tavern, where he thought he might meet a bride. But upon clumping through the door, he found only men, coastal peasantry and invalid sailors for the most part. He fell in with a couple of sharkers and swaggerers who caroused and quaffed in large silver cans to his health. Fellows they were that had good big pop mouths to cry Port a helm Saint George, and knew as well as the best what belongs to haling of boilings yare. By chance, this pair, whose names were Vermig and Beamish, were themselves both due to be married. Binder, sipping from his mug of brewer's fudgemuck, congratulated his new pals, and asked if perchance their brides-to-be had a friend to whom he might be introduced. He had already, before leaving for the tavern, taken some tips from Crepusco regarding vivacity, dash, and \'{e}lan, and was eager to practise his skills. To his surprise, Vermig and Beamish fell about in hysterical fits, spitting and sloshing and slapping their sides. Binder blushed. Eventually his companions becalmed themselves, and Vermig spoke.

'We know not yet who our brides will be,' he said, 'Tomorrow, we are heading up into the great purple hills yonder, where we will find ourselves a pair of fogwives. Come with us, dear chap, and there will be a fogwife for you too!'

Binder had no idea what Vermig was talking about, but he reasoned that a fogwife was better than no wife at all. He arranged to meet the ruined sailors next morning at the crossroads, all togged up and kitted out for a-roaming in the purple hills.

And up in those hills, the next day, in mid-afternoon, Binder and Vermig and Beamish chanced upon a trio of hill-women tending their pigs on a vertiginous slope. Both the women and the pigs were lopsided, for they spent all their lives on steep gradients. The would-be bridegrooms learned that their would-be inamoratas were known as the Ellipses, for there were three of them and they were each named Dot. Vermig did most of the talking.

'Oh cherishable if lopsided girlies,' he declared, after they had broken the ice by talking pig lore, 'Come with us down from the hills to our misty marshy estuary paradiso, where thick fog will swaddle you, and be our wives. You may bring your hill-pigs, for though we cannot promise them the sloping land they are used to, we have much mud and muck and it oozes with briny goodness.'

Dot and Dot and Dot repaired to a crevice in the hillside, where they discussed this proposal, and found it good.

The very next day, in the foul air of the village, the three couples were married in the dilapidated church, and bells would have rung out had sound been able to travel through the enshrouding mist, the mist from the marshes that smothered and muffled and swirled sluggishly around, through night and day, until it penetrated the limbs and the lungs and brought Dot and Dot and Dot, so unused to it, surely to their sickbeds within a twelvemonth.

Dot Vermig and Dot Beamish were true fogwives, and before another year was out both lay entombed in the fogbound churchyard, and their widowers sat in the tavern gathering their wits for another foray into the hills. But Dot Binder, though she suffered much, from agues and gnawing of the vitals, was nursed back to vigour by the wondrous passing manoeuvres of the dwarf Crepusco, and she lived happily with Binder for many decades, and took two of her lopsided pigs to the premiere of his forty-ninth and final symphony, and she survived him, and set up home with Crepusco, far from the estuary, in a mountain chalet, with a balcony for tubercular guests, and an eyrie adapted for the pigs, and mezzotints of her long departed friends Dot and Dot framed upon her mantel.