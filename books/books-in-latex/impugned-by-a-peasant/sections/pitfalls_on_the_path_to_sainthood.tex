\chapter{Pitfalls On The Path To Sainthood}

WHICH OF US does not wish to become a saint? Ask most people, and they will readily admit that the idea of being venerated after death is a very appealing prospect. The paraphernalia of shrines and icons and relics are attractive in themselves, the more so when compared to the utter oblivion into which almost all of us will fall after Death taps us on the shoulder and beckons us away.

And there's the rub, of course. You have to be dead to be a proper saint, so it is not a standard career option to discuss with your lifestyle coach or your community hub outreach adviser. I am assuming here that you have such a coach or adviser, for who can be expected to make their wary way through the complexities of our contemporary paradise o' pap without one? If your coach or adviser does recommend sainthood as a viable life-skill to be added to your CV, they are delusional, and you must cut your ties with them at once. Obviously this will lead to a few days of chaotic rudderless miasmic turmoil until you get a new coach or adviser, but better that than a fruitless attempt to achieve living sanctity.

That said, there are certain things you can do, while still alive, to prepare for your canonisation. Depending in large part upon your general health and vigour, and taking into account any dangerous medical conditions, the path to sainthood can be a long one, and there are many pitfalls along the way. When the time comes for your suitability to join the pantheon of enshrined ones is to be assessed, great store will be held by how you conducted yourself in various situations. It is well to be mindful of this, even when no witnesses are present to watch you comport yourself, for someone somewhere will act the tattle-tale, you can be sure of that.

Some activities are altogether safe, in that you need not worry overmuch about besmirching the purity of your soul while engaged in them. Hiking to a picnic spot in an area of outstanding natural beauty, and picnicking thereupon, while watched by sullen cows, is unlikely to threaten your future sainthood. But such opportunities are surprisingly rare, and you cannot spend your entire life on hikes and picnics, much as you might want to. So it is important that you beware of those occasions when it is oh so easy to tarnish your record.

Consider, for example, that you are out a-strolling by the railway sidings, sidings where it is known from time to time for enormous out-of-control locomotives to come thundering along the track, their drivers rendered incapable a couple of miles back by the sudden incursion into the engine cab of a darting hawk or crow. Your path crosses that of a baffled and woebegone orphan, come to pick primroses and peonies to brighten its dank hovel in the slums. What you must do is to resist the temptation to shove the orphan into the path of the oncoming train, cackling evilly as you do so, and thereafter twirling your mustachios like the most hackneyed of stage villains. If you follow your natural impulses, and enact this terrible deed, you are doing a grave disservice to your chances of nabbing that posthumous shrine where thousands will come to worship one of your bones. Instead, reach into your pocket and take out tuppence and give it to the orphan with the instruction to buy itself a choc ice or a toffee apple. No longer woebegone, the orphan will scamper away towards a tuck shop, well away from the path of the screeching train. Be careful, however, that the tuck shop is not on the other side of the railway sidings, for then the orphan will have to cross them to purchase its treat, and it may mistime its steps and end under the wheels of the giant locomotive after all. Though you would not be as culpable as if you had pushed the orphan deliberately, a forensically-minded devil's advocate at a later date may twist the facts sufficiently to have your motives questioned, with fateful results.

What we can learn from this example is the necessity of being aware at all times of the potential for mishap. Let us say you have an appointment to see your lifestyle coach - the new one, that is, not the delusional one you have dismissed. You arrive at the skyscraper and have, in your pippy bag, along with your usual jumble of tat, a copy of the latest issue of \emph{Vacuus Purgamentum} magazine, in which you have highlighted an article recommending 'best buys' in pointless gossamer fripperies. The purpose of the visit to your lifestyle coach is to seek their counsel regarding the frippery that is just right for you. But also tucked into your pippy bag is a small canister of poison gas spray. You must be very careful, when reaching into your pippy bag at the beginning of the interview, to take from it the magazine and not the canister, for if you have the latter in your hand you are likely to aim it at your lifestyle coach and depress the knob atop the canister. Remember that poison gas is usually lethal, and you may end up having to explain your slip to both an ambulance crew and officers of the law. If an incident like this comes to light when you are being appraised for sanctity, those supporting you will have their work cut out.

Most people are aware that a condition of sainthood is the performance of miracles. Pretty much every saint has at least one attested miracle to their name, so it is understandably tempting to devote some of your time on earth to learning conjuring tricks. The reasoning is that if you can, say, produce a rabbit from a previously empty cardboard box, or saw a goat in half and then make it reappear whole, to resounding applause from an audience of credulous ninnies, or pre-school infants, then such an act can count as your miracle. I am afraid to say that this is drivel. The only miracles worth their salt are ones performed when you are already dead, and there is not much you can do to ensure that a crippled mendicant sprawled in front of your shrine beseeching you to restore their withered limbs gets up and walks away with limbs duly unwithered. A few stop-at-nothing wannabe saints have tried to arrange for such exciting scenes to take place after their demise by bribing down-at-heel actors from their deathbeds, but it is a ploy not without its risks, and rarely succeeds. The average down-at-heel actor will simply flit from your side as you groan your last and spend all the coinage on strong drink, waking up some days later in a ditch with no memory of the bargain they struck.

A further pitfall for the over-ambitious saint-to-be is to be in too much of a hurry to identify the trade or beast or sickness etcetera of which you would like to be patron. Most such patronages have already been allocated, and although there are some duplications, you should not rest your hopes on any particular role. You may have your heart set on becoming the patron saint of hedgehogs, and spend much of your time doing good works among the hedgehog population, setting up sanctuaries and so forth, but this is no guarantee that you will ever be granted your desire. Those who decide on these matters are notoriously fickle, and you may find that all those hours and days and years spent feeding milk through a funnel to injured baby hedgehogs rescued from entanglement in bramble patches would have been better spent in the company of, say, seabirds or television chat show hosts.

As a general rule, lead a blameless life and avoid poison gas canisters and the temptations of railway sidings.