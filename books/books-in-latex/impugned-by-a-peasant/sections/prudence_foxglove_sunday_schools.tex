\chapter{Prudence Foxglove Sunday Schools}

\emph{This piece was written by Vladimir Ilyich Foxglove}
\\
\\
IN ONE OF his interminable screeds, Mr Key at one point tries to emphasise the horribleness of the Horrible Cave by suggesting that 'it makes every other cave seem like a Prudence Foxglove Sunday School'. We are meant, I think, to gasp and gawp at the force of the contrast. Well, I for one do not. I am Prudence Foxglove's grandson. I was both a pupil and, later, an instructor in her Sunday Schools. What I know, which Mr Key clearly does not, is that each and every one of my grandma's religious and moral education hubs was situated in a deep and dark and dank and damp and gloomy cave. This makes a nonsense of Mr Key's attempt at vividness.

My grandma decided to set up a network of Sunday Schools because she was much troubled by the moral dereliction she witnessed all around her. In a letter to the football ace known as 'the daisy cutter', Steve Bloomer (1874-1938), she listed some of the things she had seen on a weekend outing:

\emph{I saw a gravedigger with an unwaxed moustache, a butcher without a hat, a tiny cadet loitering near a bordello, an urchin with rickets whistling in the presence of a widow, a tippety fellow manipulating figs, a wretch in the gutter, a Papist on the loose, and I saw much else besides.}

Convinced that the nation's moral resurgence could only be effected by brainwashing every five-year-old in the land, Prudence Foxglove fell under the spell of the pedagogue Walter Mad. Mad is best-known today for his involvement in a curious postage stamp scandal, but it ought not be forgotten that he was the author of dozens and dozens of tracts, the one that compelled my grandma's attention being \emph{An Essay Upon The Brainwashing Of Five-Year-Olds Through A System Of Pedagogy Conducted Within The Confines Of Dank And Gloomy Caves}.

My grandma went to meet Walter Mad to discuss his ideas. The man was an athletic, and, obviously, a most powerful ruffian. On his face he carried more than one large glazed cicatrix, that assisted the savage expression of malignity impressed by nature upon his features. And his matted black hair, with its elf locks, completed the picturesque effect of a face, that proclaimed, in every lineament, a reckless abandonment to cruelty and ferocious passions. Prudence herself, familiar as she was with the faces of pedagogical madmen in the dreadful hours of sack and carnage, recoiled for one instant from this hideous ruffian, who had not even the palliations of youth in his favour, for he seemed fifty at the least. But appearances, as we know, can be deceptive, and Walter Mad was all bonhomie and good cheer, albeit of a stern Protestant kidney, and not remotely the ruffian reminiscent of a minor character in Thomas De Quincey's \emph{Klosterheim, or The Masque} (1832) whom he so strongly, if not exactly, resembled.

My grandma was astonished at the scale of the pedagogue's ambition. He had a vision of a vast network of caves where tinies would be entrapped from their fifth birthdays onwards, learning by rote a curriculum of decisive moral rectitude. She fretted, however, that Walter Mad seemed quite oblivious of the impracticality of his project. She fired questions at him, to which he had no answers save an airy wave of his arrestingly hairy hand.

Had he sent out a scouting party of spelunkers to identify caves suitable for his purpose? Conversely, was it his intention to employ a gang of geologists familiar with explosives to blast brand new caves where now stood only grim forbidding rock? How would he propel the tinies into his caves? Did he see himself acting as a sort of Pied Piper of Hamelin figure? Was he not concerned that so prolonged a stay in the cavernous gloom would ruin the sight of the tinies and render many of them almost blind?

Prudence swept away from their meeting convinced that Walter Mad was a nutcase, and a sinister one at that. She wondered if he had more in common with the Klosterheim ruffian than she had at first supposed. I am not sure if she ever learned of his later involvement in that postage stamp business, but had she done so, she would have felt to some extent vindicated in her decision, when she arrived home, to make a waxen doll of Walter Mad and to stick it with pins, with many, many pins.

But my grandma saw, too, that beneath all the weirdness, there was a grain of sense in his cave-based education hub scheme. His plans were too lavish, too grandiose, but reined in and properly organised, they could, she thought, usher in a transformation in civic life. It was this insight that made her, rather than Walter Mad, the true visionary.

She wrote her own version of Mad's tract, entitled \emph{An Essay Upon The Brainwashing Of Five-Year-Olds Through A System Of Pedagogy Conducted Within The Confines Of Dank And Gloomy Caves, But Only On Sundays}, and threw herself into making it a reality. She took on an assistant, a man even more like the Klosterheim ruffian than Walter Mad. This fellow's name was Ed Balls, incidentally, though as far as I know he was unrelated to the erstwhile Labour government minister. Balls sought out apt caves, kitted them out with Sunday School furniture and equipment, and appointed tweedy bespectacled types to teach the tinies, allowing Prudence Foxglove to concentrate on devising the curriculum, much of which was devoted to the more alarming passages in the Old Testament, with a leavening of twee drivel about fluffy animals and pretty flowers and submarine warfare.

Two generations of five-year-olds were brainwashed at my grandma's Sunday Schools, and it is an experience none of them forgot. Balls, as we know, turned out to be a less than competent judge of caves, for one by one they suffered collapses, or flooding, or seismological trauma, until no trace of Prudence Foxglove's magnificent obsession remained.

In closing, may I say what a fantastic time I have had writing this piece for Hooting Yard, and should I be invited to contribute again, I have a cupboard full of articles ready to publish, including several about beatniks, one about the shovelling of agricultural waste materials, and a potted biography of Ed Balls. Not my grandma's assistant, but the new one, the ex-government minister.