\chapter{Swans In Thunderstorms}

\begin{quotation}
'There is a superstitious belief that swans cannot hatch their eggs unless a storm is raging, the sky mad with lightning bolts and thunderclaps. I suspect this is true, for on the countless occasions I have gone blundering into swans' nests, I have never seen eggs hatching, and the weather has invariably been balmy, for that is the kind of weather I prefer when blundering about among the nests of swans.' 

-- from \emph{Quite A Few Things I Know About Swans} by Dobson (out of print)
\end{quotation}

WE HAVE ONLY Dobson's word that he blundered into 'countless' swans' nests during balmy-weather expeditions, and a number of commentators have cast doubt on this account. The out of print pamphleteer was probably lying through his teeth, just to make a point, though quite what the point is is one of those ineffable Dobsonian mysteries the like of which will keep students busy for the next thousand years.

One man who certainly did pay visits to swan habitats, in both balmy weather and thunderstorms, was Ah-Fang Van Der Houygendorp, the artist and mountaineer who perished in the Hindenburg disaster. Although his ornithological studies were decisively amateurish, even flawed, they were sincere, and he approached them with great gusto.

Ah-Fang first became interested in swans when he was asked to draw one by a manufacturer of matches. Incredibly, he had no idea what a swan was, and had to be shown engravings in a large and exhaustive encyclopaedia. It then had to be patiently explained to him that the swan was a type of bird, fond of watery places. Ah-Fang was on his way to the seaside, ready to rent a sailing vessel and ply the oceans, when he received an urgent message on his portable metal tapping machine enlightening him as to the difference between the salt sea and the oceans on the one hand, and ponds, lakes, and rivers on the other. He kept a copy of this communiqu\'{e} in his pocket until the end of his life, not out of any sentimentality, but simply as an aide memoire whenever he came upon a body of water.

Ah-Fang could of course have copied a swan from the illustrations in the encyclopaedia and earned his matchmakers'-money, but he prided himself on always drawing from life. Sometimes this could prove a considerable challenge, as when he was commissioned to provide a set of plates for an edition of H P Lovecraft's \emph{At The Mountains Of Madness}. He never spoke of the circumstances in which he drew so vividly 'that nighted, penguin-fringed abyss', for example, and indeed, when questioned, Ah-Fang trembled with an authentically Lovecraftian shudder.

Once he was on the right track in terms of watery habitats, however, the depiction of swans was a much easier task. Ah-Fang saw his first swan on St Clothard's Day 1924. He had been told, by whom it is not clear, that there was a pond within hiking distance of his temporary quarters, a shack on a patch of waste ground somewhere in the foothills of a fantastic mountain. Taking a flask of aerated lettucewater, some ready-toasted smokers' poptarts, a map, a pad and a pencil, Ah-Fang headed off towards the pond whistling an air by Hurlstone. When he got there, he sat on a municipal bench, spotted a swan, executed a quick sketch, ate and drank, and hurried home before becoming drenched by teeming rainfall. He worked up the sketch into a finished drawing with his customary \'{e}lan, popped it in the post to the match manufacturer's agent, and sat back to await his payment.

Yet he found himself unable to relax, and in the following weeks was drawn back to the pond again and again, whatever the weather, to gape at swans, hardly bothering to sketch them. It was a stormy season, as it often is after St Clothard's Day, according to folk wisdom, and Ah-Fang had much opportunity to observe swans beset by thunder and lightning. He left no record of seeing eggs hatched during a storm, and it has to be said there was a profound, if endearing, ignorance in his gaping. Ah-Fang did not actually understand what he was looking at. Perhaps it was this pop-eyed, empty-headed stupidity that made him the artist he was, one whose swan pictures now fetch preposterous sums.

One night by the pond, as storms blasted the sky and a gale howled, Ah-Fang was accosted by a mysterious figure who hove towards him from out of the darkness. Hoisting his lantern to see plainly who, or what, it was, Ah-Fang had only a moment to look before the light sputtered out. The figure wore a cloak, but her face was momentarily visible, and she bore a striking resemblance to, and may even have been the ghost of, Captivity Waite, the childhood sweetheart of Eugene Field, author of, among other works, the children's favourite \emph{Wynken, Blynken, And Nod}. When she spoke, it was with a voice both sepulchral and sweet. She told Ah-Fang Van Der Houygendorp that there were in the world other ponds, and other swans, and there were lakes and meres and rivers and streams where yet other swans might be found, and that he should go to them, one by one, and gape, and sketch, and work up his sketches, and even paint, in great splodges of emulsion, upon sheets of hardboard, the swans he saw, all around the world, in watery places. And then Captivity Waite, if indeed it was she, made a delicate gesture of farewell, and walked away into the howling blackness of the night.

And so, next morning, began the five tremendous years of Ah-Fang Van Der Houygendorp's so-called 'world swan tour'. Without the guidance of Captivity Waite, or her phantom, however, there were false starts. Ah-Fang spent three months wandering in Arabia Deserta without seeing a single swan, and a further week in the jungles of Borneo. He returned, battered and sick in spirit, to his original pond, thinking he had been in some wise deceived. The sight of the swans he knew, in balmy weather and in storms, coddled him, and he revived. He realised that, if Captivity Waite had spoken the truth, and there were indeed swans elsewhere in the world, he would have to find a more reliable way of tracing them. He hit upon the method of hanging around in the sorts of taverns frequented by waterbird enthusiasts, listening in on conversations, picking up clues, gradually learning the whereabouts of hundreds, even thousands, of locations where the chances of finding swans were high. Sometimes, of course, his information was flawed, or he misheard a significant detail in the rowdiness of whatever tavern he was loitering in, and he would take a long and uncomfortable train journey to a particular pond only to discover it brackish and stagnant and home to nothing but weird, almost Lovecraftian algae. But more often than not, Ah-Fang's unwitting informants sent him in the right direction, and that is why it is thought that he saw more swans in his five year tour than anyone had ever seen before.

He did not sketch them all, but those he did he worked up, as Captivity Waite had suggested, into huge paintings, swan after swan after swan, some in twilight, some in a blazing sun, some on ponds, some on lakes, some on meres, and some gliding with swanly grace down rivers and streams and even canals. Ah-Fang painted swans in balmy weather and in foul weather, and many a time in thunderstorms, brooding over a clutch of eggs. Many of the pictures look, to the untrained eye, almost identical, for Ah-Fang was never the most skilful of draughtsmen, and his skills lay in sloshing great daubs of emulsion on to his hardboard with haphazard zest, relying on a stencil to capture the basic swan-shape he sought. Curiously, the stencil was cut for him by a man named Bewick.

