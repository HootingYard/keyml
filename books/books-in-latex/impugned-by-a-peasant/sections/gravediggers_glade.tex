\chapter{Gravediggers' Glade}

THERE IS A glade in the woods where, once, gravediggers gathered of an evening when their day's digging was done, and so it is known as Gravediggers' Glade. They came directly from their graveyards, and brought their spades with them, leaning the spades against the larches and laburnums and sycamores that dotted the glade. Some came from graveyards far away, too far to trudge on foot, and those gravediggers came on their donkeys, with their spades tied to panniers. There can be something Christlike about a gravedigger riding a donkey along a country lane, particularly if he has a beard and a soulful look in his eyes and is dressed in a white shift. But no competent gravedigger wears white, for gravedigging is filthy work, particularly during rainfall, it is work that throws up clods of earth and muck with which the gravedigger's clothing is splattered, and so he will wear black or brown or beige, and rather than a shift he will wear overalls or dungarees, of tough cloth, if he knows what he is about. Even a gravedigger so clad, if he has a beard and a soulful look and is wending his way astride a donkey towards the glade as the sun sets can resemble Christ, however, if not quite so perfectly.

What must a spectator make, then, of a continual stream of Christs, one after another, in the evening, on the lane, as they head for their gathering in the glade? Some are on their donkeys and some are on foot, but even the latter can look like Christ, during His Passion, carrying their spades as Christ carried His cross. What a sight, indeed, and one you would have seen if, all those years ago, you had been a peasant tilling his patch beside the lane, or lolling in a haystack knocking back a flagon of cider, or engaged in some other rustic evening pursuit. Thomas Hardy wrote about such things, but as far as I know he never witnessed the parade of gravediggers on their way to Gravediggers' Glade in the woods. These are not Wessex woods.

There was little that was Christ-like about their gatherings. They leaned their spades against the larches and laburnums and sycamores, and those who had come far tied their donkeys to the same trees with donkey-tying string, so the donkeys would not stray, although they let their spades remain empanniered rather than removing them to lean against the trees alongside the spades of their fellows. No, they no longer looked Christ-like as they gathered, all in black and brown and beige and matted in muck, muck which was splattered in their beards and their hair as well as upon their tough gravediggers' clothing.

And, gathered together, they began to grunt. They grunted softly, and loudly, and kept on grunting until they had coaxed the Grunty Man, that monster from the bedtime stories of their childhoods, from his lair up in the hills. The Grunty Man came bounding down to Gravediggers' Glade at inhuman speed, hairy and slobbering and grunting, and as soon as he was among them, the gravediggers fell silent. They stooped to pick pebbles from the ground, and they chucked the pebbles at the Grunty Man, many, many pebbles, but not with great force, for they wished to tease him rather than to harm him. And when they had exhausted the pebbles, the gravediggers began to sing. They all had sheet music tucked into the pockets of their black and brown and beige overalls or dungarees, and they lined up as a choir would line up, and they belted out in their gruff gravedigger voices selections from Charles Ives' self-published collection of \emph{114 Songs} (1922). They sang \emph{At Sea} and \emph{Charlie Rutlage} and \emph{Like A Sick Eagle}. They sang \emph{Luck And Work} and \emph{Grantchester} and \emph{Ich Grolle Nicht}. They sang \emph{Songs My Mother Taught Me} and \emph{The Housatonic At Stockbridge}. They sang \emph{Marie} and \emph{Rosamunde} and \emph{Mists} and \emph{Watchman} and \emph{Those Evening Bells}. And they sang \emph{Tom Sails Away}, as a sort of farewell to the Grunty Man, for by now he had sailed, or scampered, away, for he was frightened of singing, and the gravediggers' songs always made him flee back up to the hills where he cowered in his lair until lured back to the glade by the grunting of the gravediggers when next they gathered there of an evening.

Fear stalks the countryside, especially fear of the Grunty Man, and that is how the gravediggers held their fear at bay. They could have sung songs by other composers, of course, by Schubert or Schumann or Peter Warlock or Peter Blegvad or Yoko Ono, but each of them was fond of Ives and they had splurged their wages on the sheet music for the \emph{114 Songs}, trooping into Dennis Pigstraw's sheet music shop in the village of Cack posing as a choir. All singing terrified the Grunty Man. Teasingly pelted with pebbles and sung at, he never learned that he should ignore the grunting with which he was coaxed from his lair, for his brain was tiny and hot and pitiable, and every single evening he fell for the same trick. Had it happened in Wessex, Thomas Hardy would have written about it, I am sure. But there is much that happens elsewhere in the countryside that no one speaks of or writes of, and that is as much a pity as the weakness of the Grunty Man's tiny hot brain.

