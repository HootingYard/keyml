\chapter{Along The Banks Of The Smem}

\begin{quotation}
``Many people have a prejudice against goat's milk, thinking it has a peculiarly goaty flavour. This misapprehension has probably arisen from the experience of tourists in Switzerland and Italy where goat's milk is in common use, and frequently offered in mugs or glasses which have not been properly cleaned.''

-- H S Holmes Pegler, 'Goat-Keeping', \emph{The Listener}, Vol I No 16, 1st May 1929.
\end{quotation}

THE ENGINE GAVE a hoarse shriek; we had arrived at Pinpotting, or Pottingpan. The black coaches of the train waited a minute in the silvery light of the mountain, disgorging a miscellaneous collection of people and swallowing others. Peppery voices could be heard up and down the platform. Then the wheezy engine at the front squeaked again and drew the black chain rattling away into the cavernous tunnel. The broad sweep of country lay pure and peaceful once more, with its sharply etched backcloth scoured bright and clean by the damp wind. It was good to breathe the air. I was one of those who had disembarked from the train, and I stood waiting on the platform until it was empty but for the guard, who soon vanished into his hut.

I had come to this mountain village, with my peg-leg and my religious hysteria, on the advice, even the orders, of the family physician. In his twinkly shouting guttural manner, Dr Gobbo insisted that a six-month stay in the clean mountain air would restore to me the gusto I had lost. For my part, though I did as he suggested, I was unconvinced. My life thus far had been a catalogue of maladies, mishaps, and calamities. I had an ague shortly after I was born, and then, at about three or four years old, I had a grievous ague. I vomited for twelve hours every fortnight for years. This sickness nipt my strength in the bud. At eight years old I had an issue in the coronal sutor of my head which continued running until I was twenty-one. One October I had a violent fever, it was like to have carried me off, 'twas the most dangerous sickness that ever I had. At fifteen or sixteen I had the measles, but that was nothing, I was hardly sick. I had a dangerous fall from my uncle's horse. The following year I had smallpox. When I was twenty I had a fall and broke one of my ribs, and was afraid it might cause an apostumation. Much later coming back from abroad I was like to be shipwrecked but no hurt done. The following year I had a terrible fit of the spleen and piles. Then I received laesio in testiculo, which was like to have been fatal. After that my affairs ran kim kam, there were treacheries against me. A couple of years later an impostume broke in my head. Also I was in danger of being run through with a sword, and in danger of being drowned twice. That year I was in great danger of being killed by a drunkard in the street, but one of his companions hindered his thrust. Now, standing on the deserted railway station platform, I mumbled a prayer to several saints, asking them to protect me from further harm. Perhaps Dr Gobbo was correct.

I set off towards my hotel, a mile or two distant on the banks of the Smem. Seldom had I seen a river so teeming with fish. I hoped to find, upon arrival at the hotel, that my room overlooked the river, that I might be able to spear fish from the comfort of my balcony. I had brought no spears with me, but could spend happy hours whittling sticks gathered in the gorgeous woodland. I would need to obtain some string, to attach to my whittled spears in order to be able to haul them back to the balcony, with, I hoped, a bream or gudgeon impaled upon them. I was confident, from my knowledge of Mitteleuropean mountain village hotels gleaned from various encyclopaedias, that string would be the sort of item available in a little shop attached to the hotel, much like a church repository. From my perch upon the balcony of my room, armed with string and sticks whittled into spears, I might well be able to provide myself with enough fish for my dinner each day, and thus be spared the ordeal of mucking in with the other guests in the dining room, whom I feared might snigger at my peg-leg and be dismissive of my religious hysteria. I knew only too well that Satan can lurk even in the bosom of the most innocent-seeming Mitteleuropean mountain village hotel guest.

These thoughts of succulent and private fish dinners made me peckish as I followed the path along the bank of the Smem. There was as yet no sign of the hotel, so as I approached a peasant's hut I decided to stop and ask if I might be given a snack. I had not had the opportunity to change my bank draft into the coinage of this country, assuming that I could do so at the hotel, thus I readied myself to bestow grand and holy benisons upon the peasant through the power of my voice and by swinging a tin censer from my unwithered hand. Pausing by a clump of edelweiss, I lit the censer with my World War One platoon sergeant's pump gaz lighter, then clonked up to the door of the hut and hammered upon it.

The peasant who appeared in answer to my knocking was, I am afraid to say, an irreligious lout who stank of goat. The sacred smoke from my swinging censer had absolutely no effect upon his morals. As I am sure you can appreciate, I was thoroughly perplexed at his immunity, and the consequent knotting of my tongue and clogging of my throat meant that I had much difficulty making myself understood. What ought to have been a simple snack request came out as a strangulated cry of spiritual desolation. To my surprise, however, he gestured for me to follow him into the gloom of his hut.

Within, all was filth and grease and squalor. Until now, I had harboured a hopelessly romantic view of the lives and habitations of Mitteleuropean mountain village peasantry, based to some extent upon my musings upon John Ruskin's magnificent, yet sadly unwritten, study of Swiss towns and villages. I had also watched \emph{The Sound Of Music} on more than one occasion, which explains why, despite being a botanical ignoramus, I was able correctly to identify the clump of edelweiss next to which I had paused just moments earlier.

The peasant was blundering about in the corner of his disgusting parlour, and now he emerged, bearing a beaker of milk. Though he was a sinful man, it was clear he was offering it to me as refreshment. What I wanted was something more substantial, involving pastry and salted fish and black cherries, but I supposed that some solid sweetmeats might follow, so I took the beaker and gulped down the contents in one go, to show my appreciation. Yuck. I was immediately reminded of those childhood days of fortnightly vomiting. The milk had a peculiar goaty flavour, which I ascribed to the fact that the beaker in which it came was, like everything else in the hut, the peasant included, unwashed. It would have been rude of me to suggest to the peasant that he and his beaker and each of his appurtenances would benefit from sponge and soap, so I held my tongue, now thickly coated with milk residue. I still hoped for food, even though whatever I was offered would, I supposed, be grubby and begrimed. But the peasant snatched back the beaker and flailed his arms as if shooing me away, like one of his goats. I gave the censer a desultory little swing, to waft some sanctity into the midden, gagged on the aftertaste of the goaty milk, and backed out of the door, which was immediately slammed shut. I had not even learned the peasant's name.

I looked up at the mountains. These were the steep snow-covered slopes that fictional athlete Bobnit Tivol had sprinted up and down, for hours at a time, as part of the rigorous training regime devised by his coach Old Halob, in the early years before he won all those medals. Peg-legged, I could never hope to emulate the spindly wastrel, try as I might. I allowed myself to weep. And then I gathered myself, and turned, and headed off towards the hotel, and the worst horror of all.