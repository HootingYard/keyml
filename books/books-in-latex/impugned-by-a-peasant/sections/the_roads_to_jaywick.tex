\chapter{The Roads To Jaywick}

JEAN-PAUL SARTRE'S TRILOGY \emph{The Roads To Freedom} has fallen out of fashion somewhat - as if that mattered - yet it remains a classic. But for a book with a bit more existentialist heft, I recommend Pebblehead's bestselling paperback \emph{The Roads To Jaywick}. That blighted, benighted, dilapidated seaside town, has of course, provided fodder for any number of potboilers, including \emph{Jaywick - West Of Clacton} and \emph{The Sordid Sands Of Squalor}, but Pebblehead's is a fundamentally serious work, and there is a lot about cows in it, which is always a good thing.

In an interview with \emph{The Literary Dunderpate}, the author explained the genesis of his novel. 'One morning as I was eating marmalade straight from the jar with a spoon,' he said, 'It occurred to me, in a flash of insight, that, if one is so minded, all roads lead eventually to Jaywick. Once you approach the shabby resort itself there is but the one road, pitted and unlovely and dismal, but to reach that road one must travel along many other roads, depending on where you start from. You might be in Gore Pit or Fingringhoe or Vange, or even in Messing or Fobbing or Dengie: it doesn't much matter, for there will be a road wherever you are that will lead you inexorably to the windswept collapsing hovels of Jaywick. Lord knows, even from Mambeg and Clynder, if you have Jaywick in your soul you will find a road to take you there.'

That last phrase is telling. Pebblehead originally planned to call his titanic masterpiece \emph{A Jaywick Of The Soul}, but decided against it. 'It is true enough,' he explained, 'that there is a sort of psychoJaywick that lurks within the mind of every man and woman on the planet, but I wanted to insist upon the real, physical Jaywick, that place where the sickened traveller can come to a halt and go for a pint in the Never Say Die and get their head kicked in by feral Jaywick youths.' Pebblehead goes on to describe the transfixing sense of Weltschmerz he felt when peering over the sea wall and seeing, on the gruesome beach, a big sign warning him of 'Danger - Keep Off'.

As a bestselling paperbackist, Pebblehead has sometimes been criticised for being shackled to realism, and in the process of writing \emph{The Roads To Jaywick} he did indeed test out his thesis by travelling by road to those glum coastal shacks from a variety of starting points. He proved that roads from Threekingham and Scratby and Snodland, from Coffinswell and Mugdock and Crundale, from Hoo and Swillington and Catbrain and Widdop and Slack and Splat, from each of these, in a cart pulled by inelegant horses, he could, eventually, reach Jaywick. And that, he suggests, is what makes us human, arguing his case with a vivid account - taking up more than two-thirds of the book - of the famous incident known in Jaywick lore as The Day The Cows Came Visiting. The cows, of course, came not by road but, being cows, across fields, across flat hopeless fields on a misty morning. It is a haunting tale, and one to which Pebblehead's gorgeous prose does justice.

I believe it is a great scandal that \emph{The Roads To Jaywick} is not a set text to be read by tinies in all the community education hubs in the land. It is all very well filling their heads with the likes of Sartre and De Beauvoir and Norman Spinrad, but those who devise the curriculum will reap a whirlwind. Better by far, surely, to envision a nation in which our urchins sit enraptured in their study pods, lapping up the timeless words of Pebblehead? Pebblehead who, for the sake of literature, lay drunkenly sprawled in the gutter outside the Never Say Die in a wretched seaside hellhole in the spooky mist, at risk of being trampled by roaming cows whose roaming brought them, as if by some uncanny cow-controlling propulsive force, across the fields to Jaywick, west of Clacton.