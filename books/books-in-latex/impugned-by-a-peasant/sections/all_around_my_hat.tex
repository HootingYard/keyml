\chapter{All Around My Hat}

\emph{ALL AROUND MY Hat} is an English folk song, popularised in the 1970s by folk rock titans Steeleye Span. Their version was very similar to the one published in \emph{A Garland Of Country Song} by Sabine Baring-Gould in 1895. Apart from folk song and folklore collections, Baring-Gould wrote hymns, a sixteen-volume \emph{Lives Of The Saints}, many novels, a study of werewolves, grave desecration and cannibalism, and a biography of Robert Stephen Hawker (1803-1875), the eccentric country vicar who spent much of his time smoking opium in a clifftop hut made from driftwood, talked to birds, dressed up as a mermaid, excommunicated his cat, and had a pet pig.

'All around my hat' are also the opening words of one of Dobson's more curious pamphlets, in which he describes wearing a hat lined with lead to deflect weird invisible rays aimed at his brain. It is not clear who, or what, is sending these putative rays, nor why the Dobsonian cranium needs to be protected from them.

'All around my hat' writes the pamphleteer, 'the air is a site of constant barrage from weird invisible brain rays!' Note the exclamation mark, an uncharacteristic touch which has convinced some critics that Dobson was fooling around. The idea that this pamphlet is an unserious blotch on the canon has gained ground in recent years, with Nestingbird, for one, going so far as to claim that Dobson did not even write it, but simply copied out random paragraphs from a booklet given away as a free gift with a packet of breakfast cereal. This argument loses a certain force when Nestingbird has to admit that he has not managed to identify the said booklet, nor the breakfast cereal. In any case, as upstart young Dobsonist Ted Cack has pointed out in a series of increasingly aggressive letters, Dobson usually ate bloaters for breakfast.

The Nestingbird-Cack correspondence is a perfect example of the way in which the minutiae of Dobson studies can be magnified to the point where common sense is blotted out, much as the bulk of a pig the size of Robert Stephen Hawker's pet pig would blot out the sun if you were sprawled in a particular patch of muck in its sty. It was a very large pig. Thus, the senior critic floats the idea of the breakfast cereal booklet, the upstart counters with the point about bloaters, the elder counters that the packet of breakfast cereal may have been purchased by and munched by Marigold Chew, the youngster replies with a computerised database of known breakfast cereal free gift booklets for the period in question, the old man picks out flaws in the research, the rookie lets loose a vituperative attack on his opponent's atrophied brain sinews, and before long the columns of a reputable literary journal read like the ravings of H P Lovecraft in his more hysterical passages. All of this can be great fun for those entertained by Dobson-related pap, but sober-minded scholars are, I think, ill-served. There is a great temptation to take both Nestingbird and Ted Cack by the scruffs of their necks and crack their heads together. Hairline fractures in the skulls of both might just allow in thin shafts of light, akin to the weird invisible rays Dobson feared may be beaming towards his own brain. Or, I should say, the weird invisible rays Dobson \emph{possibly} feared, unless of course he was just fooling around for reasons which must remain obscure to us.

Without wishing to generate further controversy over what is, in any case, a pointless and trivial matter, I should add that I have recently completed a lengthy work, at fifteen volumes just one book short of Sabine Baring-Gould's \emph{Lives Of The Saints}. It is a comprehensive study, with lots of illustrations and diagrams, of all Dobson's known and suspected hats. I conclude that not a single one of them was lined with lead.