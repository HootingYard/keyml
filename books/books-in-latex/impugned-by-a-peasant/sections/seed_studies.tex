\chapter{Seed Studies}

In Notes Pursuant To The Unravelling Of The Pamphleteer's Plum Plan, we saw how Dobson believed he could learn about seeds by peering at them. Other than suggesting 'entire afternoons' as the length of time for a seed peering session, much that we would wish to know remains unsaid. I suppose we can assume that Dobson planned to peer at the seeds of a plum tree, but if he is right that prolonged peering is in itself educational, one could, presumably, peer at any seed, or seeds. The pamphleteer refers to 'seeds' in the plural, but does not specify how many. And one is left uncertain whether, for example, he envisages peering at a single, different seed on different afternoons, or at the same array of unnumbered seeds each time. Nor are we told whether he would use a powerful microscope. It would also have been helpful if Dobson could have given some clue as to the type of surface on which his observed seed or seeds would be deployed.

I mention all these points because it seemed to me dubious in the extreme that one would learn very much about seeds, or about anything else for that matter, simply by peering at them. And yet who was I to doubt Dobson, unarguably the most important pamphleteer of the twentieth century, albeit an out of print one? I resolved to put this peering business to the test. Given the lack of detail in Dobson's account, I had to improvise somewhat.

I decided to buy a packet of birdseed, partly in homage to the famous 'phantom Dobson' pamphlet, and partly because, at the end of my experiment, I would have something to scatter on the serried bird tables of Pointy Town, about which more later.

I removed a single seed from the packet, and placed it carefully on a piece of beige cloth laid flat on the tabletop. I chose beige as it is the most neutral of colours. Then I pulled up a chair and sat down, and plonked my elbows on the table and rested my chin in my hands. Not having a microscope, powerful or otherwise, I had to rely on my myopic eyes, peering at the seed through my spectacles. It was morning, rather than afternoon, but that meant I had a longer period of daylight to do my peering.

I peered at the seed for nine hours. The following day, I placed a handful of seeds on to the beige cloth and peered at them for five hours. The next day, I took another single seed from the packet, laid it on a sheet of expensive creamy notepaper, and peered at it for eleven hours. I continued seed peering every day for two weeks, neglecting all other concerns, and varying the number of seeds and the surface upon which they rested. One day I removed my spectacles, to get a blurry perspective. When I wasn't peering at seeds, I made notes, not on the expensive creamy notepaper, which I save for letters to dowagers and contessas, but in a notebook I bought especially, on the cover of which I pasted a label reading 'Seed Studies'.

So was Dobson right? By the weird and eerie Gods that stalk the land of Gaar, of course he was! When all my peering was done, and I went a-scattering the birdseed on the serried bird tables of Pointy Town, my brain was fat with seed-lore. I have a hard task ahead of me, but I aim to marshal the notes in my Seed Studies notebook into a coherent text, to be issued later this year, or perhaps next, as The Hooting Yard \"Ubertome Of Seeds. I suspect it will make every other book about seeds ever published fit only for the dustbin. And all thanks to Dobson, to whom, of course, the \"Ubertome will be dedicated.