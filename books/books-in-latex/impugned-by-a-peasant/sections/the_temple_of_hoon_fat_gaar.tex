\chapter{The Temple Of Hoon Fat Gaar}

BECAUSE IT WAS constructed mostly from canvas and cloth, and the canvas and cloth were fed on by moths, the Temple Of Hoon Fat Gaar is sometimes known as the Moth-eaten Temple Of Hoon Fat Gaar. Ravaged by moths and time, and lashed by wild winds that blow across the \emph{tarputa}, it is a wonder the temple still stands, a thousand years after the first devotees entered it through the sacred flap. It has of course been much patched and stitched over the centuries, and its fabric is regularly stiffened with starch, carried in canisters for miles upon miles by worshippers of the hideous bat-god Fatso. For it is He to whom the temple is dedicated.

The wild winds that lash and batter the temple are meteorologically very interesting indeed. Students of the weather have been perplexed by them ever since modern wild wind studies began. Before our scientific age, of course, the sheer weirdness of the winds that blow across the \emph{tarputa} was ascribed to the mercurial and petulant nature of the hideous bat-god Fatso, for it was thought that He was responsible for them, as He was for everything in the universe. We are wiser now, but no closer to getting to grips with the wild lashing winds.

Those who still believe in Fatso have a simple explanation. For them, the winds are the physical manifestation of the temperament of Fatso's magic pig. Actually, He has two magic pigs, but we can safely ignore one of them for a moment or two. The idea is that this particular pig - which, it must be understood, is not a real pig in any sense - somehow sends the winds howling across the \emph{tarputa} whenever it is fractious or hungry or obstreperous or maddened or otherwise out of sorts. Why the hideous bat-god Fatso does nothing to placate His magic pig is an ineffable mystery. The religion dedicated to Him is short on theologians of any stripe, although one of the few to have addressed the problem contended that Fatso spent much of his time pacifying the other magic pig, which, if ever it fully awakened, would make the wild winds that batter the temple seem like tiny pipsqueak gusts of summer breeze. Other so-called scholars argued that this implied the other magic pig was somehow more powerful than Fatso Himself, a clear heresy, so the first theologian was put in a crusher and crushed.

There used to be at least five crushers on the mud plain around the Temple Of Hoon Fat Gaar, so we must assume that there were plenty of heretics to be crushed. Occasionally, a bright young whippersnapper archaeologist will announce plans for a dig at the site, hoping to exhume a fantastical hoard of crushed bones, but not one of these schemes ever succeeds. It is said that Fatso Himself sabotages the expeditions, by causing shipwrecks and helicopter crashes and by pickling the archaeologists' brains while they sleep. In these ploys he calls on the assistance of his flock of bitterns. Unlike the two pigs, the bitterns are not magical, but nor, of course, are they real. They are phantom, spectral bitterns, beholden to Fatso for some service He did them in the distant past. We cannot guess what that might have been, for it is a topic suspiciously neglected by all the priests and wizards and jumping-about men who interpret Fatso to his followers. Or, I should say, who used to do so. There are none of them left alive today, at least none that we know of. Believers in Fatso are a dwindling band, often greasy and myopic and spindly and gormless. They tend to lack \'{e}lan. Most of them, probably, would be crushed in the crushers if the crushers were still there, because one thing we can be quite clear about the hideous bat-god Fatso is that He expected His devotees to cut a dash. There may have been few opportunities for glittering social panache on the prehistoric \emph{tarputa}, especially with those wild winds, but what rare chances there were were seized on by Fatso's followers. Great attention was paid to the angles of hats, the tying of cravats, and affectations of toffee-nosed insouciance. This is not to discount a concomitant yearning for the mud, encouraged by one of the magic pigs.

So today there are few who haul their canisters of starch for miles and miles to stiffen the moth-eaten canvas and cloth of the Temple Of Hoon Fat Gaar. Perhaps in a hundred years there will be none at all. Yet Fatso himself will still, as far as He is concerned, hold sway over the universe, and His magic pig will still make the wild winds blow, and His other, even more frightening magic pig will doze and slumber, dreaming of havoc. It is easy for us to dismiss their very existence. Until, that is, we have struggled, stylishly, across the inhospitable \emph{tarputa}, and stooped down to crawl through the sacred flap, to enter the Temple. Then we see what all those believers through the centuries saw, a sight so magnificent and terrifying that we sprawl helplessly in the mud, shrieking, brains bedizened, gaga for the god of all gods.