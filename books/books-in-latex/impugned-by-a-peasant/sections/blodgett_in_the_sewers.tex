\chapter{Blodgett In The Sewers}

HAVING HEARD RUMOURS that the sewers of Pointy Town were teeming with strange huge bulbous blind albino beings with tentacles and suckers, Blodgett decided he would like to capture a pair and keep them as pets. Encased from head to toe in patent sewerwear, without a guide, our hero clambered down a metal ladder into the vast subterranean network of tunnels and passageways and channels and chambers. If you have seen \emph{The Third Man} (Carol Reed, 1949), you will grasp the essential seweriness.

The sewers of Pointy Town were the least pointy part of the town. Indeed, for whole stretches they were completely unpointy. Blodgett found this disorientating, and soon became lost. So used was he, in these latter days, to locating himself in relation to various pointy bits above ground, that their absence made his head swim, inside its big exciting helmet, and he toppled over. He had only been down in the sewers for a couple of minutes.

Although he had gone down without a guide, Blodgett was not entirely witless, and he carried with him, strung to a loop on the hip of his sewerwear, a hooter, which he could hoot to alert any members of the Pointy Town Sewer System Rescue Patrol who might be faffing about in the vicinity. He had been told about these tireless public servants by a man in a tavern, the same tavern where he had heard the rumours about the strange huge bulbous blind albino beings with tentacles and suckers. One tavern, two rumour-mongers. Unfortunately for Blodgett, the chap who told him about the rescue patrol was a chronic fabulist with a skewed brain, and he was peddling a fiction. Even more unfortunately, the hooter which this same scamp sold to Blodgett for forty panes mimicked the mating call of the strange huge bulbous blind albino beings with tentacles and suckers. And, as misfortune piled on misfortune, there was nothing remotely fictional about them!

So Blodgett, though as yet he did not realise it, was in something of a pickle. Unable to hoist himself back on to his feet, he lay sprawled on dank stone slabs, filth gushing past inches from his face. The lantern torch attached to his helmet gave him enough light to read by, so, having hooted the hooter a few times, he lit a cigarette and took from one of his pockets a pamphlet to pass the time. It was a curious piece of work by Dobson, a sort of potted biography of Hungarian football ace Ferenc Puskas intertwined with a muddle-headed meditation upon the unpopularity of certain card games. Not surprisingly, it is now out of print.

Blodgett was so engrossed in Dobson's description of the card game My Lady's Bonnet, interspersed as it was with vivid passages about the 1959 European Cup Final, that he only became aware of the pair of strange huge bulbous blind albino beings with tentacles and suckers slithering towards him when he felt their fetid breath on the back of his neck.

I know this is a particularly exciting point in the narrative, and I do not wish unduly to keep readers in suspense, but I can hear objections being raised by members of the pernickety community. How, they ask, does Blodgett feel breath on the back of his neck when he is encased in sewerwear, including a big gorgeous helmet? The answer of course is that a patch of muslin, punctured with many holes, is sewn into the sewerwear precisely at the back of the neck, just below the rim of the helmet, for reasons too obvious to elucidate.

Blodgett had just lit a second cigarette. With great presence of mind, and a rapidity of action learned in the testing ground of the Hideous Unlikely Swamp Of Scroonhoonpooge, he overpowered the two squelching horrors by poking one in the sucker with his burning cigarette and thwacking the other one on the tentacles with the Dobson pamphlet. He was then able to use the support of their huge bulbous bodies to lever himself upright.

Still puzzled at the non-appearance of the rescue patrol, Blodgett used his inhuman strength to drag the two strange huge bulbous blind albino beings with tentacles and suckers behind him as he trudged through the sewers seeking an exit. For the first time he was thankful for the unpointiness of his surroundings, for it made his dragging much easier than if he had had to negotiate pointy bits. That was a problem he would have to face when he was back on the surface, but he was consoled by the thought that he had enough coinage in his pockets to pay for the hire of a cart to carry the captured beings back to his decisively moderne chalet at the edge of town.

Later, much later, in fact three days later, Blodgett finally found his way out of the Pointy Town sewers. He hired a cart and carried his prizes back to his decisively moderne chalet at the edge of town. 'They proved,' he wrote, 'to be quite splendid pets, although I was a little upset when they ate my Toggenburgs.'

The highly amusing story of Blodgett and his goats will, alas, have to wait for another time.