\chapter{Dobson's Dinghy}

FOR A LONG time, for years and years and years, I have been meaning to write about Dobson's dinghy. It is a subject which I am convinced will be hypnotically fascinating to my readers, and yet whenever I settle to the task, as I did yesterday morning, with the nib of my pen polished to a gleam and a fresh stack of blank rectangular paper, I found myself once again baffled and plaintive. It is not that the contents of my skull seized up, like a polar ship in pack ice, for I had done my usual crack-of-dawn brain exercises, flexing the synapses using a set of techniques culled from a Victorian Everyday book. Some might say that synapses are not things you can flex, and they may be correct, but I am sure you understand what I am driving at. The point is that I was in tiptop writing condition, hunched over my desk, nib gleaming and paper stacked and blank, and outside my window crows were stalking across the grass, fat and black and Ted Hughesy, and the sight of crows seldom fails to inspire me, no matter what I am writing about. Sometimes I have filled pages and pages blathering on about crows, and then cleverly crossed out every mention of the bird and substituted it with another noun, for example windscreen wiper or bazooka, or even with a dozen different nouns, whimsically, in a great creative outpouring the like of which would put Dobson himself to shame, were he still with us. I have tried this technique over the years when trying to write about Dobson's dinghy, as a weapon against my bafflement, but it never quite works, and those pages are turned into scrap or made into paper aeroplanes or paper hovercraft or paper Hindenburg airships, depending on my mood of the moment. It can be very relaxing to fold one's abandoned manuscripts into toy forms of transport. Once, I was so thoroughly relaxed after folding half a hopeless novella into a paper fleet of milk floats that I fell into a coma. Other writers find different uses for their discarded scribblings. The poet Dennis Beerpint, I learned, tears his disjecta into thousands of pieces, with untold savagery, cursing and fuming as he does so, while Pebblehead, the bestselling paperbackist, binds all his up into a bundle with butchers' string and carries it down to the beach and throws it into the sea. Frankly, I am surprised that a writer as successful and prolific as Pebblehead ever has aborted works to so dispose of, but I am told that he is seen upon the sands at least once a week, casting his bundles upon the briny. Dobson never launched his dinghy into the sea. On very rare occasions, when the fancy took him, he would push it into a pond and clamber in and paddle it across, alarming any ducks, such as teal and coots, who got in his way. The dinghy was yellow, and made of rubber, and Dobson bought it at a closing-down sale from a ruined ship chandler's. There. There is the essence of my bafflement and my plaintiveness. I have just told you everything I know about Dobson's dinghy. Every time I have a mind to write about it, I reach the same impasse. I have exhausted the topic, and have nothing else to say. Give me a week, or a month, and no doubt I will wake up one morning and feel impelled, yet again, to try to write dozens of pages of vigorous and impassioned prose about Dobson's dinghy. I will mention the pond, the yellow colour, the rubber fabric, and the ruined ship's chandler, and that will be that.