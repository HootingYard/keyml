\chapter{Groovy}

\emph{DEAR FRANK,} WRITES Tim Thurn, \emph{It has long been apparent to me that Hooting Yard is by far the grooviest website on the planet. But how do I actually get down with its groove? Any tips would be most welcome.}

Tim is not the only person to ask this, or a similar, question. Boffins in a groovelab high in the Swiss mountains have spent years - or is it mere days? - trying to isolate the Hooting Yard Groove, for the betterment of humanity, while Mrs Gubbins has been indefatigable in her attempts to express the essence of the groove in the form of knitted tea-cosies. Every single time she picks up her needles she fails, fails better, but she goes on, she must go on, she can't go on, she goes on. We will soon have to build a new depot for all those groovy tea-cosies, unless we can find a charitable foundation prepared to accept them.

But are Tim Thurn and Mrs Gubbins and those Swiss boffins asking the wrong questions? Is there, in fact, a groove to be found? For the true horror may be that the grooviness is entirely superficial, and there is nothing behind it.

Some would have it that such absence of groove is unthinkable. The boffins, for example, having invested a huge amount of Swiss currency in retorts and alembics and bunsen burners and rubber tubing and bakelite knick-knacks and Coddington lenses, not to speak of elbow grease and sweat and pipe tobacco, would be unmoored, cast adrift upon a sea of cognitive anguish, were they to entertain the idea of there not being a groove. I am less fretful on behalf of Tim and Mrs Gubbins, for I know that both of them have other resources, the one a button fetish and the other a predilection for criminal mayhem. If they could but accept they will never get with the putative groove, Tim would be happy as a pig in muck with his buttons, and Mrs Gubbins could round up the old gang and embark upon a series of armed robberies.

Conversely, of course, there \emph{is} a Hooting Yard Groove, a groove so groovy it outgrooves every other groove ever dreamed up by the grooviest of groovers. Surely I would know about it?, you ask. Well, not necessarily. Take as an example Dennis Beerpint. Ever since the incorrigibly twee versifier transformed himself into a beatnik, he has been, unarguably, the grooviest poet who ever lived. I say 'unarguably' because there is not a soul who doubts this, not even Michael Horovitz. And yet Beerpint prances about the streets and coffee bars and milk bars and jazz clubs and happenings of his adopted world blithely unaware of his own irrefrangible grooviness. It is true that he makes much of his goatee beard, polo neck sweater, and hornrims, and that his trousers of choice are of the drainpipe variety, yet he remains free of affectation, almost childishly innocent, and reassuringly inept. But if anybody is down with the groove, daddy-o, it is Dennis Beerpint.

If it is the case that a Hooting Yard Groove truly exists, it is of a different order of grooviness to the Beerpint Groove. The two do not quite cancel each other out, but they cannot happily coexist in the same grooveosphere. Mrs Gubbins demonstrated this when she tried to knit a dual-groove tea-cosy and became so thoroughly entangled in stray skeins of wool that she had to be carted off to a clinic.

And on that cautionary note, I think I will leave it. Tim Thurn may remain in the dark about the groove he seeks, but that is the way with a groove. Once you stop looking, you might just find it. Or, if not, you can go and sprawl in a ditch and stare at the sky. It is immense, and blue, and spattered with clouds.