\chapter{Notes On Skippy}

SERIAL CORRESPONDENT TIM Thurn is perplexed. \emph{O beloved Mr Key}, he writes, \emph{For some years now, Hooting Yard has been my unerring guide, informing my opinions and attitudes and in some cases even my behaviour. Rarely will I venture a viewpoint upon any topic without first doing a mental accounting of what the Yard has taught me. Thus I find myself in a state of some beflummoxment on the subject of our canine pals. I am unmoored. I am unable to work out the approved 'line' on dogs. One day you tell us dogs are boring, but then you write, with some affection, of Skippy, a dog you feed and pamper and which appears to be your domestic pet. I cannot be the only reader who is utterly confused by these divergent dog attitudes, and would be extremely grateful if you could, in some wise, shed light upon the matter.}

I will not reproduce the remainder of Tim's wordy letter, which veers off into an account of the many and various dogs with which he has come into contact during his life. Better that I set him straight without further ado. Clearly, when reading of Skippy, Tim picked up on the words 'bark' and 'hound' and 'cur', and also I assume on the detail that Skippy is fed, from a bowl, on reconstituted meat chunks in jelly. Any reader, not just Tim, could be forgiven for thinking that I was referring to a dog. But dogs are not the only beasts that bark. Seals bark. Skippy is, in fact, a seal.

Now, it is certainly true that seals are rarely, if ever, referred to as hounds or curs, and are not, in the general run of things, fed in the manner or with the fare Skippy enjoys. But Skippy is a particular kind of seal, known as a selkie or sealchie, that is, an allegedly mythical type of seal which, when on land, can shape-shift, and take on other forms, often human, but sometimes dog or cat or cow or, extraordinarily, wasp or hornet. The selkie is not, however, a wholly convincing shifter of shape, and whatever form it takes as it flops onto land from its watery domains, it always retains a recognisably pinnipedian character. If one were lazy, one might call Skippy half-seal, half-dog, but that is too simplistic and gives quite the wrong impression of his physical appearance. Depending upon the time of day, and the play of light, and the humidity of the air, Skippy can look almost exactly like a common seal, or a mastiff, or even, from some angles, like a giant ungainly sparrow.

Whatever form he takes, on land, he barks, and tends towards other dog-related behaviour, such as fetching thrown sticks, drooling, and, when allowed, leading blind people safely through the many imperilments of the cityscape. Though myopic, I am not blind myself, but I make a little bit of pin money by renting Skippy to sightless folk who require a canine guide for half an hour or so while their regular dog is meeting an appointment at the veterinary surgery, which happens to be bang next door.

It will be said that I must have known, when describing Skippy as a hound and a cur, that readers would leap to the conclusion that I was referring to a dog rather than to a selkie. After all, selkies are not the most common of beasts to keep as domestic pets. I grant that. In mitigation, all I can say is that, on the day I was writing about, the quality of the light seeping in through the windows, and more particularly through the bathroom window, a sort of milky, soapy, lucence, lent Skippy a dogginess such that even I could forget for a moment that he is in fact a seal.

More problematic is the matter of Skippy's diet. As far as I am aware, most seals like to eat fish, often swallowing them whole. Whether it be sprats or sardines or dabs, the average seal, and indeed the average selkie, if there is such a being, can happily eat nothing but fish throughout its life, a life, by the way, which an actuary would calculate at roughly twenty-five to thirty years. On the other side of me from the veterinary surgery there is a seal actuary's office, and I checked those figures with him, a few minutes ago, in a break between paragraphs. I did not mention it at the time, thinking it better to present the information at a pertinent point, rather than interrupting my flow to buttonhole you with a newly-discovered fact. Incidentally, Mr Ten Boom, the actuary, is blind, and he has a sickly guide dog often in need of stomach pumpings at the veterinary surgery, so I regularly hire Skippy out to him for little trips to the newsagent or the greengrocer, located as they are on the other side of a wide boulevard frantic with hurtling container lorries. Mrs Ten Boom, the actuary's wife, knitted a splendid little tabard for Skippy to wear on such excursions, yellow with black stripes, which can give him a disconcerting resemblance to an enormous bee.

But I must keep on track and return to the important matter of Skippy's diet. I recognise that, having described my pet selkie as a hound and a cur, and mentioned his barking, the clincher for Tim Thurn and other readers, leading them to assume I was writing about a dog, was the reference to a bowl of reconstituted meat chunks in jelly. After all, long years of experience tell us, whether we are dog owners or not, that such a meal is \emph{de rigueur} for our canine pals. I will not muddy the waters by pointing out that cats are commonly fed on broadly similar lines. I have not received any readers' letters asking me to clarify whether or not Skippy is of the feline persuasion.

As a selkie, one might expect Skippy to salivate happily at the sight of the aforementioned sprats and sardines and dabs, but not at food fit for a dog presented in a bowl set upon the floor. Many seals jump in the air to catch thrown fish, rather than snuffling with their faces buried in a bowl. Yet recall, one of the defining characteristics of the selkie is that, while in water it is wholly a seal, upon land it shifts shape and takes on, partially and spookily, other forms. We cannot expect a transformation, in certain lights, of its outward appearance to go unaccompanied by a corresponding terrestrial enjumblement of its innards. The innards of a seal or a selkie are not merely blubber, they are as complex and miraculous as the innards of many another organism. If you have ever dissected anything, be it a fruit bat or a buttercup, you will know whereof I speak. Thus, once having heaved itself ashore, and bid goodbye to the sea, either temporarily or permanently, the selkie's transmogrification, even if it is mythical, is startling. In Skippy's case, by becoming in some manner doggish, he discovered doggish appetites. We have already ascertained that Skippy enjoys chasing after thrown sticks. Why, then, should he not see the allure in a bowl of reconstituted meat chunks in jelly placed before him on the floor? That does not make him a dog. It makes him a selkie which, on land, in the play of light, appears to the human eye to be a dog, more or less, if one does not examine him too closely.

It will be asked whether a selkie chooses the terrestrial form it adopts, or whether, as it emerges from the sea, it is subject to forces both eerie and inexplicable, and takes the form destined for it by the seal-gods. I confess I do not know the answer to that question. Better minds than mine have wrestled with it, not least Mrs Ten Boom, the seal actuary's wife, the tabard knitter. As she knits, she devotes her powerful brain to all sorts of abstruse and thorny problems, regarding not merely seals and selkies but to anything that exercises her. By no means does she confine herself to the aquatic and amphibious. She is, it is said, one of the few people living who has read every word ever published by the out of print pamphleteer Dobson, and not just read them but annotated them. Unfortunately, though dozens of Dobsonists have beseeched her to make public her notes and marginalia, she refuses, point blank, often with the aid of a baseball bat. She is a dear old thing, a good neighbour and a tabard knitswoman of genius, but, just as there are religionists who claim to have a personal relationship with Jesus, Mrs Ten Boom insists upon an exclusive Vulcan mind-meld with the pamphleteer, and bashes senseless with her bat anyone who tries to broach it. For my part, I salute her, as does Skippy, who has recently devised a fantastic saluting gesture with his right flipper, which looks, in a certain cast of light, like a paw.