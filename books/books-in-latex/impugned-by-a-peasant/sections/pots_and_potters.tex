\chapter{Pots And Potters}

IT OCCURRED TO me the other day that I have neglected the subject of pots and potters. That is about to change, for I have been immersing myself in the study of the pots and potters of Sibodnedwab, and am so enthralled that I cannot actually think about anything else.

You are more likely to have heard of Sibodnedwab in the context of blebs and pustules than potters and pots. 'Sibodnedwab Pimple' is the common name for a particularly distressing dermatological condition, so called because the medico boffin who first identified it was, at the time, tackling an outbreak in and around that brickish township. It is a big and unsightly pimple, often found erupting on the forehead, where it throbs and glows, like a third eye. Many Sibodnedwab potters show the scars of past pimpledom, and thus are sometimes known as the Pimple Potters. Anglepoise refers to them as such in his magisterial but unpublished survey of their pots and fragments.

That so many Sibodnedwab potters' pots survive only as shards and fragments is due to the ferocious bombardment of the township by unhinged aggressors. Every day for the past forty years, the township has been subjected to attack. Untold tons of pebbles have been catapulted from outwith its walls, and every so often a flock of trained swallows flies over, dropping other pebbles. So many Sibodnedwab pots have been smashed that the historic township glue factory cannot manufacture enough glue to gum them all back together.

The pebbleers are motivated by fear of the Sibodnedwab Pimple, which they mistakenly believe is contagious. They are also said to hold that the pimple actually is a third eye, and that it makes visible rays and beams and lights of such horror that they drive men mad. This is of course a primitive superstition, but then the pebbleers are a primitive and stupid people, not one of whom has ever had the wit to fashion a pot. They cannot even knead dough with any finesse. The baps and buns they bake to peddle by the sides of major traffic routes are not of a consistency or munchiness to tempt the major traffic route user who has made the error of purchasing a bagful in the past.

Anglepoise rightly laments the fragmentary state of much Sibodnedwab pottery, but that does not stop him cataloguing it, nor singing the praises of the pimpled potters. He rescues from obscurity some of the key figures, among them Bink, Bunk, Snop, Tegg, Wimshurst, Gock, Flum, Higg, Bleg, Zont, De Havilland, Shud, Muff, Tung, Cuck, Weck, Bipp, Fung, Rack, Ick, Snit, Puck, Cherrybib, Belch, Cracker, Font, Flip, Sunk, Bark, Dodd, Wope, Jamm, Pulp, Cousins and Lamonto. Not all of them were literally pimpled. Dodd, for example, was born with a head upon which pimples and pustules and other blebs never grew. But then, he was something of a medical anomaly in other ways, ways that confounded his carers at the Home for Startling Young Potters where he spent his childhood. All that remains of Dodd's work is a single shard, but what a shard it is! Anglepoise gives it due attention, over twenty pages of his unpublished manuscript, and tries heroically to extrapolate from the shard a vision of the whole pot. It was, he concludes, the sort of pot you would have displayed on a mantelpiece rather than one you might have put to use for holding your grain or your millet. Struck by dozens of pebbles on a single day in the summer of 1969, it was smashed to smithereens. All but one of those smithereens was then lost in a flood.

There are similar stories of the fate of pots by other named potters, and of course there are many fragments, and indeed some surviving whole pots, whose potters' names we know not.

What is it about Sibodnedwab, we may ask, that makes it such fecund ground for potters? It is a township built upon clay, of course, but then so are many other townships, including the ones where many of the unhinged pebbleers were born and raised. The clay itself does not explain things. It is true that clay was mined there long before mines of any kind existed anywhere else, and the township fathers passed an ordinance that there must be at least two working kilns for every habitable dwelling. It is equally true that virtually every other activity in the plastic arts is forbidden. But these can be seen as a response to potmania rather than its source. Sensibly, Anglepoise does not get bogged down in sterile academic argument. His is a work of celebration, and cataloguing. I found it intriguing to learn, from a postscript, that he is blind, as were a number of the Sibodnedwab potters. Some, like Muff, Rack, and Cracker, were born blind, but others lost their sight when their eyes were put out by catapulted pebbles.

That postscript is at times quite moving, for in it Anglepoise describes the many hours he spent in an underground bunker where the shards and fragments are gathered, cutting his hands to ribbons while handling, as delicately as he could, the often jagged and razor sharp remnants of the potters' art. In the end he had to wear special mittens with sensory devices sewn into them. Now and again he would drop a fragment, resmashing it as he did so, when jumping out of his skin at a sudden onslaught of pebbles up above. The roof of the bunker was pocked and pitted with holes where one of the larger trained swallows had dropped a big pebble from a great height. Thankfully, netting strung up across the ceiling meant that few if any pebbles ever landed upon the floor of the bunker itself.

If anyone has the stomach for it, there is probably a monograph to be written on the unhinged and irrational pebbleers themselves. That a type of pimple, a pimple understood and neutralised by medical boffins, could provoke such violence, over such a period of time, is, on the face of it, extraordinary. Even when we take full measure of the pebbleers' stupidity, still we shake our heads in disbelief. There have been calls for brave potters to go incognito among the pebbleers to teach them how better to knead their dough, the thinking being that if they sold more baps and buns by the sides of major traffic routes their sense of grievance, and their fear of the 'third eye' pimple, would be lessened as they grew wealthier. Rare is the potter, though, willing to take the risk after what happened to Higg. Higg it was who volunteered when first the idea was mooted to go among the pebbleers freely offering to share his kneading skills. Like Dodd, he had never fallen victim to the Sibodnedwab Pimple, his forehead remaining as smooth and unblemished as a baby's. But his potters' jargon betrayed him to the cleverer pebbleers, and they broke him on a wheel.

Anglepoise is so busy extolling and cataloguing that he does not address the curious matter of the pebbleers' undoubted ability in wheel-building, catapult-making, and the training of swallows. For unfathomably stupid people prey to ridiculous pimple superstitions, in these three areas, and in the collection of pebbles, they show ingenuity and energy. Perhaps all that is needed is for an enlightened chieftain to emerge in their midst. But alas, the rollcall of their chieftains is a litany of thick-headed brutes and ignorant hotheads. That they can show such patience and tenderness when schooling their swallows makes it all the more perplexing.

There remains the mystery of why Anglepoise cannot find a publisher for his magnificent work. Let us leave aside the bigoted conglomerate which refuses to handle any work by blind authors, a decision based, it seems, solely on the poor sales of David Blunkett's memoirs. One would think a book that combines a catalogue of pots and pottery fragments, the potted biographies of numberless potters, some with pimples and some without, and an elegiac prose style, would be snapped up by the kinds of punters who buy their books along with sausages and eggs and washing up liquid at out of township supermarkets. As it is, the manuscript languishes, as it has done since its author perished on a blizzard-wracked mountainside, on a shelf in the deepest basement of the Sibodnedwab Pimple Puncturing \& Disinfecting \& Swabbing \& Sewing Up Treatment Centre, in the heart of the township. So subterranean is the basement that no pebbles ever penetrate its vaulted ceilings. Yet the shelf upon which the manuscript rests is rickety and worm-eaten and liable to collapse at any moment. If and when it does so, Anglepoise's work will topple to the earth, and be trodden upon, repeatedly, by the big stomping boots of the miners, who, even today, are still opening up new seams of clay to supply the potters. The endless onslaught of pebbles has failed to stop their potting, and even if their pots last but ten minutes before being smashed to bits, we can but praise their industry and tireless potmanship.