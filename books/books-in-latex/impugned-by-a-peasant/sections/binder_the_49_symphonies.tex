\chapter{Binder : The 49 Symphonies}

BINDER'S FIRST SYMPHONY is cranky. The second is based on the sound of winches. In the third symphony, Binder drags us through his mental muck. The fourth is of a piece with the first. The fifth, the majestic fifth, owes much to pomposity. The sixth is tart, and the seventh tarter still, while the eighth, the so-called 'sausage-shaped symphony', is radiant. For his ninth symphony, Binder donned a pair of muslin gloves. Dachshunds and mastiffs bark and howl in the tenth, owls hoot in the eleventh. The twelfth is sordid. The thirteenth symphony had its premiere in an abandoned aircraft hangar. 'A pimple burst,' said Chumpot, of number fourteen. He never heard the fifteenth, for Binder had him shot. Gunfire is heard at the beginning of the sixteenth. Symphony Number Seventeen ends with the cracking of a plank. Eighteen is hated, nineteen sounds like a forest in the rain, and by the time he wrote number twenty Binder was taking vitamin pills twice a day. The twenty-first symphony is usually played backwards. The twenty-second has to be heard through a hat with flaps. The twenty-third is obstinate, like a mule, or like the donkeys of Binder's twenty-fourth. Take a stick to the stalls for twenty-five, and a bucket to the circle for twenty-six. The twenty-seventh symphony can look after itself. Number twenty-eight uses a motif of milk. Twenty-nine is tarter even than seven. Thirty is so groovy you might die. When the idea for his thirty-first symphony popped into Binder's brain, he was aboard a great steamship. Its sinking is mourned in the thirty-second. Thirty-three was commissioned by Stalin. The score for Symphony Number Thirty-Four, the 'Symphony of Buttons', calls for several buttons and a hurdy-gurdy. Thirty-five is mostly silent, or at least so quiet one strains one's ears to hear a damned thing. Binder's favourite was his thirty-sixth. The thirty-seventh lacks elegance. Thirty-eight is gaudy. Thirty-nine is the music of champions. Forty falls flat. Forty-one comes from outer space. The opening bars of Binder's forty-second are used as the theme to a piece of TV tosh. Symphony forty-three has a certain relentless pigginess. Symphony forty-four is birdy, not piggy. Forty-five is just grating. Forty-six, Binder's longest symphony by far, frightens both birds and pigs. Forty-seven is played in a ditch. Forty-eight is all tinkly and twee. Contemplating the pippy splendour of his forty-ninth and final symphony, Binder was heard to remark that it reminded him of the architecture of the burning cities he had skulked in as an orphan child.