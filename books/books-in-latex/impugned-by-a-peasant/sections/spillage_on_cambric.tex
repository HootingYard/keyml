\chapter{Spillage On Cambric}

HAVING SPILLED SOUP on a piece of cambric, the verger tried to amend his sloppiness by dabbing at the cambric with a damp sponge. Alas, so much chemical colourant had been added to the soup, which was of a tomato flavour, and so hastily and violently did the verger do his dabbing, that a bright orange stain was impressed into the cambric. The cambric, by the way, was blue with golden stars, like the vault of heaven. Fearing that he had ineradicably besmirched a representation of the ethereal realm, the verger hid the cambric and the sponge in a cupboard and poured the remainder of his tomato soup down the drain. He cleansed his bowl and spoon with more care than he had dabbed at the cambric, and placed them exactly where he had found them among the crockery and cutlery, having first dried them with a tea towel depicting the martyrdom of Saint Anselm. This was an historically inaccurate tea towel, as Saint Anselm died a natural death rather than being martyred for his faith. It was not the only erroneous tea towel in the kitchen.

The verger hoped that hiding the evidence of his sloppiness in the cupboard would prevent it from coming to light, but he reckoned without the involvement of Detective Captain Cargpan. The detective was called in by the bishop on an unrelated matter, something to do with the local sniper, who had been taking potshots at the cathedral hens. Cargpan was noted for his energetic approach to police work, and on this occasion he strained so many sinews that by midmorning he was exhausted and dehydrated. Characteristically, he did not whimper to the bishop begging for refreshments, but instead blundered about until he found the kitchen, where he intended to gulp down water straight from the tap. Having done enough gulping to make himself feel human again, Cargpan could not resist opening all the drawers and cupboards in the kitchen and examining the contents with his magnifying glass.. Such was his method. Although it was unlikely that either the sniper or the hens had ever been in the kitchen, the detective assumed nothing. Thus it was that he discovered the hidden cambric and sponge. He was extremely suspicious of the bright orange stain.

Under questioning, the verger admitted his part, but insisted that the sponge and the cambric had no connection to the sniper and the hens. Detective Captain Cargpan roughed him up a bit, breaking one of his arms and dislocating his jaw. This, too, was his method. The verger continued to protest his innocence throughout his subsequent trial and the long years on the prison hulk moored off an unspeakable stretch of coastline. Long after his death, campaigners sought for him a posthumous pardon. But we now know that Cargpan was right all along, and that the verger and the sniper were one and the same. Interviewed for a television documentary during his long and happy retirement, Cargpan explained that, for him, it was an open and shut case.

'A man who can smear tomato soup upon a picture of the vault of heaven, and who makes use of historically erroneous tea towels, is precisely the sort of man who would shoot at innocent clucking hens with a Mannlicher-Carcano rifle,' he said. Today we are at last able to acknowledge the wisdom of those words.