\chapter{Two Jaunts With Uncle Lars}

IT IS A frozen place, where I come from, and very far from the sea. The first word I ever spoke was 'icicle', and I was in my late twenties before I ever heard talk of tugboats and barnacles and offshore gas fields. The idea that solid ice could simply melt away was so foreign to me that when I first saw it happen it really fried my wig, Daddy-o, as the hepcats would say. Not being a hepcat, I screamed and swooned.

My ghoulish Uncle Lars grabbed me by the mitten one day and dragged me off with him on one of his jaunts. We wore snow-shoes to negotiate our way across the freezing frozenness. Resting awhile in the shadow of an immense ice mountain, Uncle Lars clamped his pipe in his jaws and took from a pocket of his enswaddling furry wrappings a box of matches. After lighting his pipe, he held the still-lit lucifer against a crag of ice and I watched as it melted and dripped and vanished away, as if it had never been there at all. It was as if the world I had grown to understand had no underpinnings, was mere figment, and so my brain collapsed and I screamed and swooned.

So severe was my trauma that I was chained up in what we called a 'mad cabin' for months on end. My recuperation was slow, but I gradually began to understand the concept that ice and water and steam were but different forms of the same substance. I cannot overestimate the importance to my recovery of a pamphlet I was given on the day when one of my chains was removed. It was called \emph{Child, Be Thunderstruck As Your Tiny Brain Copes With The Notion That Ice And Water And Steam Are But Different Forms Of The Same Substance!} The author's name, I learned, was Dobson. Sadly, the pamphlet has long been out of print.

Such was my first encounter with the twentieth century's titanic pamphleteer, an encounter which led from initial enthusiasm to wild overexcitement to monomania. I became so demented about Dobson that I risked being kept in the mad cabin for years and years. Fortunately, on a visit one day, Uncle Lars taught me to hide my light under a bushel, not literally of course, for that would have been a very foolish thing to do and despite his clumping weirdness, Lars was no fool. But I learned to temper my Dobson-zest when the warders were lurking, and went so far as occasionally asking to take delivery of works by other writers, such as Zadie Smith and Colm T\'{o}ibin. Needless to say, I never actually read such unDobsonist trash, but made use of the books as pamphlet-camouflage or as handy things to chuck at the wall with my free hand. Chucking things at the wall was my other great leisure activity in those days, and remains so. It is a great pity that Dobson had so little to say on the subject.

And yet there were so many, many topics to which the pamphleteer turned his attention. I found that, as I worked my way through the canon, I became obsessively interested in whatever Dobson was writing about, to the exclusion of anything else, even of the subject of the pamphlet I had been reading the day before. That being so, I often wonder how different my life might have been if, on the day I was eventually unchained and ejected from the mad cabin into the frozen wastes of my homeland, I had been reading something other than Dobson's short, strange, brilliant pamphlet \emph{Why Those Let Loose From Mad Cabins Should Immediately Up Sticks And Settle At A Seaside Resort}.

Before I upped sticks and settled at a seaside resort, I said farewell to Uncle Lars. For old time's sake, we went on a jaunt. He was more ghoulish than ever, and had exchanged his pipe for some sort of newfangled smoking contraption into which he crammed fistfuls of disgusting blackened vegetable matter and sent out blooming coils of miasmic fug. We stopped again beneath the great ice mountain, and Uncle Lars again struck a match for his smoke, and again he held the match against the ice and I watched it melt away. But I neither screamed nor swooned, for I had read my Dobson, and I knew what was afoot. Uncle Lars knew that I knew, and he flashed me a conspiratorial grin. For an instant I thought I might scream at that, for the Grin of Lars, seldom seen, is never forgotten, and has sent many a poor gibbering grinee to the mad cabins. I quailed at the sight of it, certainly, but it did not utterly undo me, not only because I had seen it once before, and was thus inoculated against it, but also because yet again I could call on Dobson, having read his pamphlet on terrifying facial expressions. I grinned back at Lars, as best I could, knowing that I might never see him again, and he puffed the match out and handed it to me, as a memento.

Look, there, on the mantelpiece of my seaside chalet. Between the toy binnacle and the heap of sand, you see that half-burned match? That is the match that was my parting gift from Uncle Lars. Sometimes I put it in my pocket, and I go down to the promenade, and I lean upon the railings and stare out to sea. As I stare I hold the match delicately in my fingers, and the whole world makes sense. I know that all the water I can see was once ice, until it was made hot by untold billions of matches lit and aflame, whereupon it became the sea. And the sea too will vanish, it will boil and seethe and become vapour, just as Dobson foretold.