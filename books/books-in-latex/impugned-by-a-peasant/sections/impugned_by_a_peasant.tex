\chapter{Impugned By A Peasant}

I WAS IMPUGNED by a peasant. It was a Thursday afternoon and I was walking along a lane, between aspens and larches. I saw the peasant up ahead. He was leaning against a stile and as I got closer I saw he was idly swinging a flail to no great purpose. As I passed him, he impugned me, in some sort of rustic invective I barely understood. I would have dashed him to the ground with a single blow, but alas!, I am a milksop and a weakling and I merely passed on by along the lane, blushing and furious.

Later, as I sat in a countryside canteen drinking a tumbler of Squelcho!, I reflected upon this peasant and his impugning. What was he doing, leaning against that stile? Why was he swinging a flail? In what brutish argot did he speak? Much to my disgust, I realised I was obsessed by him, as, in \emph{Death In Venice}, Gustav von Aschenbach is obsessed by Tadzio, or in \emph{Love And Death On Long Island}, Giles De'Ath is obsessed by Ronnie Bostock. But Tadzio and Ronnie are young and beautiful, whereas my peasant - \emph{my} peasant! - was old and snaggle-toothed and filthy and wretched. My hands were shaking, and I slopped some of my Squelcho! on the canteen table, drowning a fly.

As I returned along the lane, I adjusted the cravat around my neck, to give it a more rakish look, and I primped my bouffant, and I modified my trudge to a sort of flouncing prance. As I neared the bend in the lane beyond which the stile would come into view, my heart began to thump violently and my mouth became so dry I gasped. Would my peasant still be there? Would he impugn me again? I wanted to run back to the safety of the canteen, but at the same time I was desperate to see him once more, so filthy, so rustic, so ancient, so vile!

How can I express the sickening sensation I felt as I rounded the bend and saw that my peasant was gone? It was as if a knot of vipers writhed within my guts. Sunlight dappled through the aspens and the larches, a breeze refreshed the air, and there was the stile\ldots  but leaning on it now were two impossibly attractive youngsters, playing conkers. Closing in on them, panting like a monster of depravity, I saw they wore name-badges. One was Tadzio, the other Ronnie. I was barely coherent as I babbled at them, asking if they had seen a peasant, an old filthy snaggle-toothed peasant with a flail, had they seen in which direction he had gone, and when, and was he going fast or slow, with purpose or without, and did the sunlight glisten on his greasy matted hair?

First Tadzio, then Ronnie, impugned me. In particular, they impugned my cravat and my bouffant and my flouncing. I crumpled to the ground, weeping and neursathenic. I would have welcomed death, there and then. But of course, I did not die. An hour or two later, I got to my feet and dusted the muck of the lane from my Italianate suit. The sun was sinking in the west, and Tadzio and Ronnie were long gone. I picked up a pebble and chucked it inexpertly at a linnet perched in an aspen. I missed the bird, of course, and I pranced away from the stile and made my way home.

Years later, looking back on that afternoon, I can no longer picture the name-tagged youths, but the vision of the peasant is as clear to me as if he were sat here opposite me. I do not have him, of course, but I have his simulacrum, posed in the armchair, built of cardboard and wire and wool, with piano keys for his teeth and a light dusting of authentic countryside muck, and when I activate the console he impugns me in that mechanical, guttural, rustic invective I had a character actor record for me, and which, still, still, I barely understand.