\chapter{The Puckington Tunnels}

IT WAS A big fort, with delightful crenellations, and many flags, and it had the shiniest portcullis outside of Navarre. This was Fort Hoity, sister fort of Fort Toity, and an extremely interesting fort in its own right. For underneath Fort Hoity ran the Puckington Tunnels, those tunnels you may have come across in your reading, if, that is, you have been reading about tunnelling systems as a change from your usual diet of chicklit, gitlit, and zadiesmithlit.

There is a regrettable temptation to neglect the literature of tunnels and to be sidetracked by less meaty subject matter, by ephemera and winsomeness and the outpourings of knaves. I am not immune to such distractions myself, and in truth I ought to have done a lot more tunnely reading than I have, especially once I put my mind to writing about the Puckington Tunnels. There are huge chasms in my knowledge, and if I faced a quiz on the subject I suspect my score would be embarrassing. Perhaps not so bad as that of clueless David Lammy - unbelievably, the government minister for Higher Education at the time - whose appearance on the television show \emph{Mastermind} elicited such delights as his belief that Henry VII succeeded to the throne after Henry VIII, and that the surname of the Nobel prizewinning scientists Pierre and Marie was Antoinette. The nitwit was not asked any questions about tunnels, but we may safely assume he would have fluffed them.

Speaking of fluff, there is a surprising amount of it in certain sections of the Puckington Tunnels. Layers, or perhaps clouds, of dust would be explicable, but it is difficult to account for the incredible fluffiness to be found underneath Fort Hoity. After all, there is not a speck of fluff in either Fort Hoity itself or in Fort Toity, and though both forts contain their fair share of dust and orts and scum and grease-stains, all fluffiness has been eradicated, forever and ever, yea, e'en unto the Last Trump, by the installation of modern fluff obliteration technology developed by the computer giant Macrohard. Yet take the staircase down from the Fort Hoity broom cupboard and enter the Puckington Tunnels, take a left and a right and a second left, and you will be in the section of tunnel dubbed the Fluffy Zone by those in the know. There are spits and spots of fluff elsewhere in the system, but in this part it is quite simply overwhelming. Nobody knows why.

Nor does anybody know why the tunnels were dug in the first place. We know who dug them, because they are named after their digger, Puckington, the so-called 'human mole', and we know when they were dug, for every time he turned a corner or began a new stretch or created a tributary tunnel, Puckington stuck pins in a panel to form the numbers of the date and hammered the panel with nails to the tunnel wall at head height, head height for Puckington being considerably higher than for most men, for he was eerily tall, and thus all the tunnel junctions are unexpectedly cavernous, quite unlike the tunnels themselves, through which Puckington himself could only move when stooped, or by crawling upon his belly like a creeping thing as mentioned in the Bible, a pocket-sized edition of which, in the Huckabee Version, he carried in his pocket wherever he went a-digging, as a sort of charm or talisman which he insisted protected him from tunnel collapses and subterranean mudslides. Obviously there must have been one occasion when he went out with his spade and his jackhammer and his crate of dynamite and his pickaxe and his other tunnelling paraphernalia but forgot to tuck the Bible into his pocket, for on a very rainy Thursday Puckington perished, buried under a ton or two of soaking wet soil the weight of which proved too much for the wooden props with which he had shored it up in the tunnel he was digging that day, a brand new tunnel far away from the tunnels he had dug under Fort Hoity and which bear his name still and attract many a tourist and many a weekend troglodyte.

It was as a sightseer with a bent for the loveliness of crenellations that I discovered the Puckington Tunnels. I came to Fort Hoity to see the fort, as did all those in my coach party. We were a gang of fort-freaks. It happened that I became detached from my pals when, straggling at the back of the group padding through the famed Fort Hoity corridor of cupboards, I stopped to buy a carton of yoghurt from the yoghurt cupboard person. So delicious was the yoghurt that I spooned all of it into my mouth there and then, only to find that the group had gone ahead without me and I was all alone. I blundered into the broom cupboard and followed the staircase down and thus found myself at the entrance to the tunnels. I was awestruck, as who would not be? At the time I jumped to the rash conclusion that the tunnels led directly underground from Fort Hoity to Fort Toity where, I supposed, a second staircase would take me up to the sister fort's majestic pantry, an architectural wonder of the pantry and larder world if ever there was one.

I did not know, then, that the Puckington Tunnels were the work of a madman, dug without purpose, or direction, or sense. I did not know, then, that the tunnels led nowhere, that all their twists and turns and rises and plunges ended, if they ended at all, tapering ever narrower, in blockages of black adamantine stone. I did not know that Puckington had, in spite of their apparent chaos, designed his tunnels with a lunatic genius for precision, such that he, and only he, could ever find the way out. These are dark tunnels, these Puckington Tunnels, and I have dwelt within them, since snacking on that carton of yoghurt, for over a hundred years.