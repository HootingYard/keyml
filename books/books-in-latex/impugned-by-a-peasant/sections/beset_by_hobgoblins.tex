\chapter{Beset By Hobgoblins}

BE IT KNOWN that I have, for the past week or so, been beset by hobgoblins that scampered out of the wainscot. They may well have been Vatican hobgoblins, for one of them let slip, while besetting me, that the cashpoint machine or ATM from which it had that morning withdrawn funds displayed its instructions in Latin. While I was aware that the Vatican City is the only state, or statelet, to have programmed its cashpoints thus, I had no idea that hobgoblins had need of banknotes, or indeed of money in any form. So intent was this particular hobgoblin on besetting me, however, that it took not a jot of notice of my politely shrieked queries regarding its financial affairs, so I am none the wiser.

Another thing I did not know was that for decades there had been hobgoblins skulking behind the wainscot, biding their time. This was a shocking revelation. So quietly had they been preparing to beset me, over all those years, that I did not have an inkling of their existence. Nor had they betrayed their presence when I was doing the odd bit of wainscot-related do-it-yourself handiwork in the dying days of the John Major government. Armed with a secondhand Barry Bucknell book, I had been keen to cut my chops on a simple project, though I must confess that I abandoned the work before the wainscot was fully rejigged, or whatever it was I was doing to it, and I have more or less neglected to follow up with any other home improvement jobs in the succeeding years. There have been other calls upon my time, which I will not go into here, except to point out, because it is pertinent, that during the first years of the Blair administration I learned much about he-man wrestling holds, though as an observer rather than as a practitioner. I had little opportunity to watch actual wrestling bouts where real wrestlers demonstrated the holds I was learning about, and I relied for the most part on black-and-white diagrams sprinkled throughout a \emph{Teach Yourself He-Man Wrestling Holds} book I had acquired at a rummage sale.

So when the hobgoblins came scampering from the wainscot, I thought to effect a citizen's arrest - or a series of citizen's arrests - by using the techniques I had taught myself through painstaking study of he-man wrestling hold diagrams, often by candlelight on stormy winter nights. Alas, through lack of practice I had grown rusty. This was stuff I had learned when Frank Dobson was the Health Secretary, it was that long ago. I was quite unable to gain any purchase on the limbs or necks of the besetting hobgoblins, and they slithered and squirmed from my grasp with quicksilver ease.

I wondered if I might persuade them to desist by poking them with a fork. Somewhere I had read that this was an effective deterrent with other types of goblin, so it seemed a reasonable assumption that it would work with hobgoblins, even ones which originated in the Vatican, if that was indeed the case. As I dashed into the kitchenette flapping my hands at my besetters, I recalled that the article I had read - in a magazine devoted either to forks or to goblins - recommended first poking one's fork into a pickled onion. Thus would the tines of the fork be coated with anti-goblin juices, including pickling brine. Now as it happened, only the day before, in a fit of peckishness, I had gorged myself stupid on pickled onions, and on other pickled items, and my cupboards were bare, at least in that foodstuff subsection. Poking my fork into a fairy cake or a munchy reconstituted fish slice would not be remotely helpful, nor, I surmised, would steeping the tines of the fork in a jug of goat's milk. Nevertheless, I opened my cutlery drawer and took out one of my treasured Margrave of Hohenhollernbadgasgothengraff forks, from the dinner set presented to me all those years ago, around the time of Harold Wilson's shock resignation, by the Margrave's very own great-great-great-granddaughter. Her name, I recall, was Googie, the only Googie I have ever come across save for the esteemed star of stage and screen Googie Withers (born 1917). It is as fine a set of cutlery as human ingenuity has ever fashioned, and I felt that the intrinsic quality of the fork could outweigh the absence of pickled onion residue when it came time to poke it into one hobgoblin after another. That, of course, was the task now before me, and I had no idea it would prove impossible. If you have ever tried to poke a hobgoblin with a fork, you will know why.

Later, as I sat hopeless and forlorn in my armchair dangling a bent Margrave of Hohenhollernbadgasgothengraff fork from my quivering hand, still beset by hobgoblins seemingly more energetic than before, I reflected upon the manifold miseries of existence in this vale of tears. I have only just recovered enough of my wits to write about those terrible days. This morning, for reasons I cannot explain, each and every hobgoblin suddenly ceased besetting me and scuttled back behind the wainscot. Perhaps they had run out of cash and had to go to their infernal Latin cashpoint machine to replenish their hobgobliny wallets. But that would hardly take all day, and they show no sign of reappearing. I know, though, that they are lurking there, silent and still, biding their time, as hobgoblins do.