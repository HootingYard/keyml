\chapter{The Bell And The Toads Etcetera}

IN ONE CORNER of the room a huge church bell lies on its side. It is battered and dented but its clapper remains intact.

Next to the bell is a crate of toads. The toads have been counted, and re-counted, at least three times, by the toad counting person, whose coat and cap are hung on a hook on the back wall between the bell and the crate. The toad counting person himself is no longer present. We are to understand, later, that he has been called away to an urgent toad count elsewhere, one for which he is permitted to dispense with his coat and cap, hence their presence on the hook. The toads have all been injected with a narcotic drug. They are still. Some of them are toxic toads, but it is not immediately apparent which ones, and this will prove significant, later on.

Alongside the crate of toads is an occasional table on which has been placed a fiendishly embroidered tablecloth. It is a bit tatty around the edges, which overhang the table almost to the floor. The tattiness, we will learn, is the result of it having been gnawed by wild beasts. Arrayed atop the table, upon the cloth, are a whisk and a jar of unidentified paste and a Bible and a squirting utensil moulded from bright plastic. Later, we will learn that an accompanying funnel has been stolen from this tabletop arrangement, by person or persons unknown, as the legal parlance has it. Detectives will become involved.

Then there is a gap. There is a trapdoor in the gap, but it has been secured with fastening pins to prevent accidents.

Beyond the gap is a sofa, plumply upholstered. BAXTER is reclining on the sofa, wearing a cardigan and slacks and plimsolls. He is smoking a pipe, and, between puffs, is whistling \emph{Oh Danny Boy}. He is an inexpert whistler, and the sound he makes is grating upon the ears. His hair is absolutely caked with brilliantine.

Behind the sofa looms a piece of classical statuary. It is a representation of a generic Greek or Roman God. Later, there will be a brouhaha over the identity of the God, which will remain unresolved, even after blood has been shed and a terrible vendetta sparked. The head of the God is out of proportion to the body, and the legs are ill-made and of a Wordsworthian lack of ornament, although they are for the most part hidden by the sofa.

Past the sofa there is another toad, this time solo and uncrated, although like the other toads it has been anaesthetised and is still. This toad may have escaped the earlier toad counting, and it may be toxic. Or the converse may be the case, on both counts, that is, the counting and the toxicity. The toad may have been counted, and it may be a non-toxic toad. BAXTER will be compelled to address these issues later, leading to the irretrievable loss of his wits.

Finally, at the opposite end from the battered and dented bell, there is an iron spigot. By dint of a faulty washer it is leaking, and drops of water are falling into a pan placed at its foot. It is quite a big pan, tin, and pristine, as if it has come directly from the manufacturer's production line. Lights are cleverly directed to shine upon it, making it gleam brightly, almost as brightly as the sun.

Enter, stage right, THE ANTI-BAXTER

These are the opening stage directions for Istvan Scrimgeour's drawing-room tragicomedy \emph{The Bell And The Toads, Etcetera}. Its 1951 production at the Festival of Britain closed twenty minutes in to Act I. It has never been revived.