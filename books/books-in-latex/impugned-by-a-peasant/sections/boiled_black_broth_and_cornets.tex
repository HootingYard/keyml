\chapter{Boiled Black Broth And Cornets}

I PAID A visit to my friend Becke Beiderbix in her fortress in the mountains. We had known each other since childhood, growing up on a postwar housing estate, a workaday world of compactness and convention. But Becke was always a singleminded girl who followed her own strange star, and while the rest of us went off to polytechnics and office jobs and became fodder for a peculiarly dull-witted type of English fiction, Becke decamped to the mountains and built herself a fortress with her bare hands. I had no idea where she had picked up the skills to do this, and in truth, when I visited I was astonished to find how solid and immense and impregnable her fortress appeared, a massive edifice perched upon a bluff, as forbidding in its aspect as the Schloss Adler in \emph{Where Eagles Dare} (Brian G Hutton, 1968), but without the Nazi connotations, for Becke was the most apolitical person I have ever known.

When she greeted me at the gate, she was holding a cornet in her hand.

'Hello, Dennis,' she said, planting a peck on my cheek, 'As you can see I have taken up the cornet, like my near-namesake Bix Beiderbecke, the original young man with a horn, and perhaps the greatest jazzman of the nineteen-twenties.'

'From fortress-building to cornet-playing, you never cease to amaze me, Becke,' I replied, dumping my weekend luggage in a corner of the grim brickish vestibule.

'As you are well aware, I follow my own strange star,' she said, steering me into the canteen of the fortress where she ladled soup out of a tureen into a pair of bowls.

'This is my own home-made soup,' she announced, 'For in addition to building the fortress and learning the cornet I have taken a correspondence course in devising original soup recipes. In your bowl you have what I dubbed Becke Beiderbix's Boiled Black Broth, in which every single ingredient begins with the letter B. As you can see, it is a black soup, of a black so black that if you stare at it, instead of spooning it into your mouth, you will become entranced, pretty much like a voodoo zombie-person, and be entirely within my power.'

'Then I shall shut my eyes while I drain the bowl, Becke,' I said.

'Yes, I was about to recommend you do just that, Dennis,' she replied.

The soup proved to be bland and without even a hint of taste, but it warmed my innards and stopped the gurgling in my belly.

'Now that your belly has stopped gurgling, Dennis, I shall take you to see my workshop,' said Becke, and I followed her into the bowels of the fortress, to a room with a thousand padlocks and reinforced walls and sputtering candles. I half-expected to see a gibbering hunchback named Mungo, but it seemed Becke worked without assistance.

'Well now,' I said, 'You have many towering piles of metal tapping machine directories from all around the world, much thumbed through and dog-eared, as if you have been poring over them with terrific diligence, Becke'.

'That I have, Dennis,' she replied, 'It is drudgery to be sure, but necessary to the success of my project.'

Of course, I asked her what the project was, and her reply shocked me to the marrow. For all that her star was a strange one, it had never occurred to me that Becke was capable of the abduction and incarceration, in dungeons beneath her fortress, of eight completely innocent souls. She had gone through those directories searching for names, and when she alit upon an apt name she tracked the person down, wheresoever they might be, and she crept up on them and shoved a rag soaked in chloroform over their breathing channels, and shoved them into the back of her van, and drove like the devil himself at tiptop speed until back in her mountain fastness, and then she dragged the abductee down into one of her dungeons and slammed the heavy iron door shut upon them, and every day thereafter she took them a bowl of her black, black soup, and made them stare into its blackness until it was lukewarm, so they were pretty much like voodoo zombie-persons, entirely within her power, and then she commanded them to drink the soup, until the gurgling in their bellies ceased.

'But why, Becke, why?' I shrieked, as if taking part in a melodrama, wondering how the sensible, resourceful woman I had known had become quite loopy.

'Oh, this is only part one of the plan, Dennis,' she said, 'It will all make perfect sense now that I have an abductee in each of my eight dungeons. You would not believe how long it has taken me to work my way through those confounded directories to find the names I need. And then of course to travel hither and yon to wherever they are and do the bit with the chloroform, which has its own risks. You gape at me goggle-eyed, Dennis, as if I have taken leave of my senses, and I would agree with you were it not that all this is merely a preparation for a grander scheme.'

I did not discover, over that weekend, what the grander scheme was. Becke showed me a few other things in her workshop, including some mysterious small trunks, then insisted that we head on up to her rooftop pingpong area and play pingpong for hours and hours. Every so often she took a break to visit the dungeons, and left me to lie on my back, exhausted, staring at the bitter sky, trying not to think about what in heaven's name was going on far below in the subterranean depths of the fortress.

I made my farewell on the Sunday evening, after being given a bowl of a different home-made soup which I could sup without shutting my eyes. It was as bland as the black zombie soup, but extremely welcome after all that pingpong. Becke waved at me as I trudged down the mountainside towards the bus stop. I looked back, and there she stood, at her fortress gate, and above her in the now darkening sky shone a single star. I couldn't help but smile. She may have become bonkers, but she would always be my pal.

A year or so passed. I was too busy with my halibut research to give much thought to Becke and her eight abductees. I sent her the occasional metal tapping machine message, to which she always replied, although she never said much about what she was up to, confining herself to remarks about general fortress maintenance. And then one day, passing through Pointy Town, some kind of woolly-hatted student in need of pin money handed me a leaflet. I shoved it into my pocket and forgot about it, and only later, as I was rummaging through my jacket for scrunched-up halibut research notes, did I come upon it and read it.

\emph{Pointy Town Hepcat Jazz Club,} it said, \emph{is pleased to announce a concert by a thrilling new combo. For the past year, Becke Beiderbix has been teaching the cornet to an octet of eight amateurs, and she is now ready to lead them in what promises to be a fantastic debut. The Becke Beiderbix Bix Beiderbecke Tribute Cornet Octet, featuring newcomers Bixder Beibecke, Beike Bixderbec, Kebec Bixderbei, Bixbec Beiderke, Beibix Becderke, Derke Bixbecbei, Kebeider Bixbec, and Bixke Derbeibec will perform a show of Bix Beiderbecke classics. Soup will be served, in the form of Becke Beiderbix's Boiled Black Broth. Admission free.}


I attended the show, of course, but shut my eyes for the soup.