\chapter{Lord Love A Duck}

WHEN WE CONSIDER the relationship between God and humankind, we tend to think of God as the one who issues commands and decrees and ukases that mere mortals must obey. Occasionally, however, it is the other way about. I have in mind the Cockney cheeky chappie who will, from time to time, exclaim 'Lord, love a duck!'

What are we to make of this? Is our loveable scalliwag telling the Lord to bestow His ineffable benificence upon a denizen of the local duckpond? Or is it the case, as I prefer to think, of a command to God to engage in sexual congress with a duck? After all, there seems little need to be telling God to direct His abounding love upon any particular one of His creatures, for that is what He is doing all the time, apart of course from when He is smiting the sinful. It is a rare thing for a duck to require smiting, for by and large ducks do not sin.

We must ask why a chirpy eastender would command God to have sex with a duck, and the answer must be in the hope that the duck falls pregnant. For of course, a duck into whose womb wiggles a divine seed will eventually lay an egg from which will hatch, not an ordinary duckling, but a being that is half duck, half God - a duck-god, if you will.

The sexual link between Gods and aquatic birdlife is not without precedent. The most famous example is probably the story of Leda and the swan, although there the waters are muddied somewhat by the fact that God, in the form of Zeus, inhabited the body of a swan and proceeded to rape Leda, the mother of Helen of Troy, Clytemnestra, Castor and Pollux. None of Leda's children, either by the swan-God or by her husband King Tyndareus, turned out to have aquatic avian characteristics.

On one of his infrequent visits to Cockney haunts, Dobson overheard many ragamuffins and urchins shouting 'Lord, love a duck!', and he was led to wondering just how many duck-gods may have been spawned and were perhaps plashing unremarked in the ponds of the city's parks. Armed with a notebook and pencil, and some sort of pneumatic scanner device of his own invention, the out of print pamphleteer plodded around those very ponds during a wet October weekend. Sadly, he never wrote up his findings in pamphlet form, and the only record we have of his researches is a fragment from a letter Marigold Chew wrote to her cousin Basil.

\emph{Dobson has returned from his tour of east end ponds}, she reported, \emph{and appears to be convinced that a wigeon (or baldpate) he spotted plashing in a pond in [illegible] had a spark of divinity about it. I argued that a mere spark was surely insufficient, and that a true duck-god would be immediately recognisable as such, for it would probably emit a blinding efflorescence of heavenly majesty and be surrounded by duckling apostles bowed in worship of its mighty duck-god omnipotence and of its boundless love and mercy. I added, perhaps unkindly, that Dobson's ornithological ignorance was of such an unfathomable depth that it would not surprise me if he had mistaken a wigeon for a pigeon, and, the latter not being a duck at all, his whole theory would come crashing around his ears. He took umbrage at this, and retired to his escritoire to scribble some twaddle about another topic entirely.}