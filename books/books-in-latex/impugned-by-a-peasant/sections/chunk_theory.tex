\chapter{Chunk Theory}

CHUNK THEORY IS the theory that everything can be crumbled into chunks the better to apprehend its meaning. Chunks are very different to, say, lumps or clods or crumbs or bits, and must certainly never be mistaken for smithereens. The proper definition of the chunk is given in the standard work on Chunk Theory, \emph{Chunk Theory, A Primer}, by the Theory's original theoriser, Gustav Chunk. Chunk was, of course, not the name he was born with, it is a pseudonym he hit upon the better to identify himself with the Theory he propounded. When he was not writing about chunks, Chunk busied himself destroying all trace of his true surname, so successfully that today we know him only as Chunk and do not have an atom of evidence regarding his real name. There is even some doubt as to whether or not his first name was Gustav but, as the parlance of today has it, let's not even go there. Chunk was fond of demonstrating the beauty of his Theory by physically crumbling things into chunks, even things that do not readily lend themselves to crumblement. Faced with such a thing - a public telephone kiosk, for example - Chunk had no qualms about deploying hacksaws, axes, large heavy hammers and similar tools in order to effect his goal. He could regularly be seen marching about his town smashing things to bits, although when challenged he would protest that he was engaged in the 'assisted crumbling of things into chunks'. And boy oh boy was he challenged! He made innumerable complaints that he was being followed about by municipal cohesion officials and coppers, arguing that such stalking was a form of entrapment. Much of the \emph{Primer} was written in the waiting rooms of various courts and assizes where Chunk was due to face sanction. He was the kind of man who tended to topple over when shoved with sufficient force, and this led him to refine his Theory in later years. Shoving and toppling were to be incorporated alongside crumblement and chunks without doing damage to the premisses of the original Theory, save for the sort of collateral damage one might expect. How it would all have held together is something we can only guess at, for the promised second edition of the \emph{Primer} never appeared. Chunk himself was cagey whenever he was asked about it, which was seldom, as very few people - very, very few people - were remotely interested in his work. One Chunkist commentator claimed that only four copies of the \emph{Primer} were ever sold, each of them to cronies with whom Chunk used to hang around in the streets of his town, laden with axes and hammers and slicers, eyeing up likely targets for assisted crumbling. But there are other Chunkist commentators who hold radically different views. Indeed, the most intriguing feature of the whole business is that Chunkist commentators, disputants, devotees, fellow-travellers and hangers-on vastly outnumber the total number of people who have ever even laid eyes on a copy of the \emph{Primer}, let alone read the damned thing. This signal fact has led to an offshoot of Chunk Theory known as Chunk Theory Theory, a field which has spawned an entire academic industry populated by beardy good-for-nothings who would be better employed digging drainage ditches, some of whom, apparently, actually do such drainage ditch digging, in between penning abstruse articles for the numberless Chunk Theory Theory journals.

I went to interview one such peasant scholar for this piece, but when I approached him, he shoved me with considerable force, and I toppled into the drainage ditch he had just completed digging. Such are the perils of Academe. I brush them aside, for I am both hoity and toity and I know where the bodies are buried. My father was a gravedigger, and his father before him, and they knew not only where the bodies were buried but on which side their bread was buttered, and they knew their onions too. You will rarely find a peasant scholar, Chunkist or otherwise, who knows such things. I was minded to follow the family gravedigging tradition, but I proved too weedy to handle a spade. Ironic, I suppose, that I end up flailing helplessly on my back at the bottom of a drainage ditch while a Chunkist starts to shovel chunks of crumbled earth over me, like a scene from a cheap horror film. But I shall abide, though I crumble to dust, dust in crumbled chunks, and in each crumbled chunk a worm that burrows.