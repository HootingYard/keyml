\chapter{Hoistings}

I HAVE BEEN thinking a lot recently about hoistings. Well, when I say recently, I mean this afternoon, and when I say hoistings I mean more particularly Dobson's hoisting and Blodgett's hoistings. 

The pamphleteer, as is well known, was once hoist by his own petard, an incident which is the subject of an illuminating essay by Aloysius Nestingbird. Blodgett, on the other hand, was repeatedly hoist by a variety of petards, none of which he could call his own. This says much, I think, about the difference between Dobson and Blodgett, not just in terms of hoistings and petards but in all sorts of other ways. 

If only I could be as illuminating as Nestingbird! But alas, you will have to be content with something which, if it can be said to illumine at all, is the prose equivalent of a sputtering Toc H lamp hanging on a hook in an immensity of darkness, compared to the incandescence which Nestingbird sheds whenever he puts pen to paper.

Indeed, I am so cowed by the sheer damned splendour of Nestingbird's essay that I have considered abandoning this puny attempt to address those hoistings myself. And abandon it I would, as decisively as a decisive sea captain maroons a mutinous mutineer on an arid sea-girt rock girt by the sea and wholly arid, were it not that Nestingbird concerns himself solely with Dobson's hoisting by his own petard, and has not a word to say about Blodgett. So it may be that, in spite of the weediness of my own scribblings, they can yet fill a gap in the record.

Or, to be more precise, gaps, for as I have said, in Blodgett's case we are dealing with repeated and innumerable hoistings. The puzzle, of course, is that each and every one of these hoistings was upon someone, or something, else's petard, and the question that cries to high heaven for an answer is: how did Blodgett manage to get himself into so many scrapes, and each scrape so similar? 

One would have thought that, after being hoist upon the petard of an undertaker's mute at an impressionable age, he might have learned something. At the very least, he might have learned to avoid undertakers' mutes with petards. But it was not so. Barely a fortnight after that first hoisting, we find the young Blodgett once again hanging around in the vicinity of a funeral parlour, idly tugging at his incipient goatee, and dressed flamboyantly in cerise and dandelion yellow. He is lurking, inasmuch as one can lurk in cerise and dandelion yellow, in a fetid alleyway at the back of the funeral parlour, a parlour owned by an old family firm of funeral directors founded by Ferenc Fafflefoff in the eighteen-fifties. And it was upon the petard of a Fafflefoffian mute that Blodgett was hoist at two o' clock on that September afternoon, an afternoon of squalls and drizzle and abnormal bird phenomena. Not until three-fifteen did he manage to clamber from the petard and descend to the pavement, a picture of befuddlement and the laughing-stock of a gang of Fafflefoff employees, mute and otherwise, who had gathered to witness his hoisting.

Over the next several years, Blodgett was to be hoist on the petards of Dutchmen, tugboat captains, spinettists, mezzotintists, old cloth of gold dustup pan-pot men, squirrel stranglers, shove ha'penny maestros, indentured and goitred peasants, shifty knaves, chunky pockers, Marina Warner readers, fudgers, beanpoles, harum scarum tidewater mappers, dishcloth makers, farmyard freaks and sundry other petardists. Time and again, the hapless Blodgett made the same mistakes, fell into the same traps, blundered into the same emblunderments. The only person who seemed to see anything odd about this was Blodgett's mother, a ghostly white speck of a woman, who by turns remonstrated with him, sobbed, laughed, hid away from him in crannies, prodded his head with surgical implements, sent him into the mountains, and tried to marry him off to a foreign contessa. Blodgett himself just continued with his hoistings, on an almost daily basis.

Things levelled off eventually, with no more than one or two hoistings a year, particularly after his mother's death. Blodgett did not attend her funeral, which was organised by the Fafflefoffs, with many ribbons and a horse, but without an undertaker's mute. Tim, the Mute of the Day, was due to lead the procession through the hopeless rain-soaked streets of Blodgett's mother's horrible home town, but an hour before the coffin was shoved on to the funeral cart he was hoist, not by his own petard, but by Blodgett's. As far as we know, this is the only evidence we have that Blodgett had a petard of his own, and it remains an inexplicable mystery why he was never hoist upon it himself. Such are the perplexities of the human comedy.