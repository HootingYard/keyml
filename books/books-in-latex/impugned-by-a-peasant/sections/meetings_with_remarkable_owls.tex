\chapter{Meetings With Remarkable Owls}

DOBDON'S PAMPHLET \emph{MEETINGS With Remarkable Owls} (out of print) is a curious work. Ostensibly, it is a simple account of a walk he took through the owl sanctuary at Scroonhoonpooge, and of the owls he came across there. Given the unfathomable depth of his ornithological ignorance, one is tempted to suggest that the pamphleteer only knew the birds he 'met' were owls because of the big neon signage at the gate of the sanctuary.

More remarkable than the owls themselves, surely, is the fact that Dobson was able to get anywhere near them in the first place. Ever since the so-called Inexplicably Spooky Events that centred on Scroonhoonpooge Farmyard, the entire area had been cordoned off by a massive security fence patrolled by wolves and wild hogs. There had always been talk of the eerie barn and the mutant albino hens and the disturbing well, to say nothing of the farmyard itself, but after what happened on that wild and windy October weekend, so great was the terror in the surrounding villages that the fence was erected overnight, and the wolves and wild hogs let loose around the perimeter.

Dobson says nothing of this. We are asked to believe that he was out and about pounding the countryside one day when he found himself at the gate of the owl sanctuary and decided to investigate. This cannot be right. To get to the gate, he would first have had to find a way through the security fence without being savaged by wolves or wild hogs, then have had to cross the perilous bogs, avoid the piano wire strung across the pathways, clamber up the impossibly steep sludge banks, find his way through the mist-enshrouded field riddled with concealed pits in which killer spiders lay in wait, and pass through the notorious spinney of poisonous trees. Even had he accomplished all that, he would somehow have had to persuade the sentries at the owl sanctuary gate that he was a bona fide visitor, or they would have shot him on the spot and buried his corpse where it would never, ever be found. The sentries were hand-picked, undergoing rigorous psychological testing to flush out any who had a less than fanatical protective instinct towards owls.

Dobson was not a particularly boastful man, but he did have an operatic diva's sense of drama, and it seems scarcely credible that he would let pass the opportunity to prattle on about so death-defying a journey. So we must be grateful for the research done by indefatigable Dobsonist Ted Cack, whose recently published paper suggests that some weird properties in the atmosphere around Scroonhoonpooge Farmyard may have actually modified Dobson's brain, one such modification being a complete wiping clean of his memory between eating a choc ice at the ramshackle kiosk adjacent to Sawdust Bridge and arrival at the gate of the owl sanctuary three days later.

Some traditionalists have had harsh words to say about Ted Cack. After all, he made his name as a young firebrand with a deliberately provocative book arguing that Dobson was not the true author of the \emph{Bilgewater Elegies} and that the pamphleteer had never set foot in Winnipeg, let alone worked there as a janitor in an evaporated milk factory. These were, and are, preposterous theories, and Cack did himself no favours with his shoddy scholarship, cavalier approach to source material, and pomposity. Yet with his Anthony Burgess hairstyle, hornrim glasses, and barking voice he was a natural for television chatshows, and even the crustiest Dobsonists still speak in awe of his legendary appearance on \emph{Russell Harty Plus}. TV critic Loopy Sebag wrote at the time that 'Ted Cack, with his Anthony Burgess hairstyle, hornrim glasses, and barking voice, is the best thing I have ever seen on television, apart from \emph{It's A Knockout}'.

In his attempt to unravel what happened to Dobson on the day of his visit to the owl sanctuary, Ted Cack put himself in the pamphleteer's sturdy Hungarian Flying Officer's boots, and recreated the journey. Of course, Scroonhoonpooge is much changed. The whole area around the farmyard has been flattened, and there is no longer any sign of the eerie barn or the disturbing well or the albino hens or indeed of the owl sanctuary. In their place stands a derelict and abandoned shopping precinct in which feral beasts and teenpersons cavort and carouse. Only a branch of the plumbing chain Spigots R Us remains open, and its stock is covered in dust and breadcrumbs. Characteristically, Ted Cack was undeterred. He had read a lot of books about psychogeography, and though he did not really understand what he read, he was determined to pretend to be the pamphleteer in that place at that time on that day so many years ago, so much so that he prepared by eating a breakfast of bloaters and wearing a grubby pair of trousers. And, just as the painter Oskar Kokoschka had a life-size rag doll made to replace his lost love Alma Mahler, Ted Cack created a simulacrum of Marigold Chew using string and wool and scrunched-up dishcloths, and waved it goodbye as he crashed out of the door on his way to Sawdust Bridge.

The crucial paragraph in his research paper is this:

\emph{There I stood,} he wrote, \emph{in a puddle outside a boarded-up milk bar where once had stood the gate of the Scroonhoonpooge Owl Sanctuary. I had absolutely no idea how I got here. It was as if my brain had been modified in some sinister way and my memory wiped clean. This leads me to the irrefutable conclusion that exactly the same thing happened to Dobson, and that is why he never wrote about his perilous journey in his pamphlet} Meetings With Remarkable Owls \emph{(out of print). What I do not yet know is how permanent this brain modification will prove to be. God help me.}


I cannot see any holes in this argument whatsoever, so I am prepared to state that Ted Cack, pompous and irritating as he may be, has solved one of the enduring mysteries of the pamphleteer's career.

As for the pamphlet itself, as I said, it is a curious work. Trudging through the owl sanctuary, Dobson from time to time comes across an owl perched upon the branch of a tree. He attempts, first, to describe it, and this is where his lack of ornithological knowledge lets him down. Each description consists almost entirely of the words \emph{head}, \emph{beak}, \emph{wings}, \emph{big round eyes}, \emph{talons}, and \emph{hooting sound} in various combinations. But it is the second part of each 'meeting' to which we turn, wherein Dobson tries to, as he puts it, 'commune with the owls'. He hoots at them. He flaps his arms as if they are wings. He pounces upon a squirrel or a fieldmouse and savages it and swallows it. He hoots again.

\emph{I am Dobson,} he writes, \emph{and for today at least, I am become an owl.}

It is, I think, not the owls which are remarkable in this instance, but Dobson himself.