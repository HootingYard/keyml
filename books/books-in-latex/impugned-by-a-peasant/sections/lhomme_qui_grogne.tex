\chapter{L'Homme Qui Grogne}

L'HOMME QUI GROGNE was, and possibly still is, the French counterpart of the Grunty Man. History tells us that he was active in the countryside around Avignon during the period when it was the seat of the Papacy, that is from 1309 to 1377, between the reigns of Pope Clement V and Pope Gregory XI. There are some mischievous wags who claim this as evidence that the Grunty Man is a practising Roman Catholic, but I think we may safely turn our noses up at them, sniffily. It is of course the case that for as long as anybody can remember the Vatican City at Rome has been periodically menaced by the so-called l'uomo grugnito, and I will have much to say on that at a future date, if you are good.

L'homme qui grogne is said to be hairier, huger, and gruntier than the Grunty Man, and thus more terrifying. He made regular incursions, by nightfall, into Avignon itself, but never managed to scale the immense eighteen-feet thick walls of the Gothic Palais des Papes, and so was never able to carry off one of the Popes, or his nuncios, back to his nest in the woods. You raise your eyebrows at my use of the word nest, but it is carefully chosen, for one of the chief differences between the Grunty Man and l'homme qui grogne is that the latter lurked not in a lair, but in a series of nests built of twigs and branches and forest debris high in trees. When exhausted from his countryside predations, or just in need of a bit of peace and quiet, l'homme qui grogne would clamber in his ungainly and grunty way up the trunk of a mighty hornbeam or cedar and flop into the nest he had made there. It must have angered him that, with all his climbing practice, the Palais itself remained impregnable. No doubt that is why the peasantry told tales of loud and frustrated grunting noises being bellowed from treetops around Avignon. Sophisticated cityfolk sneered at their rustic neighbours for such stories, but we hear an echo of them in, for example, the \emph{Ballades et Bagatelles} of the fourteenth century \"{u}berminstrel Lothar Pangue. Dennis Beerpint's grandmother made prose translations of some of Pangue's pieces, and in one of them we find this:

\emph{A woodsman ran screaming from the woods. He had lost his cap and his hair was dishevelled. He threw himself into the river and swam downstream until he reached the village of distressed mumbling. There he was hauled out of the water by a tavern keeper who put smelling salts under his nose. 'Pray to God in heaven!' shouted the capless woodsman, 'For I have heard loud and frustrated grunting noises from atop a gigantic hornbeam and I am too frightened ever to go back into the woods! How am I to feed my children?' The kindly tavern keeper handed the woodsman a mop and told him he would give him a sou if he swabbed the filthy floor of the tavern. And every day he swabbed the floor he would get another sou. And a year passed, and the woodsman had enough sous to feed his scrawny children with slops and gruel. He thanked the tavern keeper and jumped into the river to swim upstream back to his horrible cottage bordering the woods. But the weight of the sous in his pippy bag dragged him under, and he drowned, and his children starved.}

It is a typically Panguesque fable, ending in horror and ruin and death, and it seems clear to us that the source of the woodsman's fright is l'homme qui grogne. There is a sense in which we wish Lothar Pangue was explicit about this, perhaps having the French Grunty Man lumbering out of the woods at the end of the tale and gobbling up the defenceless children in the cottage. But perhaps Pangue was mindful of a rival set of legends about l'homme qui grogne, in which, far from being a grunting ogre of terror, he is a tragic figure, a huge and hairy lumbering monster who thinks he is a little sparrow. In this tradition, his nesting habits are explained, but the constant attempts to clamber over the walls of the Palais des Papes are ignored. As a sparrow, of course, or rather, in the delusion of sparrowdom, l'homme qui grogne does not feed on human flesh, and thus would have no motive for attacking the woodsman's orphaned tinies in the cottage.

Several researchers have tried to tie the two sets of legends together. The stumbling block is always the dilemma of why a grunty man which believes itself to be a bird would be so desperate to see the Pope. Hattie Meldrum's paper entitled \emph{The Giant Savage Catholic Flightless Grunting Sparrow Theory} is bogged down by far too many footnotes to forward a convincing argument, and in any case her witterings were demolished by Tob during a television chatshow appearance. Sadly, Tob is better at crushing other's reputations than advancing his own peculiar lines of thought, and to date he has published nothing.

After the Papacy returned to Rome in the Great Schism, stories about l'homme qui grogne, in either of his incarnations, became fewer, until gradually he was utterly forgotten. Then suddenly, early in the twentieth century, he reappeared, blamed for a series of railway accidents in and around Avignon. This latterday l'homme qui grogne is yet another variation on the legend, still hairy and huge and grunty, still terrifying, but now waylaying steam trains as they putter along French rural branch lines. There is no suggestion that he thinks himself a sparrow, nor any kind of bird at all. But neither, according to the tales, does he threaten children, except inadvertently, should they be railway passengers. Now, l'homme qui grogne is impelled by a ravening hunger, a hunger that can only be sated by shovelling great pawfuls of burning coal down his gullet. And that is what he does, or did, in the last century. Every day. In and around Avignon. Grunting.