\chapter{Lars Porsena Of Clusium}

LARS PORSENA OF Clusium, by the Nine Gods he swore that the great house of Tarquin should suffer wrong no more. Over in Rome, Lucius Tarquinius Superbus had been overthrown, and he asked Lars, as a fellow Etruscan, for help. Lars thought about it for a bit, and it was when he decided to march to Tarquinius' aid that he did the sweary bit with the Nine Gods. That took a good deal of time, as some among the Gods demanded that when they were sworn by, the swearing had to be an elaborate invocation of rolling phrases, complex rhymes, and repetitive beseeching. Lars Porsena was well-prepared, taking a packed lunch and a big flask filled with a foamy hallucinogenic potation up into the Etruscan hills where he planned to do his swearing.

There has been some debate about the precise identities of the Nine Gods. E Cobham Brewer has them as Juno, Minerva and Tinia, or Tin, or Tina, the three chief Etruscan Gods, joined by Vulcan, Mars, Saturn, Hercules, Summanus, and Vedius. But his list finds no place for such exciting Etruscan deities as Catha and Usil, Selvans, Turan and Laran, nor Thalna, Turms and Fufluns, sometimes known as Puphluns. It seems scarcely credible that a king like Lars Porsena would leave Fufluns out of his swearing on a hillside. We might want to consider the alternative godly roll-call given by Pebblehead in his bestselling paperback \emph{Lars!}, where he gives pride of place to Tina and Fufluns, and chucks in seven others mentioned above. It is true that his book is a novel rather than a history, and that he veers off into a subplot about Tina and Fufluns canoodling in the Etruscan forests, but Pebblehead has studied these things and has the benefit of a number of scholarly works published since Brewer's day, including Dobson's pamphlet \emph{The Sane Person's Guide To Swearing By The Etruscan Gods} (out of print).

So there was Lars, a few days before he set out for Rome, up in the hills under a louring sky. He ate some bite-size cottage pie-style snacky chunks and washed them down with several gulps from his flask, ensuring that his brain underwent preliminary dislodgement. Then he gathered some sticks and tied to each stick a colourful ribbon he had brought with him in his kingly Etruscan pippy bag, and he poked the sticks into the hillside muck to form a magick pattern, nine sticks in all, one for each God. He took a few more swigs from his flask, further shattering his reason, and then he sprawled in front of the stick tied with a beige ribbon, representing the God Usil, and began screaming his head off.

'Usil, Usil, Usil!' he bawled, 'Ooooo! Sil! Ooooo! Sil! Grant me the will to kill, Usil! Let me not dilly dally nor be ill, Usil! If I catch a chill, Usil, up in these hills, give me some pills, Usil! Oooo! Sil!'

And so it went on, for hours, with an occasional pause for more foaming hallucinogenic potation from the flask, until Lars Porsena was completely cracked and exhausted. The God Usil let it be known that it was satisfied with the king's swearing by sending a shower of sparks to dance around his head and half-blind him. Lars Porsena fumbled about, untying the ribbon from the Usil stick, and burning both the ribbon and the stick, and stamping unsteadily upon the embers, and he ate another bite-size cottage pie-style snacky chunk and gulped from his flask, and then he took a nap. One God down, eight more to swear by.

We shall not bother to run through in detail the other swearings, although it has to be said that when it was Fufluns' turn Lars Porsena outdid himself. It took the best part of a day to complete what was the sweariest of the swearings by any stretch of the imagination. So wild and loud and crazed did the king become that he attracted the attention of a little knot of Etruscan peasants who were heading down the hillside after a hike. They recognised Lars Porsena by his kingly garb and were shocked to see him in so demented a state, alternately screeching fantastic ululations at a stick in the ground and shovelling mouthfuls of soil down his gob.

'One wonders what will become of Clusium, ruled by such a king,' said one peasant.

'I fear that it may be swallowed up by the nascent Roman republic and vanish from history,' said another peasant.

The third peasant in the knot chivvied his colleagues to continue down the hillside into downtown Clusium so that they were home in time for their Etruscan supper.

There was no such comfort for Lars Porsena. He still had two more Gods to swear by, and, having eaten the last of his bite-size cottage pie-style snacky chunks, had to grub about in the muck for barely edible roots before taking his next nap. By now, of course, his brain had been bent and cranked to such an extent by his potation, of which much still remained in his huge flask, that his naps were accompanied by strange and terrible dreams. He dreamed he was a pair of ragged claws scuttling across the floors of silent seas. He dreamed he saw his head, grown slightly bald, brought in upon a platter. He dreamed he was in rats' alley where the dead men lost their bones. And he dreamed twit twit twit jug jug jug jug jug jug.

When he woke up, in the hills, it was raining. Hard fat drops of Etruscan rainfall hammered upon the king's head. It did not take him long to swear by Turms, for Turms was an easily-assuaged God. Lars Porsena remembered with brilliant clarity the words he had learned as an infant at his Royal Etruscan Faith-Based Community Education Hub. He had had an excellent teacher, a beardy robed figure with a squeaky voice and a genius for arresting similes. 'The God Turms,' he had said, 'Is like a silken girl bringing sherbet and at the same time like a camel man cursing and grumbling.' Lars had never forgotten that, it had been beaten into him with a stick, a stick rather bigger than the stick he now burned upon the hillside together with the ribbon he had unfastened from it. He had one more God to go, and when all nine sticks and their ribbons had been burned to nothingness he would be ready to follow the peasants' trail down the hillside and march off in aid of Lucius Tarquinius Superbus.

As he glugged another draught of foamy hallucinogenic potation, Lars wondered if, in ages to come, he too might be known as Superbus. Lars Porsena Superbus. Or even Lars Porsena Ubersuperbus. It had a ring to it. He imagined that there might come a time when a future princeling, preparing to wage war upon a foe, might come to these very same hills and swear by him, by Lars, and burn a beribboned stick in his name, and be thus emboldened and blessed. It was not beyond the bounds of Etruscan possibility that he might become a God. Would Clusium be a fit stamping ground for a deity? He would have to ensure when he made the transformation from mortal to divine that his bodily remnants were placed in an elaborate tomb in or under the city he ruled, with a fifteen-metre high rectangular base and sides ninety metres long, adorned by pyramids and massive bells.

He polished off the sweary stuff with the final God, burned the final ribbon and the final stick, and emptied what was left in the huge flask down his throat. And then Lars Porsena stumbled away down the hillside, rain-battered and brain-bedizened, leaving behind him a pile of ashes. Soon he would hasten to Rome, and come face to face with heroic one-eyed Horatius Cocles, and make history.

Curiously, in his bestselling paperback \emph{Lars!}, Pebblehead has absolutely nothing to say about this history. The novel ends with Tina and Fufluns doing goddy things in the ethereal realm, the eponymous king quite forgotten, and not remotely Superbus.