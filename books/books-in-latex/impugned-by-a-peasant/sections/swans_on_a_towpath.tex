\chapter{Swans On A Towpath}

CLUTCHING A BAG of feathers in his sweaty fist, the nameless miscreant stalked along the towpath of the canal. It was a paper bag. It was a stinking canal, into which thoughtless gits dumped such debris as toffee apple wrappers and cartons and bent bits of metal. The feathers in the miscreant's bag were the feathers of swans. He had not plucked them himself, but stumbled upon them, a pile of swan feathers swept into a heap by the side of the canal a mile or so south, where there had been a short-lived swan war earlier in the day. The feathers had been scattered, and swept together by a broom wielded by the lock keeper. The lock keeper was the father of the miscreant, the long lost father. He did not know the miscreant was his son, nor did the miscreant know the lock keeper was his father. In physical appearance, there was nothing to connect them. The one was improbably tall, and loose-jointed, and lantern-jawed, like a giant, the other chubby and squat. Morally, too, a chasm lay between them, for the lock keeper was civic-minded and held down a steady job and used his broom to sweep up that which was scattered alongside the canal, be it the feathers of swans or the discarded wrappings of toffee apples and other confectionery. The miscreant, by contrast, was a miscreant, who would, if given a broom, use it not to sweep up rubbish but to beat about the head of someone weaker than he whom he could rob.

The miscreant had not stuffed the feathers into a paper bag for any motive of beautifying the canalside. He had simply taken them, as miscreants will take, opportunistically, anything they can take. He saw the swan feathers in a pile and thought to himself that he could sell them to a dishonest milliner of his acquaintance. This milliner, he knew, would decorate his hats with all sorts of gaudies of dubious provenance. His customers never asked questions, not even when their heads grew boils and sores because their hat harboured toxic elements the milliner was too careless to decontaminate. A ruffian might sell him a box of beads for cash, and the milliner would stitch the beads into a hat, and sell the hat, and neither know nor care that the box of beads had been robbed from a hazardous waste compound.

So the miscreant was confident he could sell the bag of feathers to the dishonest milliner, even though the swan war was occasioned by disturbance in the brains of the swans caused by weird turquoise sludge through which they had glid, so gracefully, some hours earlier. The sludge was almost unbelievably poisonous. It had been dumped in the canal and in nearby ponds by boffins, bad boffins, who were engaged in secret and fiendish experiments in their lab. One of these boffins was the cousin of the lock keeper. There had been much interbreeding, over the course of untallied generations, in this part of the land. The milliner shared a bloodline with the boffin too, and thus with the lock keeper, and thus with the miscreant.

There was even talk, in the tavern, of squalid couplings with swans, in the past, when people knew no better. 'There's been traffic with beings aquatic,' an oldster might mutter, staring gloomily into his tankard. Only the most observant might note the feathers visible when he hitched up his trousers, or the way he waddled slightly as he left the tavern, later, heading into the night to none knew where. And was that a faint splash that could be heard, not so distant, within minutes of his leaving?

It is tempting, when writing of swans and boffins and canals, to regurgitate great chunks of prose written on these topics by the acknowledged masters, men like Dobson and Definzi and women like Hattie Meldrum. It is a temptation which must be fought and defeated, partly in respect for the copyright laws and partly from sheer pomposity. One must breathe through one's nose, in an actorly way, and make a world with words of one's own. The fact that that last sentence is a direct quotation from Definzi is neither here nor there.

The toxic sludge dumped by the boffin was a by-product of the experiments going on in the lab. The purpose of those experiments was monstrous, and related in some wise to the matter muttered by the oldster in the tavern. Not all boffins are miscreants, by any means, but some are, and the bad boffin whose daily duty it was to roam the canalside and the ponds pumping hazardous sludge into the waters was one such. He was impudent in his criminality, not caring a jot if it was witnessed by innocents. He was of the view that all souls are besmirched, that guilt gnaws away at the innards of everybody. In this part of the land, with its history of sordid breeding, he may have been correct.

Just as not all boffins are miscreants, nor are all miscreants boffins. For example, the nameless miscreant with his paper bag of feathers was no boffin. He had a tiny brain, and one which did not always work properly, not due to contact with toxic sludge, but rather because of repeated blows to the head received from other miscreants, once upon a time, in his fighting days. Had he been a boffin of any kind, he might have devised a way of resolving the dilemma that faced him now, as he stalked along the canal towpath. For his way was blocked by a gaggle of swans. Recently at war, the swans were peppy with adrenalin, their aggression by no means diminished. A truce with each other struck, they turned their cold horrifying eyes on the miscreant, whose approach was impertinent.

How one wishes it were meet to copy out a screed by Hattie Meldrum here. In her magnificent compendium of violent swan anecdotage, she relates dozens of instances not unlike this confrontation. Some of her tales even take place alongside canals, albeit that in her world the canals are clean and well tended, the lock keepers need not go brooming about, and miscreants are few and far between. It is true that more than one bad boffin hoves into view in her five hundred pages, but their sins are not of a sludge-dumping character. The badness of Hattie's boffins is limited to swan cruelty, or one should say attempted swan cruelty, with one exception, and in that case the boffin is a blasphemer.

Had our paper bag-clutching miscreant known that, deep in his past, there had been several occurrences of congress with swans and other waterfowl, he might have burrowed into the nooks and crevices of his tiny brain to deploy some atavistic sound or gesture with the effect of placating the swans blocking his path. But he knew nothing of his past. He did not even know his own father, who at that very moment was coming towards the swans from the rear, having swept up, with his broom, a couple of tin cans and stray toggles torn from a duffel coat. The lock keeper was returning to his lock keeper's hut for tea and toast, his broom over his shoulder, his lips pursed in the whistling of a happy, happy tune.

So. We have a gang of swans, rancorous swans, their innate savagery compounded by the effects of the traces of toxic sludge still present in their systems, their malevolence focused upon a hapless miscreant carrying a paper bag of swan feathers destined for a dishonest milliner. Unseen by the swans, because behind them, and not one of them looks back, comes a lanky man with a broom. That there will be violence, pecking, bashing, blood, screeching, laceration, splashing, all in the dapple of sunlight by the side of the canal, along its towpath, is, it would appear, inevitable. We expect the swans to attack the miscreant, the lock keeper to attack the swans, the outcome of course being beyond our wit to foresee. Oblivious to their parts, the boffin is siphoning sludge into a pannier, the milliner is sewing infected buttons on to a cap. For make no mistake, these two cannot be forgotten in the telling of the tale. Hattie Meldrum, for one, would have done more than sketch them in. You would get potted biographies as likely as not.

There is a moment, in all anecdotage, where we can stop, as if freezing a frame in a motion picture. Some say events are foretold in the stars. Even if we disparage such twaddle, it remains the case that sometimes circumstances are such that we are convinced we know what is about to happen. Ah, but we forget. We forget, in the present instance, that there is another fellow who plays a part. At the very moment the swans are about to launch their attack on the miscreant, the same moment the lock keeper takes his broom from his shoulder ready to attack the swans\ldots  there is a gurgling in the canal. Bubbles disturb the filthy surface. And with a mighty splash, emerging from the depths comes the oldster, the mutter-man from the tavern, now transformed, half man, half swan, gigantic, and he, it, enwraps the lock keeper and the miscreant and all the swans within the folds of its enormous beating wings. It holds them close, close enough almost to suffocate them. But it does not suffocate them. It holds them whole.