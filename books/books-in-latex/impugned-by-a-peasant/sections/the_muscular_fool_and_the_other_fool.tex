\chapter{The Muscular Fool And The Other Fool}

A FOOL DUG a hole in the ground with a spade. When he had dug deep enough, the fool put aside the spade and sat down in the hole, deep enough in this instance meaning that from his sitting position his head was below ground level. The ground itself was fallow. We should remember, even if the fool did not, that 'The lark's shrill fife may come / At the daybreak from the fallow'. So, at least, was the assertion of Sir Walter Scott in \emph{The Lady Of The Lake}. He goes on to say that 'the bittern sound[s] his drum / Booming from the sedgy shallow', but there were no shallows, sedgy or otherwise, in this fallow where the fool sat in a hole he'd dug, nor any bitterns to boom. Scott brought a curse upon himself by making disparaging remarks about the Muggletonians in one of his novels\footnote{NOTE : In \emph{Woodstock, or The Cavalier} (1826), a character named Tonkins meets a violent end. Scott regrets that 'his brains had not been beaten out in his cradle' to prevent him growing up into 'one of those Muggletonians'. For this he was cursed by Robert Wallis, a Muggletonian from Islington.}, but the fool had not been cursed. He was simply a fool.

It is pointless to ask of such a person, 'why have you done what you have done?' Either he will not reply, or, if he does, he will dizzy your brain with his explanation. You might understand the individual words he shouts or mutters at you, but you will be hard pressed to make any sense of them when you join them together. That is one of the things about fools, they drive a stake through the heart of reason. I used to be a fool, so I know that only too well.

I was not the kind of fool to dig holes in the ground with a spade, for my foolishness led me down other pathways. I could often be found in department stores, wandering from one floor to another, via the escalators, up and down, all day long, never making a purchase, followed about by in-house detectives, chanting. I mean that I was chanting, not the detectives. The detectives had no time to chant, they were too busy keeping track of me.

They would have had no trouble tracking the fool with the spade, for he had dug his hole in the ground and now he was sat in it, quite still. The in-house department store detectives could have gathered in a ring at the rim of the hole and kept their eyes on the fool for hours. But they would not be likely to do so. Being in-house types they did not do any detecting outwith the precincts of the department store itself, not unless one of them went rogue, and was overzealous in his duties. That had been known to happen. But not on this occasion.

Fools come in all shapes and sizes. The fool in the hole in the fallow was muscular, which ought not be a surprise when you consider that it takes some strength to dig a hole in the ground deep enough to sit in so that one's head is hidden from view. You or I might be panting and shaking after such exertion, but the fool sat in his newly dug hole looking for all the world as if the most energetic thing he had done all day was to stir a spoon in a bowl. That was what he had done first thing, before marching across the fallow with a spade over his shoulder. He had poured porridge into a bowl, like a bear in a fairy tale, and then stirred it with a spoon, and then spooned it bit by bit into his mouth and swallowed it, and then he licked the bowl clean. These were not the actions of a fool by any means, but then most fools have moments, even whole mornings, of lucidity. Yet as soon as he had prepared and eaten a sensible breakfast, the fool reverted to inexplicably foolish behaviour, and went out and dug a hole and sat in it.

When I was a fool, I too usually ate a proper breakfast at the start of the day, though in my case it was rarely porridge. I had been traumatised by the version of the fairy tale of the three bears told to me as a bedtime story by my Ma. In Miss Eleanor Mure's telling of 1832, it is not Goldilocks who enters the bears' cottage while they are out and about, but an ill-tempered old woman. When the bears come home, they first try to burn her, then to drown her, before finally chucking her aloft on St Paul's churchyard steeple, upon which she is impaled. Thus as an infant there was fixed within my little brain the association of porridge with impalement, and I became keen on cornflakes. Even at the peak of my foolishness, I seldom set out without a stomach full of Mr Kellogg's finest. When milk was scarce, as it often was, given the pitiable state of the cows where I grew up, I would just shovel the cornflakes down my gullet straight from the carton.

It may be the case that the fool in the hole sometimes had to make do with dry oats for his breakfast, if he too experienced problems obtaining uncontaminated milk. That would depend upon the cows in his locality, and whether they were hale or sickly. Irrespective of their health, and the potability or otherwise of their milk, it could happen that a blundering cow might roam into the fallow field and topple into the hole dug by the fool. If the fool was still sitting in his hole, he would find himself underneath a panic-stricken and possibly injured cow. Being a muscular fool, he would probably be able to push the cow off him and to climb out of the hole. One might hope that pangs of compassion would burst through his foolishness, at least temporarily, and that he would rush away to find the farmer or a veterinary surgeon, but there can be no guarantee of that. Fools can be so well wrapped up in their own foolishness that their behaviour appears ruthless and despicable. So it may be that, his hole now being occupied by a cow, the fool would simply retrieve his spade from where he chucked it and dig himself a second hole, in which, once dug, he would sit, in the fallow where no bitterns boomed but a lark would fife at daybreak, shrilly. If the fool was still sitting in his hole come the dawn, having spent the night under glittering stars, the lark's fifing would almost certainly awaken him.

Now, consider the situation. We have already ascertained that this fool is the sort of fool who thrives on a proper breakfast of porridge. You cannot make porridge while sat in a hole dug in the ground, even if you are a brainbox rather than a fool. So the fool, intent upon his breakfast, would clamber up out of his hole and begin mincing across the field towards wherever it was he could rely upon the makings of porridge. One can be both muscular and of the mincing sort, whether fool or no. Again, we may hope that the fool's heart would be stirred at the sight of the stricken cow in the adjoining hole, but with a stomach clamouring for porridge the fool is not likely even to notice the other hole or the cow within it. The cow may bellow, but the fool, in the extremity of his foolishness, will misconstrue the bellowing as something else, as, say, the fifing of a mutant lark, or a factory hooter, for there is a factory over yonder, for the making of fireworks and other explosive devices, and its workers are summoned at dawn by means of a hooter. To a fool's ears, all sounds can become confused.

On this bright morning, as the fool minced across the fallow towards his porridge, a cherry-cheeked farmer was in his barn counting his cows. He counted them thrice, just to make sure he was correct in his apprehension that one was missing. And then he resolved to tramp his fields until he found his cow. In the distance, the bells of St Bibblybibdib's clanged, and in the fallow, the fool and the farmer met, the one mincing from the east and the other tramping from the west. The farmer asked the fool if he had seen his cow. The fool replied, but as we have seen, though his individual words were coherent, they made no sense when joined together. There can be danger when fuddleheadedness shares space in the brain with a hot temper, and the farmer was hot-tempered, as was betrayed by his cherry cheeks. They were cherry because his blood often boiled. Little things enraged him, from misdirected farm postage to creaking wheelbarrows. Fretting about his cow, and stupefied by the fool's response to his simple question, he lashed out at the fool with his big hairy farmer's fists. Though muscular, and quite able to triumph in any fight he got into, the fool was disadvantaged by the sudden ferocity of the farmer's onslaught, and he toppled to the ground. In so doing, he clonked his head on a large pebble. Miraculously, the clonk caused an enjugglement of bits inside his brain, and he became instantly, and permanently, lucid, and no more a fool than you or I. He jumped to his feet, and shook the farmer's hairy hand, and pointed to where the cow languished in a hole dug in the fallow, and promised the farmer he would help to rescue it as soon as he had had his breakfast porridge, and away he minced with a clear head and bright eyes.

My own transformation from being a fool to no longer being a fool had nothing in common with this tale. I suffered no clonk on the head, nor was I mincing across the fallow as church bells clanged. I had dug no holes in the ground, and there were no cows to be seen. I was in a city far away when my foolishness evaporated, sprawled on a divan, twitching and shattered, with a belly full of cornflakes and milk that, had I but known it, was contaminated. It was the seething microscopic beings lurking in the milk that burrowed their way up into my brain and nibbled away the weird bits that made me a fool. When their nibbling was done, the tiny, tiny beings burrowed through the top of my skull and were smothered by my bouffant. I picked out the dead shrivelled things with tweezers, and put them in a jar, and I put the jar on my mantelpiece, and there it stands. It can do no other. Jars stay put, once you have placed them where you want them. I only learned that lesson when I was no longer a fool.